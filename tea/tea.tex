\documentclass[a4paper, platex, dvipdfmx]{jsarticle}
\title{茶の本(第一章 人情の碗)}
\author{岡倉 天心}
\date{1929年3月10日(邦訳発行日)}
\begin{document}
\maketitle

茶は薬用として始まり後飲料となる。
シナにおいては八世紀に高雅な遊びの一つとして詩歌の域に達した。
十五世紀に至り日本はこれを高めて一種の審美的宗教、すなわち茶道にまで進めた。
茶道は日常生活の俗事の中に存する美しきものを崇拝することに基づく一種の儀式であって、
純粋と調和、相互愛の神秘、社会秩序のローマン主義を諄々と教えるものである。
茶道の要義は「不完全なもの」を崇拝するにある。いわゆる人生というこの不可解なもののうちに、
何か可能なものを成就しようとするやさしい企てであるから。

茶の原理は普通の意味でいう単なる審美主義ではない。
というのは、倫理、宗教と合して、天人に関するわれわれの
いっさいの見解を表わしているものであるから。それは衛生学である、清潔をきびしく説くから。
それは経済学である、というのは、複雑なぜいたくというよりも
むしろ単純のうちに慰安を教えるから。それは精神幾何学である、
なんとなれば、宇宙に対するわれわれの比例感を定義するから。
それはあらゆるこの道の信者を趣味上の貴族にして、東洋民主主義の真精神を表わしている。

日本が長い間世界から孤立していたのは、自省をする一助となって
茶道の発達に非常に好都合であった。われらの住居、習慣、衣食、
陶漆器、絵画等——文学でさえも——すべてその影響をこうむっている。
いやしくも日本の文化を研究せんとする者は、この影響の存在を無視することはできない。
茶道の影響は貴人の優雅な閨房にも、下賤の者の住み家にも行き渡ってきた。
わが田夫は花を生けることを知り、わが野人も山水を愛でるに至った。
俗に「あの男は茶気がない」という。もし人が、わが身の上におこる
まじめながらの滑稽を知らないならば。また浮世の悲劇にとんじゃくもなく、
浮かれ気分で騒ぐ半可通を「あまり茶気があり過ぎる」と言って非難する。

よその目には、つまらぬことをこのように騒ぎ立てるのが、実に不思議に思われるかもしれぬ。
一杯のお茶でなんという騒ぎだろうというであろうが、考えてみれば、
煎ずるところ人間享楽の茶碗は、いかにも狭いものではないか、
いかにも早く涙であふれるではないか、無辺を求むる渇のとまらぬあまり、
一息に飲みほされるではないか。してみれば、茶碗をいくらもてはやしたとて
とがめだてには及ぶまい。人間はこれよりもまだまだ悪いことをした。
酒の神バッカスを崇拝するのあまり、惜しげもなく奉納をし過ぎた。
軍神マーズの血なまぐさい姿をさえも理想化した。してみれば、
カメリヤの女皇に身をささげ、その祭壇から流れ出る暖かい同情の流れを、
心ゆくばかり楽しんでもよいではないか。象牙色の磁器にもられた液体琥珀の中に、
その道の心得ある人は、孔子の心よき沈黙、老子の奇警、
釈迦牟尼の天上の香にさえ触れることができる。

おのれに存する偉大なるものの小を感ずることのできない人は、
他人に存する小なるものの偉大を見のがしがちである。一般の西洋人は、
茶の湯を見て、東洋の珍奇、稚気をなしている千百の奇癖のまたの例に過ぎないと思って、
袖の下で笑っているであろう。西洋人は、日本が平和な文芸にふけっていた間は、
野蛮国と見なしていたものである。しかるに満州の戦場に
大々的殺戮を行ない始めてから文明国と呼んでいる。
近ごろ武士道——わが兵士に喜び勇んで身を捨てさせる死の術——について盛んに論評されてきた。
しかし茶道にはほとんど注意がひかれていない。
この道はわが生の術を多く説いているものであるが。
もしわれわれが文明国たるためには、血なまぐさい戦争の名誉によらなければならないとするならば、
むしろいつまでも野蛮国に甘んじよう。われわれはわが芸術および理想に対して、
しかるべき尊敬が払われる時期が来るのを喜んで待とう。

いつになったら西洋が東洋を了解するであろう、否、了解しようと努めるであろう。
われわれアジア人はわれわれに関して織り出された事実や想像の妙な話に
しばしば胆を冷やすことがある。われわれは、ねずみや油虫を食べて
生きているのでないとしても、蓮の香を吸って生きていると思われている。
これは、つまらない狂信か、さもなければ見さげ果てた逸楽である。
インドの心霊性を無知といい、シナの謹直を愚鈍といい、日本の愛国心をば
宿命論の結果といってあざけられていた。はなはだしきは、
われわれは神経組織が無感覚なるため、傷や痛みに対して感じが薄いとまで言われていた。

西洋の諸君、われわれを種にどんなことでも言ってお楽しみなさい。アジアは返礼いたします。
まだまだおもしろい種になることはいくらでもあろう、もしわれわれ諸君についてこれまで、
想像したり書いたりしたことがすっかりおわかりになれば。すべて遠きものをば美しと見、
不思議に対して知らず知らず感服し、新しい不分明なものに対しては、
口には出さねど憤るということがそこに含まれている。諸君はこれまで、
うらやましく思うこともできないほど立派な徳を負わされて、あまり美しくて、
とがめることのできないような罪をきせられている。
わが国の昔の文人は——その当時の物知りであった——まあこんなことを言っている。
諸君には着物のどこか見えないところに、毛深いしっぽがあり、
そしてしばしば赤ん坊の細切り料理を食べていると! 否、
われわれは諸君に対してもっと悪いことを考えていた。すなわち諸君は、
地球上で最も実行不可能な人種と思っていた。
というわけは、諸君は決して実行しないことを口では説いているといわれていたから。

かくのごとき誤解はわれわれのうちからすみやかに消え去ってゆく。
商業上の必要に迫られて欧州の国語が、東洋幾多の港に用いられるようになって来た。
アジアの青年は現代的教育を受けるために、西洋の大学に群がってゆく。
われわれの洞察力は、諸君の文化に深く入り込むことはできない。
しかし少なくともわれわれは喜んで学ぼうとしている。
私の同国人のうちには、諸君の習慣や礼儀作法をあまりに多く取り入れた者がある。
こういう人は、こわばったカラや丈の高いシルクハットを得ることが、
諸君の文明を得ることと心得違いをしていたのである。かかる様子ぶりは、
実に哀れむべき嘆かわしいものであるが、ひざまずいて西洋文明に近づこうとする証拠となる。
不幸にして、西洋の態度は東洋を理解するに都合が悪い。
キリスト教の宣教師は与えるために行き、受けようとはしない。諸君の知識は、
もし通りすがりの旅人のあてにならない話に基づくのでなければ、
わが文学の貧弱な翻訳に基づいている。ラフカディオ・ハーンの義侠的ペン、
または『インド生活の組織\footnote{『インド生活の組織』―― The Sister Nivedita 著。}』の
著者のそれが、われわれみずからの感情の松明をもって東洋の闇を明るくすることはまれである。

私はこんなにあけすけに言って、たぶん茶道についての私自身の無知を表わすであろう。
茶道の高雅な精神そのものは、人から期待せられていることだけ言うことを要求する。
しかし私は立派な茶人のつもりで書いているのではない。新旧両世界の誤解によって、
すでに非常な禍をこうむっているのであるから、お互いがよく了解することを助けるために、
いささかなりとも貢献するに弁解の必要はない。二十世紀の初めに、
もしロシアがへりくだって日本をよく了解していたら、血なまぐさい戦争の光景は見ないで済んだであろうに。
東洋の問題をさげすんで度外視すれば、なんという恐ろしい結果が人類に及ぶことであろう。
ヨーロッパの帝国主義は、黄禍のばかげた叫びをあげることを恥じないが、
アジアもまた、白禍の恐るべきをさとるに至るかもしれないということは、わかりかねている。
諸君はわれわれを「あまり茶気があり過ぎる」と笑うかもしれないが、
われわれはまた西洋の諸君には天性「茶気がない」と思うかもしれないではないか。

東西両大陸が互いに奇警な批評を飛ばすことはやめにして、
東西互いに得る利益によって、よし物がわかって来ないとしても、
お互いにやわらかい気持ちになろうではないか。
お互いに違った方面に向かって発展して来ているが、しかし互いに長短相補わない道理はない。
諸君は心の落ちつきを失ってまで膨張発展を遂げた。
われわれは侵略に対しては弱い調和を創造した。
諸君は信ずることができますか、東洋はある点で西洋にまさっているということを!

不思議にも人情は今までのところ茶碗に東西相合している。
茶道は世界的に重んぜられている唯一のアジアの儀式である。
白人はわが宗教道徳を嘲笑した。しかしこの褐色飲料は躊躇もなく受け入れてしまった。
午後の喫茶は、今や西洋の社会における重要な役をつとめている。
盆や茶托の打ち合う微妙な音にも、ねんごろにもてなす婦人の柔らかい絹ずれの音にも、
また、クリームや砂糖を勧められたり断わったりする普通の問答にも、
茶の崇拝は疑いもなく確立しているということがわかる。渋いか甘いか疑わしい煎茶の味は、
客を待つ運命に任せてあきらめる。この一事にも東洋精神が強く現われているということがわかる。

ヨーロッパにおける茶についての最も古い記事は、アラビヤの旅行者の物語にあると言われていて、
八七九年以後広東における主要なる歳入の財源は塩と茶の税であったと述べてある。
マルコポーロは、シナの市舶司が茶税を勝手に増したために、一二八五年免職になったことを記録している。
ヨーロッパ人が、極東についていっそう多く知り始めたのは、実に大発見時代のころである。
十六世紀の終わりにオランダ人は、東洋において灌木の葉からさわやかな飲料が造られることを報じた。
ジオヴァーニ・バティスタ・ラムージオ(一五五九)、エル・アルメイダ(一五七六)、
マフェノ(一五八八)、タレイラ(一六一〇)らの旅行者たちも
また茶のことを述べている\footnote{Paul Kransel 著、Dissertations, Berlin, 1902.}。
一六一〇年に、オランダ東インド会社の船がヨーロッパに初めて茶を輸入した。
一六三六年にはフランスに伝わり、一六三八年にはロシアにまで達した。
英国は一六五〇年これを喜び迎えて、「かの卓絶せる、かつすべての医者の推奨するシナ飲料、
シナ人はこれをチャと呼び、他国民はこれをテイまたはティーと呼ぶ。」と言っていた。

この世のすべてのよい物と同じく、茶の普及もまた反対にあった。
ヘンリー・セイヴィル(一六七八)のような異端者は、茶を飲むことを不潔な習慣として
口をきわめて非難した。ジョウナス・ハンウェイは言った。(茶の説・一七五六)
茶を用いれば男は身のたけ低くなり、みめをそこない、女はその美を失うと。
茶の価の高いために(一ポンド約十五シリング)初めは一般の人の消費を許さなかった。
「歓待饗応用の王室御用品、王侯貴族の贈答用品」として用いられた。
しかしこういう不利な立場にあるにもかかわらず、喫茶は、すばらしい勢いで広まって行った。
十八世紀前半におけるロンドンのコーヒー店は、実際喫茶店となり、
アディソンやスティールのような文士のつどうところとなり、
茶を喫しながらかれらは退屈しのぎをしたものである。
この飲料はまもなく生活の必要品——課税品——となった。これに関連して、
現代の歴史において茶がいかに主要な役を務めているかを思い出す。
アメリカ植民地は圧迫を甘んじて受けていたが、ついに、
茶の重税に堪えかねて人間の忍耐力も尽きてしまった。
アメリカの独立は、ボストン港に茶箱を投じたことに始まる。

茶の味には微妙な魅力があって、人はこれに引きつけられないわけにはゆかない、
またこれを理想化するようになる。西洋の茶人たちは、
茶のかおりとかれらの思想の芳香を混ずるに鈍ではなかった。
茶には酒のような傲慢なところがない。コーヒーのような自覚もなければ、
またココアのような気取った無邪気もない。
一七一一年にすでにスペクテイター紙に次のように言っている。
「それゆえに私は、この私の考えを、毎朝、茶とバタつきパンに
一時間を取っておかれるような、すべての立派な御家庭へ特にお勧めしたいと思います。
そして、どうぞこの新聞を、お茶のしたくの一部分として、
時間を守って出すようにお命じになることを、せつにお勧めいたします。」
サミュエル・ジョンソンはみずからの人物を描いて次のように言っている。
「因業な恥知らずのお茶飲みで、二十年間も食事を薄くするに
ただこの魔力ある植物の振り出しをもってした。そして茶をもって夕べを楽しみ、
茶をもって真夜中を慰め、茶をもって晨を迎えた。」

ほんとうの茶人チャールズ・ラムは、
「ひそかに善を行なって偶然にこれが現われることが何よりの愉快である。」というところに
茶道の真髄を伝えている。というわけは、茶道は美を見いださんがために美を隠す術であり、
現わすことをはばかるようなものをほのめかす術である。この道はおのれに向かって、
落ち着いてしかし充分に笑うけだかい奥義である。従ってヒューマーそのものであり、
悟りの微笑である。すべて真に茶を解する人はこの意味において茶人と言ってもよかろう。
たとえばサッカレー、それからシェイクスピアはもちろん、文芸廃頽期の詩人もまた、
(と言っても、いずれの時か廃頽期でなかろう)物質主義に対する反抗のあまり
いくらか茶道の思想を受け入れた。たぶん今日においてもこの「不完全」を
真摯に静観してこそ、東西相会して互いに慰めることができるであろう。

道教徒はいう、「無始」の始めにおいて「心」と「物」が決死の争闘をした。
ついに大日輪黄帝は闇と地の邪神祝融に打ち勝った。
その巨人は死苦のあまり頭を天涯に打ちつけ、硬玉の青天を粉砕した。
星はその場所を失い、月は夜の寂寞たる天空をあてもなくさまようた。
失望のあまり黄帝は、遠く広く天の修理者を求めた。
捜し求めたかいはあって東方の海から女という女皇、角をいただき竜尾をそなえ、
火の甲冑をまとって燦然たる姿で現われた。その神は不思議な大釜に五色の虹を焼き出し、
シナの天を建て直した。しかしながら、また女は蒼天にある二個の小隙を埋めることを
忘れたと言われている。かくのごとくして愛の二元論が始まった。
すなわち二個の霊は空間を流転してとどまることを知らず、ついに合して始めて完全な宇宙をなす。
人はおのおの希望と平和の天空を新たに建て直さなければならぬ。

現代の人道の天空は、富と権力を得んと争う莫大な努力によって全く粉砕せられている。
世は利己、俗悪の闇に迷っている。知識は心にやましいことをして得られ、
仁は実利のために行なわれている。東西両洋は、立ち騒ぐ海に投げ入れられた二竜のごとく、
人生の宝玉を得ようとすれどそのかいもない。この大荒廃を繕うために再び女を必要とする。
われわれは大権化の出現を待つ。まあ、茶でも一口すすろうではないか。
明るい午後の日は竹林にはえ、泉水はうれしげな音をたて、松籟はわが茶釜に聞こえている。
はかないことを夢に見て、美しい取りとめのないことをあれやこれやと考えようではないか。

\end{document}