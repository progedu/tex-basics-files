\documentclass[a4paper, platex, dvipdfmx]{jsarticle}
\title{学問のすすめ}
\date{1872年2月}
\author{福沢 諭吉}
\begin{document}
\maketitle
\section{初編}
「天は人の上に人を造らず人の下に人を造らず」と言えり。されば天より人を生ずるには、万人は万人みな同じ位にして、生まれながら貴賤上下の差別なく、万物の霊たる身と心との働きをもって天地の間にあるよろずの物を資り、もって衣食住の用を達し、自由自在、互いに人の妨げをなさずしておのおの安楽にこの世を渡らしめ給うの趣意なり。されども今、広くこの人間世界を見渡すに、かしこき人あり、おろかなる人あり、貧しきもあり、富めるもあり、貴人もあり、下人もありて、その有様雲と泥との相違あるに似たるはなんぞや。その次第はなはだ明らかなり。『実語教』に、「人学ばざれば智なし、智なき者は愚人なり」とあり。されば賢人と愚人との別は学ぶと学ばざるとによりてできるものなり。また世の中にむずかしき仕事もあり、やすき仕事もあり。そのむずかしき仕事をする者を身分重き人と名づけ、やすき仕事をする者を身分軽き人という。すべて心を用い、心配する仕事はむずかしくして、手足を用うる力役はやすし。ゆえに医者、学者、政府の役人、または大なる商売をする町人、あまたの奉公人を召し使う大百姓などは、身分重くして貴き者と言うべし。

身分重くして貴ければおのずからその家も富んで、下々の者より見れば及ぶべからざるようなれども、その本を尋ぬればただその人に学問の力あるとなきとによりてその相違もできたるのみにて、天より定めたる約束にあらず。諺にいわく、「天は富貴を人に与えずして、これをその人の働きに与うるものなり」と。されば前にも言えるとおり、人は生まれながらにして貴賤・貧富の別なし。ただ学問を勤めて物事をよく知る者は貴人となり富人となり、無学なる者は貧人となり下人となるなり。

学問とは、ただむずかしき字を知り、解し難き古文を読み、和歌を楽しみ、詩を作るなど、世上に実のなき文学を言うにあらず。これらの文学もおのずから人の心を悦ばしめずいぶん調法なるものなれども、古来、世間の儒者・和学者などの申すよう、さまであがめ貴むべきものにあらず。古来、漢学者に世帯持ちの上手なる者も少なく、和歌をよくして商売に巧者なる町人もまれなり。これがため心ある町人・百姓は、その子の学問に出精するを見て、やがて身代を持ち崩すならんとて親心に心配する者あり。無理ならぬことなり。畢竟その学問の実に遠くして日用の間に合わぬ証拠なり。

されば今、かかる実なき学問はまず次にし、もっぱら勤むべきは人間普通日用に近き実学なり。譬えば、いろは四十七文字を習い、手紙の文言、帳合いの仕方、算盤の稽古、天秤の取扱い等を心得、なおまた進んで学ぶべき箇条ははなはだ多し。地理学とは日本国中はもちろん世界万国の風土道案内なり。究理学とは天地万物の性質を見て、その働きを知る学問なり。歴史とは年代記のくわしきものにて万国古今の有様を詮索する書物なり。経済学とは一身一家の世帯より天下の世帯を説きたるものなり。修身学とは身の行ないを修め、人に交わり、この世を渡るべき天然の道理を述べたるものなり。

これらの学問をするに、いずれも西洋の翻訳書を取り調べ、たいていのことは日本の仮名にて用を便じ、あるいは年少にして文才ある者へは横文字をも読ませ、一科一学も実事を押え、その事につきその物に従い、近く物事の道理を求めて今日の用を達すべきなり。右は人間普通の実学にて、人たる者は貴賤上下の区別なく、みなことごとくたしなむべき心得なれば、この心得ありて後に、士農工商おのおのその分を尽くし、銘々の家業を営み、身も独立し、家も独立し、天下国家も独立すべきなり。

学問をするには分限を知ること肝要なり。人の天然生まれつきは、繋がれず縛られず、一人前の男は男、一人前の女は女にて、自由自在なる者なれども、ただ自由自在とのみ唱えて分限を知らざればわがまま放蕩に陥ること多し。すなわちその分限とは、天の道理に基づき人の情に従い、他人の妨げをなさずしてわが一身の自由を達することなり。自由とわがままとの界は、他人の妨げをなすとなさざるとの間にあり。譬えば自分の金銀を費やしてなすことなれば、たとい酒色に耽り放蕩を尽くすも自由自在なるべきに似たれども、けっして然らず、一人の放蕩は諸人の手本となり、ついに世間の風俗を乱りて人の教えに妨げをなすがゆえに、その費やすところの金銀はその人のものたりとも、その罪許すべからず。

また自由独立のことは人の一身にあるのみならず、一国の上にもあることなり。わが日本はアジヤ州の東に離れたる一個の島国にて、古来外国と交わりを結ばず、ひとり自国の産物のみを衣食して不足と思いしこともなかりしが、嘉永年中アメリカ人渡来せしより外国交易のこと始まり、今日の有様に及びしことにて、開港の後もいろいろと議論多く、鎖国攘夷などとやかましく言いし者もありしかども、その見るところはなはだ狭く、諺に言う「井の底の蛙」にて、その議論とるに足らず。日本とても西洋諸国とても同じ天地の間にありて、同じ日輪に照らされ、同じ月を眺め、海をともにし、空気をともにし、情合い相同じき人民なれば、ここに余るものは彼に渡し、彼に余るものは我に取り、互いに相教え互いに相学び、恥ずることもなく誇ることもなく、互いに便利を達し互いにその幸いを祈り、天理人道に従いて互いの交わりを結び、理のためにはアフリカの黒奴にも恐れ入り、道のためにはイギリス・アメリカの軍艦をも恐れず、国の恥辱とありては日本国中の人民一人も残らず命を棄てて国の威光を落とさざるこそ、一国の自由独立と申すべきなり。

しかるを支那人などのごとく、わが国よりほかに国なきごとく、外国の人を見ればひとくちに夷狄夷狄と唱え、四足にてあるく畜類のようにこれを賤しめこれを嫌い、自国の力をも計らずしてみだりに外国人を追い払わんとし、かえってその夷狄に窘しめらるるなどの始末は、実に国の分限を知らず、一人の身の上にて言えば天然の自由を達せずしてわがまま放蕩に陥る者と言うべし。王制一度新たなりしより以来、わが日本の政風大いに改まり、外は万国の公法をもって外国に交わり、内は人民に自由独立の趣旨を示し、すでに平民へ苗字・乗馬を許せしがごときは開闢以来の一美事、士農工商四民の位を一様にするの基ここに定まりたりと言うべきなり。

されば今より後は日本国中の人民に、生まれながらその身につきたる位などと申すはまずなき姿にて、ただその人の才徳とその居処とによりて位もあるものなり。たとえば政府の官吏を粗略にせざるは当然のことなれども、こはその人の身の貴きにあらず、その人の才徳をもってその役儀を勤め、国民のために貴き国法を取り扱うがゆえにこれを貴ぶのみ。人の貴きにあらず、国法の貴きなり。旧幕府の時代、東海道にお茶壺の通行せしは、みな人の知るところなり。そのほか御用の鷹は人よりも貴く、御用の馬には往来の旅人も路を避くる等、すべて御用の二字を付くれば、石にても瓦にても恐ろしく貴きもののように見え、世の中の人も数千百年の古よりこれを嫌いながらまた自然にその仕来りに慣れ、上下互いに見苦しき風俗を成せしことなれども、畢竟これらはみな法の貴きにもあらず、品物の貴きにもあらず、ただいたずらに政府の威光を張り人を畏して人の自由を妨げんとする卑怯なる仕方にて、実なき虚威というものなり。今日に至りてはもはや全日本国内にかかる浅ましき制度、風俗は絶えてなきはずなれば、人々安心いたし、かりそめにも政府に対して不平をいだくことあらば、これを包みかくして暗に上を怨むることなく、その路を求め、その筋により静かにこれを訴えて遠慮なく議論すべし。天理人情にさえ叶うことならば、一命をも抛ちて争うべきなり。これすなわち一国人民たる者の分限と申すものなり。

前条に言えるとおり、人の一身も一国も、天の道理に基づきて不覊自由なるものなれば、もしこの一国の自由を妨げんとする者あらば世界万国を敵とするも恐るるに足らず、この一身の自由を妨げんとする者あらば政府の官吏も憚るに足らず。ましてこのごろは四民同等の基本も立ちしことなれば、いずれも安心いたし、ただ天理に従いて存分に事をなすべしとは申しながら、およそ人たる者はそれぞれの身分あれば、またその身分に従い相応の才徳なかるべからず。身に才徳を備えんとするには物事の理を知らざるべからず。物事の理を知らんとするには字を学ばざるべからず。これすなわち学問の急務なるわけなり。

昨今の有様を見るに、農工商の三民はその身分以前に百倍し、やがて士族と肩を並ぶるの勢いに至り、今日にても三民のうちに人物あれば政府の上に採用せらるべき道すでに開けたることなれば、よくその身分を顧み、わが身分を重きものと思い、卑劣の所行あるべからず。およそ世の中に無知文盲の民ほど憐れむべくまた悪むべきものはあらず。智恵なきの極みは恥を知らざるに至り、己が無智をもって貧窮に陥り飢寒に迫るときは、己が身を罪せずしてみだりに傍の富める人を怨み、はなはだしきは徒党を結び強訴・一揆などとて乱暴に及ぶことあり。恥を知らざるとや言わん、法を恐れずとや言わん。天下の法度を頼みてその身の安全を保ち、その家の渡世をいたしながら、その頼むところのみを頼みて、己が私欲のためにはまたこれを破る、前後不都合の次第ならずや。あるいはたまたま身本慥かにして相応の身代ある者も、金銭を貯うることを知りて子孫を教うることを知らず。教えざる子孫なればその愚なるもまた怪しむに足らず。ついには遊惰放蕩に流れ、先祖の家督をも一朝の煙となす者少なからず。

かかる愚民を支配するにはとても道理をもって諭すべき方便なければ、ただ威をもって畏すのみ。西洋の諺に「愚民の上に苛き政府あり」とはこのことなり。こは政府の苛きにあらず、愚民のみずから招く災なり。愚民の上に苛き政府あれば、良民の上には良き政府あるの理なり。ゆえに今わが日本国においてもこの人民ありてこの政治あるなり。仮りに人民の徳義今日よりも衰えてなお無学文盲に沈むことあらば、政府の法も今一段厳重になるべく、もしまた、人民みな学問に志して、物事の理を知り、文明の風に赴くことあらば、政府の法もなおまた寛仁大度の場合に及ぶべし。法の苛きと寛やかなるとは、ただ人民の徳不徳によりておのずから加減あるのみ。人誰か苛政を好みて良政を悪む者あらん、誰か本国の富強を祈らざる者あらん、誰か外国の侮りを甘んずる者あらん、これすなわち人たる者の常の情なり。今の世に生まれ報国の心あらん者は、必ずしも身を苦しめ思いを焦がすほどの心配あるにあらず。ただその大切なる目当ては、この人情に基づきてまず一身の行ないを正し、厚く学に志し、博く事を知り、銘々の身分に相応すべきほどの智徳を備えて、政府はその政を施すに易く、諸民はその支配を受けて苦しみなきよう、互いにその所を得てともに全国の太平を護らんとするの一事のみ。今余輩の勧むる学問ももっぱらこの一事をもって趣旨とせり。

\subsection{端書}
このたび余輩の故郷中津に学校を開くにつき、学問の趣意を記して旧く交わりたる同郷の友人へ示さんがため一冊を綴りしかば、或る人これを見ていわく、「この冊子をひとり中津の人へのみ示さんより、広く世間に布告せばその益もまた広かるべし」との勧めにより、すなわち慶応義塾の活字版をもってこれを摺り、同志の一覧に供うるなり。

明治四年未十二月

\begin{flushright}
福沢諭吉  

記

小幡篤次郎 
\end{flushright}

\section{二編}
\subsection{端書}
学問とは広き言葉にて、無形の学問もあり、有形の学問もあり。心学、神学、理学等は形なき学問なり。天文、地理、窮理、化学等は形ある学問なり。いずれにてもみな知識見聞の領分を広くして、物事の道理をわきまえ、人たる者の職分を知ることなり。知識見聞を開くためには、あるいは人の言を聞き、あるいはみずから工夫を運らし、あるいは書物をも読まざるべからず。ゆえに学問には文字を知ること必要なれども、古来世の人の思うごとく、ただ文字を読むのみをもって学問とするは大なる心得違いなり。

文字は学問をするための道具にて、譬えば家を建つるに槌・鋸の入用なるがごとし。槌・鋸は普請に欠くべからざる道具なれども、その道具の名を知るのみにて家を建つることを知らざる者はこれを大工と言うべからず。まさしくこのわけにて、文字を読むことのみを知りて物事の道理をわきまえざる者はこれを学者と言うべからず。いわゆる「論語よみの論語しらず」とはすなわちこれなり。わが国の『古事記』は暗誦すれども今日の米の相場を知らざる者は、これを世帯の学問に暗き男と言うべし。経書・史類の奥義には達したれども商売の法を心得て正しく取引きをなすこと能わざる者は、これを帳合いの学問に拙き人と言うべし。数年の辛苦を嘗め、数百の執行金を費やして洋学は成業したれども、なおも一個私立の活計をなし得ざる者は、時勢の学問に疎き人なり。これらの人物はただこれを文字の問屋と言うべきのみ。その功能は飯を食う字引に異ならず。国のためには無用の長物、経済を妨ぐる食客と言うて可なり。ゆえに世帯も学問なり、帳合いも学問なり、時勢を察するもまた学問なり。なんぞ必ずしも和漢洋の書を読むのみをもって学問と言うの理あらんや。

この書の表題は『学問のすすめ』と名づけたれども、けっして字を読むことのみを勧むるにあらず。書中に記すところは、西洋の諸書よりあるいはその文を直ちに訳し、あるいはその意を訳し、形あることにても形なきことにても、一般に人の心得となるべき事柄を挙げて学問の大趣意を示したるものなり。先に著わしたる一冊を初編となし、なおその意を拡めてこのたびの二編を綴り、次いで三、四編にも及ぶべし。

\subsection{人は同等なること}
初編の首に、人は万人みな同じ位にて生まれながら上下の別なく自由自在云々とあり。今この義を拡めて言わん。人の生まるるは天の然らしむるところにて人力にあらず。この人々互いに相敬愛しておのおのその職分を尽くし互いに相妨ぐることなき所以は、もと同類の人間にしてともに一天を与にし、ともに与に天地の間の造物なればなり。譬えば一家の内にて兄弟相互に睦しくするは、もと同一家の兄弟にしてともに一父一母を与にするの大倫あればなり。

ゆえに今、人と人との釣合いを問えばこれを同等と言わざるを得ず。ただしその同等とは有様の等しきを言うにあらず、権理道義の等しきを言うなり。その有様を論ずるときは、貧富、強弱、智愚の差あることはなはだしく、あるいは大名華族とて御殿に住居し美服、美食する者もあり、あるいは人足とて裏店に借屋して今日の衣食に差しつかえる者もあり、あるいは才智逞しゅうして役人となり商人となりて天下を動かす者もあり、あるいは智恵分別なくして生涯、飴やおこしを売る者もあり、あるいは強き相撲取りあり、あるいは弱きお姫様あり、いわゆる雲と泥との相違なれども、また一方より見てその人々持ち前の権理通義をもって論ずるときは、いかにも同等にして一厘一毛の軽重あることなし。すなわちその権理通義とは、人々その命を重んじ、その身代所持のものを守り、その面目名誉を大切にするの大義なり。天の人を生ずるや、これに体と心との働きを与えて、人々をしてこの通義を遂げしむるの仕掛けを設けたるものなれば、なんらのことあるも人力をもってこれを害すべからず。

大名の命も人足の命も、命の重きは同様なり。豪商百万両の金も、飴やおこし四文の銭も、己がものとしてこれを守るの心は同様なり。世の悪しき諺に、「泣く子と地頭には叶わず」と。またいわく、「親と主人は無理を言うもの」などとて、あるいは人の権理通義をも枉ぐべきもののよう唱うる者あれども、こは有様と通義とを取り違えたる論なり。地頭と百姓とは、有様を異にすれどもその権理を異にするにあらず。百姓の身に痛きことは地頭の身にも痛きはずなり、地頭の口に甘きものは百姓の口にも甘からん。痛きものを遠ざけ甘きものを取るは人の情欲なり他の妨げをなさずして達すべきの情を達するはすなわち人の権理なり。この権理に至りては地頭も百姓も厘毛の軽重あることなし。ただ地頭は富みて強く、百姓は貧にして弱きのみ。貧富、強弱は人の有様にてもとより同じかるべからず。

しかるに今、富強の勢いをもって貧弱なる者へ無理を加えんとするは、有様の不同なるがゆえにとて他の権理を害するにあらずや。これを譬えば力士がわれに腕の力ありとて、その力の勢いをもって隣の人の腕を捻り折るがごとし。隣の人の力はもとより力士よりも弱かるべけれども、弱ければ弱きままにてその腕を用い自分の便利を達して差しつかえなきはずなるに、いわれなく力士のために腕を折らるるは迷惑至極と言うべし。

また右の議論を世の中のことに当てはめて言わん。旧幕府の時代には士民の区別はなはだしく、士族はみだりに権威を振るい、百姓・町人を取り扱うこと目の下の罪人のごとくし、あるいは切捨て御免などの法あり。この法によれば、平民の生命はわが生命にあらずして借り物に異ならず。百姓・町人は由縁もなき士族へ平身低頭し、外にありては路を避け、内にありて席を譲り、はなはだしきは自分の家に飼いたる馬にも乗られぬほどの不便利を受けたるはけしからぬことならずや。

右は士族と平民と一人ずつ相対したる不公平なれども、政府と人民との間柄にいたってはなおこれよりも見苦しきことあり。幕府はもちろん、三百諸侯の領分にもおのおの小政府を立てて、百姓・町人を勝手次第に取り扱い、あるいは慈悲に似たることあるもその実は人に持ち前の権理通義を許すことなくして、実に見るに忍びざること多し。そもそも政府と人民との間柄は、前にも言えるごとく、ただ強弱の有様を異にするのみにて権理の異同あるの理なし。百姓は米を作りて人を養い、町人は物を売買して世の便利を達す。これすなわち百姓・町人の商売なり。政府は法令を設けて悪人を制し、善人を保護す。これすなわち政府の商売なり。この商売をなすには莫大の費えなれども、政府には米もなく金もなきゆえ、百姓・町人より年貢・運上を出だして政府の勝手方を賄わんと、双方一致のうえ相談を取り極めたり。これすなわち政府と人民との約束なり。ゆえに百姓・町人は年貢・運上を出だして固く国法を守れば、その職分を尽くしたりと言うべし。政府は年貢・運上を取りて正しくその使い払いを立て人民を保護すれば、その職分を尽くしたりと言うべし。双方すでにその職分を尽くして約束を違うることなきうえは、さらになんらの申し分もあるべからず、おのおのその権理通義を逞しゅうして少しも妨げをなすの理なし。

しかるに幕府のとき政府のことをお上様と唱え、お上の御用とあればばかに威光を振るうのみならず、道中の旅籠までもただ食い倒し、川場に銭を払わず、人足に賃銭を与えず、はなはだしきは旦那が人足をゆすりて酒代を取るに至れり。沙汰の限りと言うべし。あるいは殿様のものずきにて普請をするか、または役人の取り計らいにていらざる事を起こし、無益に金を費やして入用不足すれば、いろいろ言葉を飾りて年貢を増し御用金を言いつけ、これを御国恩に報ゆると言う。そもそも御国恩とは何事をさすや。百姓・町人らが安穏に家業を営み、盗賊・人殺しの心配もなくして渡世するを、政府の御恩と言うことなるべし。もとよりかく安穏に渡世するは政府の法あるがためなれども、法を設けて人民を保護するはもと政府の商売柄にて当然の職分なり。これを御恩と言うべからず。政府もし人民に対しその保護をもって御恩とせば、百姓・町人は政府に対しその年貢・運上をもって御恩と言わん。政府もし人民の公事訴訟をもってお上の御厄介と言わば、人民もまた言うべし、「十俵作り出だしたる米のうちより五俵の年貢を取らるるは百姓のために大なる御厄介なり」と。いわゆる売り言葉に買い言葉にて、はてしもあらず。とにかくに等しく恩のあるものならば、一方より礼を言いて一方より礼を言わざるの理はなかるべし。

かかる悪風俗の起こりし由縁を尋ぬるに、その本は人間同等の大趣意を誤りて、貧富強弱の有様を悪しき道具に用い、政府富強の勢いをもって貧弱なる人民の権理通義を妨ぐるの場合に至りたるなり。ゆえに人たる者は常に同位同等の趣意を忘るべからず。人間世界にもっとも大切なることなり。西洋の言葉にてこれをレシプロシチまたはエクウオリチと言う。すなわち初編の首に言える万人同じ位とはこのことなり。

右は百姓・町人に左袒して思うさまに勢いを張れという議論なれども、また一方より言えば別に論ずることあり。およそ人を取り扱うには、その相手の人物次第にておのずからその法の加減もなかるべからず。元来、人民と政府との間柄はもと同一体にてその職分を区別し、政府は人民の名代となりて法を施し、人民は必ずこの法を守るべしと、固く約束したるものなり。譬えば今、日本国中にて明治の年号を奉ずる者は、今の政府の法に従うべしと条約を結びたる人民なり。ゆえにひとたび国法と定まりたることは、たといあるいは人民一個のために不便利あるも、その改革まではこれを動かすを得ず。小心翼々謹みて守らざるべからず。これすなわち人民の職分なり。しかるに無学文盲、理非の理の字も知らず、身に覚えたる芸は飲食と寝ると起きるとのみ、その無学のくせに欲は深く、目の前に人を欺きて巧みに政府の法を遁れ、国法の何ものたるを知らず、己が職分の何ものたるを知らず、子をばよく生めどもその子を教うるの道を知らず、いわゆる恥も法も知らざる馬鹿者にて、その子孫繁盛すれば一国の益はなさずして、かえって害をなす者なきにあらず。かかる馬鹿者を取り扱うにはとても道理をもってすべからず、不本意ながら力をもって威し、一時の大害を鎮むるよりほかに方便あることなし。

これすなわち世に暴政府のある所以なり。ひとりわが旧幕府のみならず、アジヤ諸国古来みな然り。されば一国の暴政は必ずしも暴君暴吏の所為のみにあらず、その実は人民の無智をもってみずから招く禍なり。他人にけしかけられて暗殺を企つる者もあり、新法を誤解して一揆を起こす者あり、強訴を名として金持の家を毀ち、酒を飲み銭を盗む者あり。その挙動はほとんど人間の所業と思われず。かかる賊民を取り扱うには、釈迦も孔子も名案なきは必定、ぜひとも苛刻の政を行なうことなるべし。ゆえにいわく、人民もし暴政を避けんと欲せば、すみやかに学問に志しみずから才徳を高くして、政府と相対し同位同等の地位に登らざるべからず。これすなわち余輩の勧むる学問の趣意なり。

\section{三編}
\subsection{国は同等なること}
およそ人とさえ名あれば、富めるも貧しきも、強きも弱きも、人民も政府も、その権義において異なるなしとのことは、第二編に記せり〔二編にある権理通義の四字を略して、ここにはただ権義と記したり。いずれも英語のライト、right という字に当たる〕。今この義を拡めて国と国との間柄を論ぜん。国とは人の集まりたるものにて、日本国は日本人の集まりたるものなり、英国は英国人の集まりたるものなり。日本人も英国人も等しく天地の間の人なれば、互いにその権義を妨ぐるの理なし。一人が一人に向かいて害を加うるの理なくば、二人が二人に向かいて害を加うるの理もなかるべし。百万人も千万人も同様のわけにて、物事の道理は人数の多少によりて変ずべからず。今、世界中を見渡すに、文明開化とて文学も武備も盛んにして富強なる国あり、あるいは蛮野未開とて文武ともに不行届きにして貧弱なる国あり。一般にヨーロッパ・アメリカの諸国は富んで強く、アジヤ・アフリカの諸国は貧にして弱し。されどもこの貧富・強弱は国の有様なれば、もとより同じかるべからず。しかるにいま、自国の富強なる勢いをもって貧弱なる国へ無理を加えんとするは、いわゆる力士が腕の力をもって病人の腕を握り折るに異ならず、国の権義において許すべからざることなり。

近くはわが日本国にても、今日の有様にては西洋諸国の富強に及ばざるところあれども、一国の権義においては厘毛の軽重あることなし。道理に戻りて曲を蒙るの日に至りては、世界中を敵にするも恐るるに足らず。初編第六葉にも言えるごとく、「日本国中の人民一人も残らず命を棄てて国の威光を落とさず」とはこの場合なり。しかのみならず、貧富・強弱の有様は天然の約束にあらず、人の勉と不勉とによりて移り変わるべきものにて、今日の愚人も明日は智者となるべく、昔年の富強も今世の貧弱となるべし。古今その例少なからず。わが日本国人も今より学問に志し気力を慥かにして、まず一身の独立を謀り、したがって一国の富強を致すことあらば、なんぞ西洋人の力を恐るるに足らん。道理あるものはこれに交わり、道理なきものはこれを打ち払わんのみ。一身独立して一国独立するとはこのことなり。

\subsection{一身独立して一国独立すること}
前条に言えるごとく、国と国とは同等なれども、国中の人民に独立の気力なきときは一国独立の権義を伸ぶること能わず。その次第三ヵ条あり。

第一条 独立の気力なき者は国を思うこと深切ならず。

独立とは自分にて自分の身を支配し他によりすがる心なきを言う。みずから物事の理非を弁別して処置を誤ることなき者は、他人の智恵によらざる独立なり。みずから心身を労して私立の活計をなす者は、他人の財によらざる独立なり。人々この独立の心なくしてただ他人の力によりすがらんとのみせば、全国の人はみな、よりすがる人のみにてこれを引き受くる者はなかるべし。これを譬えば盲人の行列に手引きなきがごとし、はなはだ不都合ならずや。或る人いわく、「民はこれによらしむべしこれを知らしむべからず、世の中は目くら千人目あき千人なれば、智者上にありて諸民を支配し上の意に従わしめて可なり」と。この議論は孔子様の流儀なれども、その実は大いに非なり。一国中に人を支配するほどの才徳を備うる者は千人のうち一人に過ぎず。

仮りにここに人口百万人の国あらん。このうち千人は智者にして九十九万余の者は無智の小民ならん。智者の才徳をもってこの小民を支配し、あるいは子のごとくして愛し、あるいは羊のごとくして養い、あるいは威しあるいは撫し、恩威ともに行なわれてその向かうところを示すことあらば、小民も識らず知らずして上の命に従い、盗賊、人殺しの沙汰もなく、国内安穏に治まることあるべけれども、もとこの国の人民、主客の二様に分かれ、主人たる者は千人の智者にて、よきように国を支配し、その余の者は悉皆何も知らざる客分なり。すでに客分とあればもとより心配も少なく、ただ主人にのみよりすがりて身に引き受くることなきゆえ、国を患うることも主人のごとくならざるは必然、実に水くさき有様なり。国内のことなればともかくもなれども、いったん外国と戦争などのことあらばその不都合なること思い見るべし。無智無力の小民ら、戈を倒にすることもなかるべけれども、われわれは客分のことなるゆえ一命を棄つるは過分なりとて逃げ走る者多かるべし。さすればこの国の人口、名は百万人なれども、国を守るの一段に至りてはその人数はなはだ少なく、とても一国の独立は叶い難きなり。

右の次第につき、外国に対してわが国を守らんには自由独立の気風を全国に充満せしめ、国中の人々、貴賤上下の別なく、その国を自分の身の上に引き受け、智者も愚者も目くらも目あきも、おのおのその国人たるの分を尽くさざるべからず。英人は英国をもってわが本国と思い、日本人は日本国をもってわが本国と思い、その本国の土地は他人の土地にあらず、わが国人の土地なれば、本国のためを思うことわが家を思うがごとし。国のためには財を失うのみならず、一命をも抛ちて惜しむに足らず。これすなわち報国の大義なり。

もとより国の政をなす者は政府にて、その支配を受くる者は人民なれども、こはただ便利のために双方の持ち場を分かちたるのみ。一国全体の面目にかかわることに至りては、人民の職分として政府のみに国を預け置き、傍よりこれを見物するの理あらんや。すでに日本国の誰、英国の誰と、その姓名の肩書に国の名あればその国に住居し、起居眠食、自由自在なるの権義あり。すでにその権義あればまたしたがってその職分なかるべからず。

昔戦国の時、駿河の今川義元、数万の兵を率いて織田信長を攻めんとせしとき、信長の策にて桶狭間に伏勢を設け、今川の本陣に迫りて義元の首を取りしかば、駿河の軍勢は蜘蛛の子を散らすがごとく、戦いもせずして逃げ走り、当時名高き駿河の今川政府も一朝に亡びてその痕なし。近く両三年以前、フランスとプロイセンとの戦いに、両国接戦のはじめ、フランス帝ナポレオンはプロイセンに生け捕られたれども、仏人はこれによりて望みを失わざるのみならず、ますます憤発して防ぎ戦い、骨をさらし血を流し、数月籠城ののち和睦に及びたれども、フランスは依然として旧のフランスに異ならず。かの今川の始末に比ぶれば日を同じゅうして語るべからず。そのゆえはなんぞや。駿河の人民はただ義元一人によりすがり、その身は客分のつもりにて、駿河の国をわが本国と思う者なく、フランスには報国の士民多くして国の難を銘々の身に引き受け、人の勧めを待たずしてみずから本国のために戦う者あるゆえ、かかる相違もできしことなり。これによりて考うれば、外国へ対して自国を守るに当たり、その国人に独立の気力ある者は国を思うこと深切にして、独立の気力なき者は不深切なること推して知るべきなり。

第二条 内に居て独立の地位を得ざる者は、外にありて外国人に接するときもまた独立の権義を伸ぶること能わず。

独立の気力なき者は必ず人に依頼す、人に依頼する者は必ず人を恐る、人を恐るる者は必ず人に諛うものなり。常に人を恐れ人に諛う者はしだいにこれに慣れ、その面の皮、鉄のごとくなりて、恥ずべきを恥じず、論ずべきを論ぜず、人をさえ見ればただ腰を屈するのみ。いわゆる「習い、性となる」とはこのことにて、慣れたることは容易に改め難きものなり。譬えば今、日本にて平民に苗字・乗馬を許し、裁判所の風も改まりて、表向きはまず士族と同等のようなれども、その習慣にわかに変ぜず、平民の根性は依然として旧の平民に異ならず、言語も賤しく応接も賤しく、目上の人に逢えば一言半句の理屈を述ぶること能わず、立てと言えば立ち、舞えと言えば舞い、その柔順なること家に飼いたる痩せ犬のごとし。実に無気無力の鉄面皮と言うべし。

昔鎖国の世に旧幕府のごとき窮屈なる政を行なう時代なれば、人民に気力なきもその政事に差しつかえざるのみならずかえって便利なるゆえ、ことさらにこれを無智に陥れ、無理に柔順ならしむるをもって役人の得意となせしことなれども、今、外国と交わるの日に至りてはこれがため大なる弊害あり。譬えば田舎の商人ら、恐れながら外国の交易に志して横浜などへ来る者あれば、まず外国人の骨格たくましきを見てこれに驚き、金の多きを見てこれに驚き、商館の洪大なるに驚き、蒸気船の速きに驚き、すでにすでに胆を落として、追い追いこの外国人に近づき取引きするに及んでは、その駆引きのするどきに驚き、あるいは無理なる理屈を言いかけらるることあればただに驚くのみならず、その威力に震い懼れて、無理と知りながら大なる損亡を受け大なる恥辱を蒙ることあり。こは一人の損亡にあらず、一国の損亡なり。一人の恥辱にあらず、一国の恥辱なり。実に馬鹿らしきようなれども、先祖代々独立の気を吸わざる町人根性、武士には窘しめられ、裁判所には叱られ、一人扶持取る足軽に逢いてもお旦那さまと崇めし魂は腹の底まで腐れつき、一朝一夕に洗うべからず、かかる臆病神の手下どもが、かの大胆不敵なる外国人に逢いて、胆をぬかるるは無理ならぬことなり。これすなわち内に居て独立を得ざる者は外にありても独立すること能わざるの証拠なり。

第三条 独立の気力なき者は人に依頼して悪事をなすことあり。

旧幕府の時代に名目金とて、御三家などと唱うる権威強き大名の名目を借りて金を貸し、ずいぶん無理なる取引きをなせしことあり。その所業はなはだ悪むべし。自分の金を貸して返さざる者あらば、再三再四力を尽くして政府に訴うべきなり。しかるにこの政府を恐れて訴うることを知らず、きたなくも他人の名目を借り他人の暴威によりて返金を促すとは卑怯なる挙動ならずや。今日に至りては名目金の沙汰は聞かざれども、あるいは世間に外国人の名目を借る者はあらずや。余輩いまだその確証を得ざるゆえ明らかにここに論ずること能わざれども、昔日のことを思えば今の世の中にも疑念なきを得ず。こののち万々一も外国人雑居などの場合に及び、その名目を借りて奸を働く者あらば、国の禍、実に言うべからざるべし。ゆえに人民に独立の気力なきはその取扱いに便利などとて油断すべからず。禍は思わぬところに起こるものなり。国民に独立の気力いよいよ少なければ、国を売るの禍もまたしたがってますます大なるべし。すなわちこの条のはじめに言える、人に依頼して悪事をなすとはこのことなり。

右三ヵ条に言うところはみな、人民に独立の心なきより生ずる災害なり。今の世に生まれいやしくも愛国の意あらん者は、官私を問わずまず自己の独立を謀り、余力あらば他人の独立を助け成すべし。父兄は子弟に独立を教え、教師は生徒に独立を勧め、士農工商ともに独立して国を守らざるべからず。概してこれを言えば、人を束縛してひとり心配を求むるより、人を放ちてともに苦楽を与にするに若かざるなり。

\section{四編}
\subsection{学者の職分を論ず}
近来ひそかに識者の言を聞くに、「今後日本の盛衰は人智をもって明らかに計り難しといえども、つまり、その独立を失うの患いはなかるべしや、方今目撃するところの勢いによりてしだいに進歩せば、必ず文明盛大の域に至るべしや」と言いて、これを問う者あり。あるいは「その独立の保つべきと否とは、今より二、三十年を過ぎざれば明らかにこれを期すること難かるべし」と言いて、これを疑う者あり。あるいははなはだしくこの国を蔑視したる外国人の説に従えば、「とても日本の独立は危し」と言いて、これを難ずる者あり。もとより人の説を聞いてにわかにこれを信じわが望みを失するにはあらざれども、畢竟この諸説はわが独立の保つべきと否とについての疑問なり。事に疑いあらざれば問いのよって起こるべき理なし。今試みに英国に行き、「ブリテンの独立保つべきや否や」と言いてこれを問わば、人みな笑いて答うる者なかるべし。その答うる者なきはなんぞや、これを疑わざればなり。しからばすなわちわが国文明の有様、今日をもって昨日に比すればあるいは進歩せしに似たることあるも、その結局に至りてはいまだ一点の疑いあるを免れず。いやしくもこの国に生まれて日本人の名ある者は、これに寒心せざるを得んや。今わが輩もこの国に生まれて日本人の名あり、すでにその名あればまたおのおのその分を明らかにして尽くすところなかるべからず。もとより政の字の義に限りたることをなすは政府の任なれども、人間の事務には政府の関わるべからざるものもまた多し。ゆえに一国の全体を整理するには、人民と政府と両立してはじめてその成功を得べきものなれば、わが輩は国民たるの分限を尽くし、政府は政府たるの分限を尽くし、互いに相助けもって全国の独立を維持せざるべからず。

すべて物を維持するには力の平均なかるべからず。譬えば人身のごとし。これを健康に保たんとするには、飲食なかるべからず、大気、光線なかるべからず、寒熱、痛痒、外より刺衝して内よりこれに応じ、もって一身の働きを調和するなり。今にわかにこの外物の刺衝を去り、ただ生力の働くところにまかしてこれを放頓することあらば、人身の健康は一日も保つべからず。国もまた然り。政は一国の働きなり。この働きを調和して国の独立を保たんとするには、内に政府の力あり、外に人民の力あり、内外相応じてその力を平均せざるべからず。ゆえに政府はなお生力のごとく、人民はなお外物の刺衝のごとし。今にわかにこの刺衝を去り、ただ政府の働くところにまかしてこれを放頓することあらば、国の独立は一日も保つべからず。いやしくも人身窮理の義を明らかにし、その定則をもって一国経済の議論に施すことを知る者は、この理を疑うことなかるべし。

方今わが国の形勢を察し、その外国に及ばざるものを挙ぐれば、いわく学術、いわく商売、いわく法律、これなり。世の文明はもっぱらこの三者に関し、三者挙がらざれば国の独立を得ざること識者を俟たずして明らかなり。しかるにいまわが国において一もその体をなしたるものなし。

政府一新の時より在官の人物、力を尽くさざるにあらず、その才力また拙劣なるにあらずといえども、事を行なうに当たり如何ともすべからざるの原因ありて、意のごとくならざるもの多し。その原因とは人民の無知文盲すなわちこれなり。政府すでにその原因のあるところを知り、しきりに学術を勧め、法律を議し、商法を立つるの道を示す等、あるいは人民に説諭し、あるいはみずから先例を示し、百方その術を尽くすといえども、今日に至るまでいまだ実効の挙がるを見ず、政府は依然たる専制の政府、人民は依然たる無気無力の愚民のみ。あるいはわずかに進歩せしことあるも、これがため労するところの力と費やすところの金とに比すれば、その奏功見るに足るもの少なきはなんぞや。けだし一国の文明はひとり政府の力をもって進むべきものにあらざるなり。

人あるいはいわく、「政府はしばらくこの愚民を御するに一時の術策を用い、その智徳の進むを待ちて後にみずから文明の域に入らしむるなり」と。この説は言うべくして行なうべからず。わが全国の人民数千百年専制の政治に窘しめられ、人々その心に思うところを発露すること能わず、欺きて安全を偸み、詐りて罪を遁れ、欺詐術策は人生必需の具となり、不誠不実は日常の習慣となり、恥ずる者もなく怪しむ者もなく、一身の廉恥すでに地を払いて尽きたり、豈国を思うに遑あらんや。政府はこの悪弊を矯めんとしてますます虚威を張り、これを嚇しこれを叱し、強いて誠実に移らしめんとしてかえってますます不信に導き、その事情あたかも火をもって火を救うがごとし。ついに上下の間隔絶しておのおの一種無形の気風をなせり。その気風とはいわゆるスピリットなるものにて、にわかにこれを動かすべからず。近日に至り政府の外形は大いに改まりたれども、その専制抑圧の気風は今なお存せり。人民もやや権利を得るに似たれども、その卑屈不信の気風は依然として旧に異ならず。この気風は無形無体にして、にわかに一個の人につき一場の事を見て名状すべきものにあらざれども、その実の力ははなはだ強くして、世間全体の事跡に顕わるるを見れば、明らかにその虚にあらざるを知るべし。

試みにその一を挙げて言わん。今、在官の人物少なしとせず、私にその言を聞きその行を見ればおおむねみな闊達大度の士君子にて、わが輩これを間然する能わざるのみならず、その言行あるいは慕うべきものあり。また一方より言えば平民といえども悉皆無気無力の愚民のみにあらず、万に一人は公明誠実の良民もあるべし。しかるに今この士君子、政府に会して政をなすに当たり、その為政の事跡を見ればわが輩の悦ばざるものはなはだ多く、またかの誠実なる良民も、政府に接すればたちまちその節を屈し、偽詐術策、もって官を欺き、かつて恥ずるものなし。この士君子にしてこの政を施し、この民にしてこの賤劣に陥るはなんぞや。あたかも一身両頭あるがごとし。私にありては智なり、官にありては愚なり。これを散ずれば明なり、これを集むれば暗なり。政府は衆智者の集まるところにして一愚人の事を行なうものと言うべし。豈怪しまざるを得んや。畢竟その然る所以はかの気風なるものに制せられて、人々みずから一個の働きを逞しゅうすること能わざるによりて致すところならんか。維新以来、政府にて学術、法律、商売等の道を興さんとして効験なきも、その病の原因はけだしここにあるなり。しかるにいま一時の術を用いて下民を御しその知徳の進むを待つとは、威をもって人を文明に強ゆるものか、しからざれば欺きて善に帰せしむるの策なるべし。政府威を用うれば人民は偽をもってこれに応ぜん、政府欺を用うれば人民は容を作りてこれに従わんのみ。これを上策と言うべからず。たといその策は巧みなるも、文明の事実に施して益なかるべし。ゆえにいわく、世の文明を進むるにはただ政府の力のみに依頼すべからざるなり。

右所論をもって考うれば、方今わが国の文明を進むるには、まずかの人心に浸潤したる気風を一掃せざるべからず。これを一掃するの法、政府の命をもってし難し、私の説諭をもってし難し、必ずしも人に先だって私に事をなし、もって人民のよるべき標的を示す者なかるべからず。今この標的となるべき人物を求むるに、農の中にあらず、商の中にあらず、また和漢の学者中にもあらず、その任に当たる者はただ一種の洋学者流あるのみ。

しかるにまたこれに依頼すべからざるの事情あり。近来この流の人ようやく世間に増加し、あるいは横文を講じあるいは訳書を読み、もっぱら力を尽くすに似たりといえども、学者あるいは字を読みて義を解さざるか、あるいは義を解してこれを事実に施すの誠意なきか、その所業につきわが輩の疑いを存するもの少なからず。その疑いを存するとは、この学者士君子、みな官あるを知りて私あるを知らず、政府の上に立つの術を知りて、政府の下に居るの道を知らざるの一事なり。畢竟、漢学者流の悪習を免れざるものにて、あたかも漢を体にして洋を衣にするがごとし。

試みにその実証を挙げて言わん。方今世の洋学者流はおおむねみな官途につき、私に事をなす者はわずかに指を屈するに足らず。けだしその官にあるはただ利これ貪るのためのみにあらず、生来の教育に先入してひたすら政府に眼を着し、政府にあらざればけっして事をなすべからざるものと思い、これに依頼して宿昔青雲の志を遂げんと欲するのみ。あるいは世に名望ある大家先生といえどもこの範囲を脱するを得ず。その所業あるいは賤しむべきに似たるも、その意は深く咎むるに足らず、けだし意の悪しきにあらず、ただ世間の気風に酔いてみずから知らざるなり。名望を得たる士君子にしてかくのごとし。天下の人豈その風に倣わざるを得んや。

青年の書生わずかに数巻の書を読めばすなわち官途に志し、有志の町人わずかに数百の元金あればすなわち官の名を仮りて商売を行なわんとし、学校も官許なり、説教も官許なり、牧牛も官許、養蚕も官許、およそ民間の事業、十に七、八は官の関せざるものなし。これをもって世の人心ますますその風に靡き、官を慕い官を頼み、官を恐れ官に諂い、毫も独立の丹心を発露する者なくして、その醜体見るに忍びざることなり。譬えば方今出版の新聞紙および諸方の上書建白の類もその一例なり。出版の条令はなはだしく厳なるにあらざれども、新聞紙の面を見れば政府の忌諱に触るることは絶えて載せざるのみならず、官に一毫の美事あればみだりにこれを称誉してその実に過ぎ、あたかも娼妓の客に媚びるがごとし。またかの上書建白を見ればその文つねに卑劣を極め、みだりに政府を尊崇すること鬼神のごとく、みずから賤しんずること罪人のごとくし、同等の人間世界にあるべからざる虚文を用い、恬として恥ずる者なし。この文を読みてその人を想えばただ狂人をもって評すべきのみ。しかるに今この新聞紙を出版し、あるいは政府に建白する者は、おおむねみな世の洋学者流にて、その私について見れば必ずしも娼妓にあらず、また狂人にもあらず。

しかるにその不誠不実、かくのごときのはなはだしきに至る所以は、いまだ世間に民権を首唱する実例なきをもって、ただかの卑屈の気風に制せられその気風に雷同して、国民の本色を見わし得ざるなり。これを概すれば、日本にはただ政府ありていまだ国民あらずと言うも可なり。ゆえにいわく、人民の気風を一洗して世の文明を進むるには、今の洋学者流にもまた依頼すべからざるなり。

前条所記の論説はたして是ならば、わが国の文明を進めてその独立を維持するは、ひとり政府の能くするところにあらず。また今の洋学者流も依頼するに足らず、必ずわが輩の任ずるところにして、まずわれより事の端を開き、愚民の先をなすのみならず、またかの洋学者流のために先駆してその向かうところを示さざるべからず。今わが輩の身分を考うるに、その学識もとより浅劣なりといえども、洋学に志すこと日すでに久しく、この国にありては中人以上の地位にある者なり。輓近世の改革も、もしわが輩の主として始めしことにあらざれば暗にこれを助けなしたるものなり。あるいは助成の力なきもその改革はわが輩の悦ぶところなれば、世の人もまたわが輩を目するに改革家流の名をもってすること必せり。すでに改革家の名ありて、またその身は中人以上の地位にあり、世人あるいはわが輩の所業をもって標的となす者あるべし。しからばすなわち今、人に先だって事をなすはまさにこれをわが輩の任と言うべきなり。

そもそも事をなすに、これを命ずるはこれを諭すに若かず、これを諭すはわれよりその実の例を示すに若かず。然りしこうして政府はただ命ずるの権あるのみ、これを諭して実の例を示すは私の事なれば、わが輩まず私立の地位を占め、あるいは学術を講じ、あるいは商売に従事し、あるいは法律を議し、あるいは書を著わし、あるいは新聞紙を出版するなど、およそ国民たるの分限に越えざることは忌諱を憚らずしてこれを行ない、固く法を守りて正しく事を処し、あるいは政令信ならずして曲を被ることあらば、わが地位を屈せずしてこれを論じ、あたかも政府の頂門に一針を加え、旧弊を除きて民権を恢復せんこと方今至急の要務なるべし。

もとより私立の事業は多端、かつこれを行なう人にもおのおの所長あるものなれば、わずかに数輩の学者にて悉皆その事をなすべきにあらざれども、わが目的とするところは事を行なうの巧みなるを示すにあらず、ただ天下の人に私立の方向を知らしめんとするのみ。百回の説諭を費やすは一回の実例を示すに若かず。今われより私立の実例を示し、「人間の事業はひとり政府の任にあらず。学者は学者にて私に事を行なうべし、町人は町人にて私に事をなすべし、政府も日本の政府なり、人民も日本の人民なり、政府は恐るべからず近づくべし、疑うべからず親しむべし」との趣を知らしめなば、人民ようやく向かうところを明らかにし、上下固有の気風もしだいに消滅して、はじめて真の日本国民を生じ、政府の玩具たらずして政府の刺衝となり、学術以下三者もおのずからその所有に帰して、国民の力と政府の力と互いに相平均し、もって全国の独立を維持すべきなり。

以上論ずるところを概すれば、今の世の学者、この国の独立を助けなさんとするに当たりて、政府の範囲に入り官にありて事をなすと、その範囲を脱して私立するとの利害得失を述べ、本論は私立に左袒したるものなり。すべて世の事物をくわしく論ずれば、利あらざるものは必ず害あり、得あらざるものは必ず失あり、利害得失相半ばするものはあるべからず。わが輩もとよりためにするところありて私立を主張するにあらず、ただ平生の所見を証してこれを論じたるのみ。世人もし確証を掲げてこの論説を排し明らかに私立の不利を述ぶる者あらば、余輩は悦んでこれに従い、天下の害をなすことなかるべし。

\subsection{付録}
本論につき二、三の問答あり、よってこれを巻末に記す。

その一にいわく、「事をなすは有力なる政府によるの便利に若かず」と。答えていわく、「文明を進むるはひとり政府の力のみに依頼すべからず、その弁論すでに本文に明らかなり。かつ政府にて事をなすはすでに数年の実験あれどもいまだその奏功を見ず、あるいは私の事もはたしてその功を期し難しといえども、議論上において明らかに見込みあればこれを試みざるべからず。いまだ試みずしてまずその成否を疑う者はこれを勇者と言うべからず」

二にいわく、「政府、人に乏し、有力の人物、政府を離れなば官務に差しつかえあるべし」と。答えていわく、けっして然らず、今の政府は官員の多きを患うるなり。事を簡にして官員を減ずれば、その事務はよく整理してその人員は世間の用をなすべし、一挙して両得なり。ことさらに政府の事務を多端にし、有用の人を取りて無用の事をなさしむるは策の拙なるものと言うべし。かつこの人物政府を離るるも去りて外国に行くにあらず、日本に居て日本の事をなすのみ、なんぞ患うるに足らん」

三にいわく、「政府のほかに私立の人物、集まることあらば、おのずから政府のごとくなりて、本政府の権を落とすに至らん」と。答えていわく、「この説は小人の説なり。私立の人も在官の人も等しく日本人なり。ただ地位を異にして事をなすのみ。その実は相助けてともに全国の便利を謀るものなれば、敵にあらず真の益友なり。かつこの私立の人物なる者、法を犯すことあらばこれを罰して可なり、毫も恐るるに足らず」

四にいわく、「私立せんと欲する人物あるも、官途を離るれば他に活計の道なし」と。答えていわく、「この言は士君子の言うべきにあらず。すでにみずから学者と唱えて天下の事を患うる者、豈無芸の人物あらんや。芸をもって口を糊するは難きにあらず。かつ官にありて公務を司るも私にいて業を営むも、その難易、異なるの理なし。もし官の事務易くしてその利益私の営業よりも多きことあらば、すなわちその利益は働きの実に過ぎたるものと言うべし。実に過ぐるの利を貪るは君子のなさざるところなり。無芸無能、僥倖によりて官途につき、みだりに給料を貪りて奢侈の資となし、戯れに天下のことを談ずる者はわが輩の友にあらず」

\section{五編}
『学問のすすめ』はもと民間の読本または小学の教授本に供えたるものなれば、初編より二編三編までも勉めて俗語を用い文章を読みやすくするを趣意となしたりしが、四編に至り少しく文の体を改めてあるいはむずかしき文字を用いたるところもあり。またこの五編も明治七年一月一日、社中会同の時に述べたる詞を文章に記したるものなれば、その文の体裁も四編に異ならずしてあるいは解し難きの恐れなきにあらず。畢竟四、五の二編は学者を相手にして論を立てしものなるゆえ、この次第に及びたるなり。

世の学者はおおむねみな腰ぬけにてその気力は不慥かなれども、文字を見る眼はなかなか慥かにして、いかなる難文にても困る者なきゆえ、この二冊にも遠慮なく文章をむずかしく書きその意味もおのずから高上になりて、これがためもと民間の読本たるべき学問のすすめの趣意を失いしは、初学の輩に対してはなはだ気の毒なれども、六編より後はまたもとの体裁に復り、もっぱら解しやすきを主として初学の便利に供しさらに難文を用いることなかるべきがゆえに、看官この二冊をもって全部の難易を評するなかれ。

\subsection{明治七年一月一日の詞}
わが輩今日慶応義塾にありて明治七年一月一日に逢えり。この年号はわが国独立の年号なり、この塾はわが社中独立の塾なり。独立の塾に居て独立の新年に逢うを得るはまた悦ばしからずや。けだしこれを得て悦ぶべきものは、これを失えば悲しみとなるべし。ゆえに今日悦ぶの時において他日悲しむの時あるを忘るべからず。

古来わが国治乱の沿革により政府はしばしば改まりたれども、今日に至るまで国の独立を失わざりし所以は、国民鎖国の風習に安んじ、治乱興廃、外国に関することなかりしをもってなり。外国に関係あらざれば、治も一国内の治なり、乱も一国内の乱なり、またこの治乱を経て失わざりし独立もただ一国内の独立にて、いまだ他に対して鋒を争いしものにあらず。これを譬えば、小児の家内に育せられていまだ外人に接せざる者のごとし。その薄弱なることもとより知るべきなり。

今や外国の交際にわかに開け、国内の事務一としてこれに関せざるものなし。事々物々みな外国に比較して処置せざるべからざるの勢いに至り、古来わが国人の力にてわずかに達し得たる文明の有様をもって、西洋諸国の有様に比すれば、ただに三舎を譲るのみならず、これに倣わんとしてあるいは望洋の歎を免れず、ますますわが独立の薄弱なるを覚ゆるなり。

国の文明は形をもって評すべからず。学校と言い、工業と言い、陸軍と言い、海軍と言うも、みなこれ文明の形のみ。この形を作るは難きにあらず、ただ銭をもって買うべしといえども、ここにまた無形の一物あり、この物たるや、目見るべからず、耳聞くべからず、売買すべからず、貸借すべからず、あまねく国人の間に位してその作用はなはだ強く、この物あらざればかの学校以下の諸件も実の用をなさず、真にこれを文明の精神と言うべき至大至重のものなり。けだしその物とはなんぞや。いわく、人民独立の気力、すなわちこれなり。

近来わが政府、しきりに学校を建て工業を勧め、海陸軍の制も大いに面目を改め、文明の形、ほぼ備わりたれども、人民いまだ外国へ対してわが独立を固くしともに先を争わんとする者なし。ただにこれと争わざるのみならず、たまたまかの事情を知るべき機会を得たる人にても、いまだこれを詳らかにせずしてまずこれを恐るるのみ。他に対してすでに恐怖の心をいだくときは、たとい、我にいささか得るところあるもこれを外に施すに由なし。畢竟、人民に独立の気力あらざれば、かの文明の形もついに無用の長物に属するなり。

そもそもわが国の人民に気力なきその原因を尋ぬるに、数千百年の古より全国の権柄を政府の一手に握り、武備・文学より工業・商売に至るまで、人間些末の事務といえども政府の関わらざるものなく、人民はただ政府の嗾するところに向かいて奔走するのみ。あたかも国は政府の私有にして、人民は国の食客たるがごとし。すでに無宿の食客となりてわずかにこの国中に寄食するを得るものなれば、国を視ること逆旅のごとく、かつて深切の意を尽くすことなく、またその気力を見わすべき機会をも得ずして、ついに全国の気風を養いなしたるなり。

しかのみならず今日に至りては、なおこれよりはなはだしきことあり。おおよそ世間の事物、進まざる者は必ず退き、退かざる者は必ず進む。進まず退かずして潴滞する者はあるべからざるの理なり。今、日本の有様を見るに、文明の形は進むに似たれども、文明の精神たる人民の気力は日に退歩に赴けり。請う、試みにこれを論ぜん。在昔、足利・徳川の政府においては民を御するにただ力を用い、人民の政府に服するは力足らざればなり。力足らざる者は心服するにあらず、ただこれを恐れて服従の容をなすのみ。今の政府はただ力あるのみならず、その智恵すこぶる敏捷にして、かつて事の機に後るることなし。一新の後、いまだ十年ならずして、学校・兵備の改革あり、鉄道・電信の設あり、その他石室を作り、鉄橋を架する等、その決断の神速なるとその成功の美なるとに至りては、実に人の耳目を驚かすに足れり。しかるにこの学校・兵備は、政府の学校・兵備なり、鉄道・電信も、政府の鉄道・電信なり、石室・鉄橋も、政府の石室・鉄橋なり。人民はたしてなんの観をなすべきや。人みな言わん、「政府はただに力あるのみならず兼ねてまた智あり、わが輩の遠く及ぶところにあらず、政府は雲上にありて国を司り、わが輩は下にいてこれに依頼するのみ、国を患うるは上の任なり、下賤の関わるところにあらず」と。概してこれを言えば、古の政府は力を用い、今の政府は力と智とを用ゆ。古の政府は民を御するの術に乏しく、今の政府はこれに富めり。古の政府は民の力を挫き、今の政府はその心を奪う。古の政府は民の外を犯し、今の政府はその内を制す。古の民は政府を視ること鬼のごとくし、今の民はこれを視ること神のごとくす。古の民は政府を恐れ、今の民は政府を拝む。この勢いに乗じて事の轍を改むることなくば、政府にて一事を起こせば文明の形はしだいに具わるに似たれども、人民にはまさしく一段の気力を失い文明の精神はしだいに衰うるのみ。

いま政府に常備の兵隊あり、人民これを認めて護国の兵となし、その盛んなるを祝して意気揚々たるべきはずなるに、かえってこれを威民の具とみなして恐怖するのみ。いま政府に学校、鉄道あり、人民これを一国文明の徴として誇るべきはずなるに、かえってこれを政府の私恩に帰し、ますますその賜に依頼するの心を増すのみ。人民すでに自国の政府に対して萎縮震慄の心をいだけり、豈外国に競うて文明を争うに遑あらんや。ゆえにいわく、人民に独立の気力あらざれば文明の形を作るもただに無用の長物のみならず、かえって民心を退縮せしむるの具となるべきなり。

右に論ずるところをもって考うれば、国の文明は上政府より起こるべからず、下小民より生ずべからず、必ずその中間より興りて衆庶の向かうところを示し、政府と並び立ちてはじめて成功を期すべきなり。西洋諸国の史類を案ずるに、商売・工業の道、一として政府の創造せしものなし、その本はみな中等の地位にある学者の心匠に成りしもののみ。蒸気機関はワットの発明なり、鉄道はステフェンソンの工夫なり、はじめて経済の定則を論じ商売の法を一変したるはアダム・スミスの功なり。この諸大家はいわゆるミッヅル・カラッスなる者にて、国の執政にあらず、また力役の小民にあらず、まさに国人の中等に位し、智力をもって一世を指揮したる者なり。その工夫発明、まず一人の心に成れば、これを公にして実地に施すには私立の社友を結び、ますますその事を盛大にして人民無量の幸福を万世に遺すなり。この間に当たり政府の義務は、ただその事を妨げずして適宜に行なわれしめ、人心の向かうところを察してこれを保護するのみ。

ゆえに文明の事を行なう者は私立の人民にして、その文明を護する者は政府なり。これをもって一国の人民あたかもその文明を私有し、これを競いこれを争い、これを羨みこれを誇り、国に一の美事あれば全国の人民手を拍ちて快と称し、ただ他国に先鞭を着けられんことを恐るるのみ。ゆえに文明の事物悉皆人民の気力を増すの具となり、一事一物も国の独立を助けざるものなし。その事情まさしくわが国の有様に相反すと言うも可なり。

今わが国においてかのミッヅル・カラッスの地位に居り、文明を首唱して国の独立を維持すべき者はただ一種の学者のみなれども、この学者なるもの時勢につき眼を着すること高からざるか、あるいは国を患うること身を患うるがごとく切ならざるか、あるいは世の気風に酔いひたすら政府に依頼して事をなすべきものと思うか、おおむね皆その地位に安んぜずして去りて官途に赴き、些末の事務に奔走していたずらに身心を労し、その挙動笑うべきもの多しといえども、みずからこれを甘んじ、人もまたこれを怪しまず、はなはだしきは「野に遺賢なし」と言いてこれを悦ぶ者あり。もとより時勢の然らしむるところにて、その罪一個の人にあらずといえども、国の文明のためには一大災難と言うべし。文明を養いなすべき任に当たりたる学者にして、その精神の日に衰うるを傍観してこれを患うる者なきは、実に長大息すべきなり、また痛哭すべきなり。

ひとりわが慶応義塾の社中はわずかにこの災難を免れて、数年独立の名を失わず、独立の塾にいて独立の気を養い、その期するところは全国の独立を維持するの一事にあり。然りといえども、時勢の世を制するやその力急流のごとくまた大風のごとし。この勢いに激して屹立するはもとより易きにあらず、非常の勇力あるにあらざれば、知らずして流れ識らずして靡き、ややもすればその脚を失するの恐れあるべし。そもそも人の勇力はただ読書のみによりて得べきものにあらず。読書は学問の術なり、学問は事をなすの術なり。実地に接して事に慣るるにあらざればけっして勇力を生ずべからず。わが社中すでにその術を得たる者は、貧苦を忍び艱難を冒して、その所得の知見を文明の事実に施さざるべからず。その科は枚挙に遑あらず。商売勤めざるべからず、法律議せざるべからず、工業起こさざるべからず、農業勧めざるべからず、著書・訳術・新聞の出版、およそ文明の事件はことごとく取りてわが私有となし、国民の先をなして政府と相助け、官の力と私の力と互いに平均して一国全体の力を増し、かの薄弱なる独立を移して動かすべからざるの基礎に置き、外国と鋒を争いて毫も譲ることなく、今より数十の新年を経て、顧みて今月今日の有様を回想し、今日の独立を悦ばずしてかえってこれを愍笑するの勢いに至るは、豈一大快事ならずや。学者よろしくその方向を定めて期するところあるべきなり。

\section{六編}
\subsection{国法の貴きを論ず}
政府は国民の名代にて、国民の思うところに従い事をなすものなり。その職分は罪ある者を取り押えて罪なき者を保護するよりほかならず。すなわちこれ国民の思うところにして、この趣意を達すれば一国内の便利となるべし。元来罪ある者とは悪人なり、罪なき者とは善人なり。いま悪人来たりて善人を害せんとすることあらば、善人みずからこれを防ぎ、わが父母妻子を殺さんとする者あらば捕えてこれを殺し、わが家財を盗まんとする者あらば捕えてこれを笞うち、差しつかえなき理なれども、一人の力にて多勢の悪人を相手にとり、これを防がんとするも、とても叶うべきことにあらず。

たとい、あるいはその手当てをなすも莫大の入費にて益もなきことなるゆえ、右のごとく国民の総代として政府を立て、善人保護の職分を勤めしめ、その代わりとして役人の給料はもちろん、政府の諸入用をば悉皆国民より賄うべしと約束せしことなり。かつまた政府はすでに国民の総名代となりて事をなすべき権を得たるものなれば、政府のなすことはすなわち国民のなすことにて、国民は必ず政府の法に従わざるべからず。これまた国民と政府との約束なり。ゆえに国民の政府に従うは政府の作りし法に従うにあらず、みずから作りし法に従うなり。国民の法を破るは政府の作りし法を破るにあらず、みずから作りし法を破るなり。その法を破りて刑罰を被るは政府に罰せらるるにあらず、みずから定めし法によりて罰せらるるなり。この趣を形容して言えば、国民たる者は一人にて二人前の役目を勤むるがごとし。すなわちその一の役目は、自分の名代として政府を立て、一国中の悪人を取り押えて善人を保護することなり。その二の役目は、固く政府の約束を守りその法に従いて保護を受くることなり。

右のごとく、国民は政府と約束して政令の権柄を政府に任せたる者なれば、かりそめにもこの約束を違えて法に背くべからず。人を殺す者を捕えて死刑に行なうも政府の権なり、盗賊を縛りて獄屋に繋ぐも政府の権なり、公事訴訟を捌くも政府の権なり、乱暴・喧嘩を取り押うるも政府の権なり。これらの事につき、国民は少しも手を出だすべからず。もし心得違いして私に罪人を殺し、あるいは盗賊を捕えてこれを笞うつ等のことあれば、すなわち国の法を犯し、みずから私に他人の罪を裁決する者にて、これを私裁と名づけ、その罪免すべからず。この一段に至りては文明諸国の法律はなはだ厳重なり。いわゆる威ありて猛からざるものか。わが日本にては政府の威権盛んなるに似たれども、人民ただ政府の貴きを恐れてその法の貴きを知らざる者あり。今ここに私裁のよろしからざる所以と国法の貴き所以とを記すこと左のごとし。

譬えばわが家に強盗の入り来たりて、家内の者を威し金を奪わんとすることあらん。この時に当たり、家の主人たる者の職分は、この事の次第を政府に訴え、政府の処置を待つべきはずなれども、事火急にして出訴の間合いもなく、かれこれするうちにかの強盗はすでに土蔵へ這入りて金を持ち出さんとするの勢いあり。これを止めんとすれば主人の命も危き場合なるゆえ、やむを得ず家内申し合わせて私にこれを防ぎ、当座の取り計らいにてこの強盗を捕え置き、しかる後に政府へ訴え出ずるなり。これを捕うるにつきては、あるいは棒を用い、あるいは刃物を用い、あるいは賊の身に疵つくることもあるべし、あるいはその足を打ち折ることもあるべし、事急なるときは鉄砲をもって打ち殺すこともあるべしといえども、結局主人たる者は、わが生命を護り、わが家財を守るために一時の取り計らいをなしたるのみにて、けっして賊の無礼を咎め、その罪を罰するの趣意にあらず。

罪人を罰するは政府に限りたる権なり、私の職分にあらず。ゆえに私の力にてすでにこの強盗を取り押え、わが手に入りしうえは、平人の身としてこれを殺しこれを打擲すべからざるはもちろん、指一本を賊の身に加うることをも許さず、ただ政府に告げて政府の裁判を待つのみ。もしも賊を取り押えしうえにて、怒りに乗じてこれを殺し、これを打擲することあれば、その罪は無罪の人を殺し、無罪の人を打擲するに異ならず。譬えば某国の律に、「金十円を盗む者はその刑、笞一百、また足をもって人の面を蹴る者もその刑、笞一百」とあり。しかるにここに盗賊ありて、人の家に入り金十円を盗み得て出でんとするとき、主人に取り押えられ、すでに縛られしうえにて、その主人なおも怒りに乗じ足をもって賊の面を蹴ることあらん、しかるときその国の律をもってこれを論ずれば、賊は金十円を盗みし罪にて一百の笞を被り、主人もまた平人の身をもって私に賊の罪を裁決し足をもってその面を蹴りたる罪により笞うたるること一百なるべし。国法の厳なることかくのごとし。人々恐れざるべからず。

右の理をもって考うれば敵討ちのよろしからざることも合点すべし。わが親を殺したる者はすなわちその国にて一人の人を殺したる公の罪人なり。この罪人を捕えて刑に処するは政府に限りたる職分にて、平人の関わるところにあらず。しかるにその殺されたる者の子なればとて、政府に代わりて私にこの公の罪人を殺すの理あらんや。差し出がましき挙動と言うべきのみならず、国民たるの職分を誤り、政府の約束に背くものと言うべし。もしこのことにつき、政府の処置よろしからずして罪人を贔屓するなどのことあらば、その不筋なる次第を政府に訴うべきのみ。なんらの事故あるもけっしてみずから手を出だすべからず。たとい親の敵は目の前に徘徊するも私にこれを殺すの理なし。

昔、徳川の時代に、浅野家の家来、主人の敵討ちとて吉良上野介を殺したることあり。世にこれを赤穂の義士と唱えり。大なる間違いならずや。この時日本の政府は徳川なり。浅野内匠頭も吉良上野介も浅野家の家来もみな日本の国民にて、政府の法に従いその保護を蒙るべしと約束したるものなり。しかるに一朝の間違いにて上野介なる者内匠頭へ無礼を加えしに、内匠頭これを政府に訴うることを知らず、怒りに乗じて私に上野介を切らんとしてついに双方の喧嘩となりしかば、徳川政府の裁判にて内匠頭へ切腹を申しつけ、上野介へは刑を加えず、この一条は実に不正なる裁判というべし。浅野家の家来どもこの裁判を不正なりと思わば、何がゆえにこれを政府へ訴えざるや。四十七士の面々申し合わせて、おのおのその筋により法に従いて政府に訴え出でなば、もとより暴政府のことゆえ、最初はその訴訟を取り上げず、あるいはその人を捕えてこれを殺すこともあるべしといえども、たとい一人は殺さるるもこれを恐れず、また代わりて訴え出で、したがって殺されしたがって訴え、四十七人の家来、理を訴えて命を失い尽くすに至らば、いかなる悪政府にてもついには必ずその理に伏し、上野介へも刑を加えて裁判を正しゅうすることあるべし。

かくありてこそはじめて真の義士とも称すべきはずなるに、かつてこの理を知らず、身は国民の地位にいながら国法の重きを顧みずしてみだりに上野介を殺したるは、国民の職分を誤り、政府の権を犯して、私に人の罪を裁決したるものと言うべし。幸いにしてその時、徳川の政府にてこの乱暴人を刑に処したればこそ無事に治まりたれども、もしもこれを免すことあらば、吉良家の一族また敵討ちとて赤穂の家来を殺すことは必定なり。しかるときはこの家来の一族、また敵討ちとて吉良の一族を攻むるならん。敵討ちと敵討ちとにて、はてしもあらず、ついに双方の一族朋友死し尽くるに至らざれば止まず。いわゆる無政無法の世の中とはこのことなるべし。私裁の国を害することかくのごとし。謹まざるべからざるなり。

古は日本にて百姓・町人の輩、士分の者に対して無礼を加うれば切捨て御免という法あり。こは政府より公に私裁を許したるものなり。けしからぬことならずや。すべて一国の法はただ一政府にて施行すべきものにて、その法の出ずるところいよいよ多ければその権力もまたしたがっていよいよ弱し。譬えば封建の世に三百の諸侯おのおの生殺の権ありし時は、政府の力もその割合にて弱かりしはずなり。

私裁のもっともはなはだしくして、政を害するのもっとも大なるものは暗殺なり。古来暗殺の事跡を見るに、あるいは私怨のためにする者あり、あるいは銭を奪わんがためにする者あり。この類の暗殺を企つるものはもとより罪を犯す覚悟にて、自分にも罪人のつもりなれども、別にまた一種の暗殺あり。この暗殺は私のためにあらず、いわゆるポリチカル・エネミ〔政敵〕を悪んでこれを殺すものなり。天下の事につき銘々の見込みを異にし、私の見込みをもって他人の罪を裁決し、政府の権を犯して恣に人を殺し、これを恥じざるのみならずかえって得意の色をなし、みずから天誅を行なうと唱うれば、人またこれを称して報国の士と言う者あり。そもそも天誅とは何事なるや。天に代わりて誅罰を行なうというつもりか。もしそのつもりならば、まず自分の身の有様を考えざるべからず。元来この国に居り、政府へ対していかなる約定を結びしや。「必ずその国法を守りて身の保護を被るべし」とこそ約束したることなるべし。もし国の政事につき不平の箇条を見いだし、国を害する人物ありと思わば、静かにこれを政府へ訴うべきはずなるに、政府を差し置き、みずから天に代わりて事をなすとは商売違いもまたはなはだしきものと言うべし。畢竟この類の人は、性質律儀なれども物事の理に暗く、国を患うるを知りて国を患うる所以の道を知らざる者なり。試みに見よ、天下古今の実験に、暗殺をもってよく事をなし世間の幸福を増したるものは、いまだかつてこれあらざるなり。

国法の貴きを知らざる者は、ただ政府の役人を恐れ、役人の前をほどよくして、表向きに犯罪の名あらざれば内実の罪を犯すもこれを恥とせず。ただにこれを恥じざるのみならず、巧みに法を破りて罪を遁るる者あればかえってこれをその人の働きとしてよき評判を得ることあり。今、世間日常の話に、此も上の御大法なり、彼も政府の表向きなれども、この事を行なうにかく私に取り計らえば、表向きの御大法には差しつかえもあらず、表向きの内証などとて笑いながら談話して咎むるものもなく、はなはだしきは小役人と相談のうえ、この内証事を取り計らい、双方ともに便利を得て罪なき者のごとし。実はかの御大法なるもの、あまり煩わしきに過ぎて事実に施すべからざるよりして、この内証事も行なわるることなるべしといえども、一国の政治をもってこれを論ずれば、もっとも恐るべき悪弊なり。かく国法を軽蔑するの風に慣れ、人民一般に不誠実の気を生じ、守りて便利なるべき法をも守らずして、ついには罪を蒙ることあり。

譬えば今往来に小便するは政府の禁制なり。しかるに人民みなこの禁令の貴きを知らずしてただ邏卒を恐るるのみ。あるいは日暮れなど邏卒のあらざるを窺いて法を破らんとし、はからずも見咎めらるることあればその罪に伏すといえども、本人の心中には貴き国法を犯したるがゆえに罰せらるるとは思わずして、ただ恐ろしき邏卒に逢いしをその日の不幸と思うのみ。実に歎かわしきことならずや。ゆえに政府にて法を立つるは勉めて簡なるを良とす。すでにこれを定めて法となすうえは必ず厳にその趣意を達せざるべからず。人民は政府の定めたる法を見て不便なりと思うことあらば、遠慮なくこれを論じて訴うべし。すでにこれを認めてその法の下に居るときは、私にその法を是非することなく謹んでこれを守らざるべからず。

近くは先月わが慶応義塾にも一事あり。華族太田資美君、一昨年より私金を投じて米国人を雇い、義塾の教員に供えしが、このたび交代の期限に至り、他の米人を雇い入れんとして、当人との内談すでに整いしにつき、太田氏より東京府へ書面を出だしこの米人を義塾に入れて文学・科学の教師に供えんとの趣を出願せしところ、文部省の規則に、「私金をもって私塾の教師を雇い、私に人を教育するものにても、その教師なる者、本国にて学科卒業の免状を得てこれを所持するものにあらざれば雇入れを許さず」との箇条あり。しかるにこのたび雇い入れんとする米人、かの免状を所持せざるにつき、ただ語学の教師とあればともかくもなれども、文学・科学の教師としては願いの趣、聞き届け難き旨、東京府より太田氏へ御沙汰なり。

よって福沢諭吉より同府へ書を呈し、「この教師なる者、免状を所持せざるもその学力は当塾の生徒を教うるため十分なるゆえ、太田氏の願いのとおりに命ぜられたく、あるいは語学の教師と申し立てなば願いも済むべきなれども、もとよりわが生徒は文学・科学を学ぶつもりなれば、語学と偽り官を欺くことはあえてせざるところなり」と出願したりしかども、文部省の規則変ずべからざる由にて、諭吉の願書もまた返却したり。これがためすでに内約の整いし教師を雇い入るるを得ず、去年十二月下旬、本人は去りて米国へ帰り、太田君の素志も一時の水の泡となり、数百の生徒も望みを失い、実に一私塾の不幸のみならず、天下文学のためにも大なる妨げにて、馬鹿らしく苦々しきことなれども、国法の貴重なる、これを如何ともすべからず、いずれ近日また重ねて出願のつもりなり。今般の一条につきては、太田氏をはじめ社中集会してその内話に、「かの文部省にて定めたる私塾教師の規則もいわゆる御大法なれば、ただ文学・科学の文字を消して語学の二字に改むれば、願いも済み、生徒のためには大幸ならん」と再三商議したれども、結局のところ、このたびの教師を得ずして社中生徒の学業あるいは退歩することあるも、官を欺くは士君子の恥ずべきところなれば、謹んで法を守り国民たるの分を誤らざるの方、上策なるべしとて、ついにこの始末に及びしことなり。もとより一私塾の処置にてこのこと些末に似たれども、議論の趣意は世教にも関わるべきことと思い、ついでながらこれを巻末に記すのみ。

\section{七編}
\subsection{国民の職分を論ず}
第六編に国法の貴きを論じ、「国民たる者は一人にて二人前の役目を勤むるものなり」と言えり。今またこの役目職分の事につき、なおその詳らかなるを説きて六編の補遺となすこと左のごとし。

およそ国民たる者は一人の身にして二ヵ条の勤めあり。その一の勤めは政府の下に立つ一人の民たるところにてこれを論ず、すなわち客のつもりなり。その二の勤めは国中の人民申し合わせて、一国と名づくる会社を結び、社の法を立ててこれを施し行なうことなり、すなわち主人のつもりなり。譬えばここに百人の町人ありてなんとかいう商社を結び、社中相談のうえにて社の法を立て、これを施し行なうところを見れば、百人の人はその商社の主人なり。すでにこの法を定めて、社中の人いずれもこれに従い違背せざるところを見れば、百人の人は商社の客なり。ゆえに一国はなお商社のごとく、人民はなお社中の人のごとく、一人にて主客二様の職を勤むべき者なり。

第一 客の身分をもって論ずれば、一国の人民は国法を重んじ人間同等の趣意を忘るべからず。他人の来たりてわが権義を害するを欲せざれば、われもまた他人の権義を妨ぐべからず。わが楽しむところのものは他人もまたこれを楽しむがゆえに、他人の楽しみを奪いてわが楽しみを増すべからず、他人の物を盗んでわが富となすべからず、人を殺すべからず、人を讒すべからず、まさしく国法を守りて彼我同等の大義に従うべし。また国の政体によりて定まりし法は、たといあるいは愚かなるも、あるいは不便なるも、みだりにこれを破るの理なし。師を起こすも外国と条約を結ぶも政府の権にあることにて、この権はもと約束にて人民より政府へ与えたるものなれば、政府の政に関係なき者はけっしてそのことを評議すべからず。

人民もしこの趣意を忘れて、政府の処置につきわが意に叶わずとて恣に議論を起こし、あるいは条約を破らんとし、あるいは師を起こさんとし、はなはだしきは一騎先駆け、自刃を携えて飛び出すなどの挙動に及ぶことあらば、国の政は一日も保つべからず。これを譬えばかの百人の商社兼ねて申し合せのうえ、社中の人物十人を選んで会社の支配人と定め置き、その支配人の処置につき、残り九十人の者どもわが意に叶わずとて銘々に商法を議し、支配人は酒を売らんとすれば九十人の者は牡丹餅を仕入れんとし、その評議区々にて、はなはだしきは一了簡をもって私に牡丹餅の取引きを始め、商社の法に背きて他人と争論に及ぶなどのことあらば、会社の商売は一日も行なわるべからず。ついにその商社の分散するに至らば、その損亡は商社百人一様の引受けなるべし。愚もまたはなはだしきものと言うべし。ゆえに国法は不正不便なりといえども、その不正不便を口実に設けてこれを破るの理なし。もし事実において不正不便の箇条あらば、一国の支配人たる政府に説き勧めて静かにその法を改めしむべし。政府もしわが説に従わずんば、かつ力を尽くしかつ堪忍して時節を待つべきなり。

第二 主人の身分をもって論ずれば、一国の人民はすなわち政府なり。そのゆえは一国中の人民悉皆政をなすべきものにあらざれば、政府なるものを設けてこれに国政を任せ、人民の名代として事務を取り扱わしむべしとの約束を定めたればなり。ゆえに人民は家元なり、また主人なり。政府は名代人なり、また支配人なり。譬えば商社百人のうちより選ばれたる十人の支配人は政府にて、残り九十人の社中は人民なるがごとし。この九十人の社中は自分にて事務を取り扱うことなしといえども、己が代人として十人の者へ事を任せたるゆえ、己れの身分を尋ぬればこれを商社の主人と言わざるを得ず。またかの十人の支配人は現在の事を取り扱うといえども、もと社中の頼みを受けその意に従いて事をなすべしと約束したる者なれば、その実は私にあらず、商社の公務を勤むる者なり。いま世間にて政府に関わることを公務と言い公用と言うも、その字のよって来たるところを尋ぬれば、政府の事は役人の私事にあらず、国民の名代となりて一国を支配する公の事務という義なり。

右の次第をもって、政府たるものは人民の委任を引き受け、その約束に従いて一国の人をして貴賤上下の別なくいずれもその権義を逞しゅうせしめざるべからず、法を正しゅうし罰を厳にして一点の私曲あるべからず。今ここに一群の賊徒来たりて人の家に乱入するとき、政府これを見てこれを制すること能わざれば政府もその賊の徒党と言いて可なり。政府もし国法の趣意を達すること能わずして人民に損亡を蒙らしむることあらば、その高の多少を論ぜずその事の新旧を問わず、必ずこれを償わざるべからず。譬えば役人の不行届きにて国内の人か、または外国人へ損亡をかけ、三万円の償金を払うことあらん。政府にはもとより金のあるべき理なければ、その償金の出ずるところはかならず人民なり。この三万円を日本国中およそ三千万人の人口に割り付くれば、一人前十文ずつに当たる。役人の不行届き十度を重ぬれば、人民の出金一人前百文に当たり、家内五人の家なれば五百文なり。田舎の小百姓に五百文の銭あれば、妻子打ち寄り、山家相応の馳走を設けて一夕の愉快を尽くすべきはずなるに、ただ役人の不行届きのみにより、全日本国中無辜の小民をしてその無上の歓楽を失わしむるは実に気の毒の至りならずや。人民の身としてはかかる馬鹿らしき金を出だすべき理なきに似たれども、如何せん、その人民は国の家元主人にて、最初より政府へこの国を任せて事務を取り扱わしむるの約束をなし、損得ともに家元にて引き受くべきはずのものなれば、ただ金を失いしときのみに当たりて、役人の不調法をかれこれと議論すべからず。ゆえに人民たる者は平生よりよく心を用い、政府の処置を見て不安心と思うことあらば、深切にこれを告げ、遠慮なく穏やかに論ずべきなり。

人民はすでに一国の家元にて、国を護るための入用を払うはもとよりその職分なれば、この入用を出だすにつきけっして不平の顔色を見わすべからず。国を護るためには役人の給料なかるべからず、海陸の軍費なかるべからず、裁判所の入用もあり、地方官の入用もあり、その高を集めてこれを見れば大金のように思わるれども、一人前の頭に割り付けてなにほどなるや。日本にて歳入の高を全国の人口に割り付けなば、一人前に一円か二円なるべし。一年の間にわずか一、二円の金を払うて政府の保護を被り、夜盗押込みの患いもなく、ひとり旅行に山賊の恐れもなくして、安穏にこの世を渡るは大なる便利ならずや。およそ世の中に割合よき商売ありといえども、運上を払うて政府の保護を買うほど安きものはなかるべし。世上の有様を見るに、普請に金を費やす者あり、美服美食に力を尽くす者あり、はなはだしきは酒色のために銭を棄てて身代を傾くる者もあり、これらの費えをもって運上の高に比較しなば、もとより同日の話にあらず、不筋の金なればこそ一銭をも惜しむべけれども、道理において出だすべきはずのみならず、これを出だして安きものを買うべき銭なれば、思案にも及ばず快く運上を払うべきなり。

右のごとく人民も政府もおのおのその分限を尽くして互いに居り合うときは申し分もなきことなれども、あるいは然らずして政府なるものその分限を越えて暴政を行なうことあり。ここに至りて人民の分としてなすべき挙動はただ三ヵ条あるのみ。すなわち節を屈して政府に従うか、力をもって政府に敵対するか、正理を守りて身を棄つるか、この三ヵ条なり。

第一 節を屈して政府に従うははなはだよろしからず。人たる者は天の正道に従うをもって職分とす。しかるにその節を屈して政府人造の悪法に従うは、人たるの職分を破るものと言うべし。かつひとたび節を屈して不正の法に従うときは、後世子孫に悪例を遺して天下一般の弊風を醸しなすべし。古来日本にても愚民の上に暴政府ありて、政府虚威を逞しゅうすれば人民はこれに震い恐れ、あるいは政府の処置を見て現に無理とは思いながら、事の理非を明らかに述べなば必ずその怒りに触れ、後日に至りて暗に役人らに窘しめらるることあらんを恐れて言うべきことをも言うものなし。その後日の恐れとは俗にいわゆる犬の糞でかたきなるものにて、人民はひたすらこの犬の糞を憚り、いかなる無理にても政府の命には従うべきものと心得て、世上一般の気風をなし、ついに今日の浅ましき有様に陥りたるなり。すなわちこれ人民の節を屈して禍を後世に残したる一例と言うべし。

第二 力をもって政府に敵対するはもとより一人の能くするところにあらず、必ず徒党を結ばざるべからず。すなわちこれ内乱の師なり。けっしてこれを上策というべからず。すでに師を起こして政府に敵するときは、事の理非曲直はしばらく論ぜずして、ただ力の強弱のみを比較せざるべからず。しかるに古今内乱の歴史を見れば、人民の力はつねに政府よりも弱きものなり。また内乱を起こせば、従来その国に行なわれたる政治の仕組みをひとたび覆えすはもとより論を俟たず。しかるにその旧の政府なるもの、たといいかなる悪政府にても、おのずからまた善政良法あるにあらざれば政府の名をもって若干の年月を渡るべき理なし。

ゆえに一朝の妄動にてこれを倒すも、暴をもって暴に代え、愚をもって愚に代うるのみ。また内乱の源を尋ぬれば、もと人の不人情を悪みて起こしたるものなり。しかるにおよそ人間世界に内乱ほど不人情なるものはなし。世間朋友の交わりを破るはもちろん、はなはだしきは親子相殺し兄弟相敵し、家を焼き人を屠り、その悪事至らざるところなし。かかる恐ろしき有様にて人の心はますます残忍に陥り、ほとんど禽獣とも言うべき挙動をなしながら、かえって旧の政府よりもよき政を行ない寛大なる法を施して天下の人情を厚きに導かんと欲するか。不都合なる考えと言うべし。

第三 正理を守りて身を棄つるとは、天の道理を信じて疑わず、いかなる暴政の下に居ていかなる苛酷の法に窘しめらるるも、その苦痛を忍びてわが志を挫くことなく、一寸の兵器を携えず片手の力を用いず、ただ正理を唱えて政府に迫ることなり。以上三策のうち、この第三策をもって上策の上とすべし。理をもって政府に迫れば、その時その国にある善政良法はこれがため少しも害を被ることなし。その正論あるいは用いられざることあるも、理のあるところはこの論によりてすでに明らかなれば、天然の人心これに服せざることなし。ゆえに今年に行なわれざればまた明年を期すべし。かつまた力をもって敵対するものは一を得んとして百を害するの患いあれども、理を唱えて政府に迫るものはただ除くべきの害を除くのみにて他に事を生ずることなし。その目的とするところは政府の不正を止むるの趣意なるがゆえに、政府の処置、正に帰すれば議論もまたともにやむべし。また力をもって政府に敵すれば、政府は必ず怒りの気を生じ、みずからその悪を顧みずしてかえってますます暴威を張り、その非を遂げんとするの勢いに至るべしといえども、静かに正理を唱うる者に対しては、たとい暴政府といえどもその役人もまた同国の人類なれば、正者の理を守りて身を棄つるを見て必ず同情相憐れむの心を生ずべし。すでに他を憐れむの心を生ずれば、おのずから過ちを悔い、おのずから胆を落として、必ず改心するに至るべし。

かくのごとく世を患いて身を苦しめあるいは命を落とすものを、西洋の語にてマルチルドムという。失うところのものはただ一人の身なれども、その功能は千万人を殺し千万両を費やしたる内乱の師よりもはるかに優れり。古来日本にて討死せし者も多く切腹せし者も多し、いずれも忠臣義士とて評判は高しといえども、その身を棄てたる所以を尋ぬるに、多くは両主政権を争うの師に関係する者か、または主人の敵討ちなどによりて花々しく一命を抛ちたる者のみ。その形は美に似たれどもその実は世に益することなし。己が主人のためと言い己が主人に申し訳なしとて、ただ一命をさえ棄つればよきものと思うは不文不明の世の常なれども、いま文明の大義をもってこれを論ずれば、これらの人はいまだ命の棄てどころを知らざる者と言うべし。元来、文明とは、人の智徳を進め、人々身みずからその身を支配して世間相交わり、相害することもなく害せらるることもなく、おのおのその権義を達して一般の安全繁盛を致すを言うなり。さればかの師にもせよ敵討ちにもせよ、はたしてこの文明の趣意に叶い、この師に勝ちてこの敵を滅ぼし、この敵討ちを遂げてこの主人の面目を立つれば、必ずこの世は文明に赴き、商売も行なわれ工業も起こりて、一般の安全繁盛を致すべしとの目的あらば、討死も敵討ちも尤ものようなれども、事柄においてけっしてその目的あるべからず。

かつかの忠臣義士にもそれほどの見込みはあるまじ。ただ因果ずくにて旦那へ申し訳までのことなるべし。旦那へ申し訳にて命を棄てたる者を忠臣義士と言わば、今日も世間にその人は多きものなり。権助が主人の使いに行き、一両の金を落として途方に暮れ、旦那へ申し訳なしとて思案を定め、並木の枝にふんどしを掛けて首を縊るの例は世に珍しからず。今この義僕がみずから死を決する時の心を酌んで、その情実を察すれば、また憐れむべきにあらずや。使いに出でていまだ帰らず、身まず死す。長く英雄をして涙を襟に満たしむべし。主人の委託を受けてみずから任じたる一両の金を失い、君臣の分を尽くすに一死をもってするは、古今の忠臣義士に対して毫も恥ずることなし。その誠忠は日月とともに燿き、その功名は天地とともに永かるべきはずなるに、世人みな薄情にしてこの権助を軽蔑し、碑の銘を作りてその功業を称する者もなく、宮殿を建てて祭る者もなきはなんぞや。人みな言わん、「権助の死はわずかに一両のためにしてその事の次第はなはだ些細なり」と。然りといえども事の軽重は金高の大小、人数の多少をもって論ずべからず、世の文明に益あると否とによりてその軽重を定むべきものなり。しかるに今かの忠臣義士が一万の敵を殺して討死するも、この権助が一両の金を失うて首を縊るも、その死をもって文明を益することなきに至りてはまさしく同様のわけにて、いずれを軽しとしいずれを重しとすべからざれば、義士も権助もともに命の棄てどころを知らざる者と言いて可なり。これらの挙動をもってマルチルドムと称すべからず。余輩の聞くところにて、人民の権義を主張し正理を唱えて政府に迫り、その命を棄てて終わりをよくし、世界中に対して恥ずることなかるべき者は、古来ただ一名の佐倉宗五郎あるのみ。ただし宗五郎の伝は俗間に伝わる草紙の類のみにて、いまだその詳らかなる正史を得ず。もし得ることあらば他日これを記してその功徳を表し、もって世人の亀鑑に供すべし。

\section{八編}
\subsection{わが心をもって他人の身を制すべからず}
アメリカのエイランドなる人の著わしたる『モラル・サイヤンス』という書に、人の身心の自由を論じたることあり。その論の大意にいわく、人の一身は他人と相離れて一人前の全体をなし、みずからその身を取り扱い、みずからその心を用い、みずから一人を支配して、務むべき仕事を務むるはずのものなり。ゆえに、第一、人にはおのおの身体あり。身体はもって外物に接し、その物を取りてわが求むるところを達すべし。譬えば種を蒔きて米を作り、綿を取りて衣服を製するがごとし。第二、人にはおのおの智恵あり。智恵はもって物の道理を発明し、事を成すの目途を誤ることなし。譬えば米を作るに肥しの法を考え、木綿を織るに機の工夫をするがごとし。みな智恵分別の働きなり。

第三、人にはおのおの情欲あり。情欲はもって心身の働きを起こし、この情欲を満足して一身の幸福をなすべし。たとえば人として美服美食を好まざる者なし。されどもこの美服美食はおのずから天地の間に生ずるものにあらず。これを得んとするには人の働きなかるべからず。ゆえに人の働きはたいていみな情欲の催促を受けて起こるものなり。この情欲あらざれば働きあるべからず、この働きあらざれば安楽の幸福あるべからず。禅坊主などは働きもなく幸福もなきものと言うべし。

第四、人にはおのおの至誠の本心あり。誠の心はもって情欲を制し、その方向を正しくして止まるところを定むべし。たとえば情欲には限りなきものにて、美服美食もいずれにて十分と界を定め難し。今もし働くべき仕事をば捨て置き、ひたすらわが欲するもののみを得んとせば、他人を害してわが身を利するよりほかに道なし。これを人間の所業と言うべからず。この時に当たりて欲と道理とを分別し、欲を離れて道理の内に入らしむるものは誠の本心なり。第五、人にはおのおの意思あり。意思はもって事をなすの志を立つべし。譬えば世の事は怪我の機にてできるものなし。善き事も悪き事もみな人のこれをなさんとする意ありてこそできるものなり。

以上、五つのものは人に欠くべからざる性質にして、この性質を自由自在に取り扱い、もって一身の独立をなすものなり。さて独立といえば、ひとり世の中の偏人奇物にて世間の付合いもなき者のように聞こゆれども、けっして然らず。人として世に居れば、もとより朋友なかるべからずといえども、その朋友もまたわれに交わりを求むることなおわが朋友を慕うがごとくなれば、世の交わりは相互いのことなり。ただこの五つの力を用うるに当たり、天より定めたる法に従いて、分限を越えざること緊要なるのみ。すなわちその分限とは、我もこの力を用い、他人もこの力を用いて、相互にその働きを妨げざるを言うなり。かくのごとく、人たる者の分限を誤らずして世を渡るときは、人に咎めらるることもなく、天に罪せらるることもなかるべし。これを人間の権義と言うなり。

右の次第により、人たる者は他人の権義を妨げざれば、自由自在に己が身体を用うるの理あり。その好むところに行き、その欲するところに止まり、あるいは働き、あるいは遊び、あるいはこの事を行ない、あるいはかの業をなし、あるいは昼夜勉強するも、あるいは意に叶わざれば無為にして終日寝るも、他人に関係なきことなれば、傍よりかれこれとこれを議論するの理なし。

今もし前の説に反し、「人たる者は理非にかかわらず他人の心に従いて事をなすものなり、わが了簡を出だすはよろしからず」という議論を立つる者あらん。この議論はたして理の当然なるか。理の当然ならばおよそ人と名のつきたる者の住居する世界には通用すべきはずなり。仮りにその一例を挙げて言わん。禁裏さまは公方さまよりも貴きものなるゆえ、禁裏さまの心をもって公方さまの身を勝手次第に動かし、行かんとすれば「止まれ」と言い、止まらんとすれば「行け」と言い、寝るも起きるも飲むも食うもわが思いのままに行なわるることなからん。公方さまはまた手下の大名を制し、自分の心をもって大名の身を自由自在に取り扱わん。大名はまた自分の心をもって家老の身を制し、家老は自分の心をもって用人の身を制し、用人は徒士を制し、徒士は足軽を制し、足軽は百姓を制するならん。

さて百姓に至りてはもはや目下の者もあらざれば少し当惑の次第なれども、元来この議論は人間世界に通用すべき当然の理に基づきたるものなれば、百万遍の道理にて、回れば本に返らざるを得ず。「百姓も人なり、禁裏さまも人なり、遠慮はなし」と御免を蒙り、百姓の心をもって禁裏さまの身を勝手次第に取り扱い、行幸あらんとすれば「止まれ」と言い、行在に止まらんとすれば「還御」と言い、起居眠食、みな百姓の思いのままにて、金衣玉食を廃して麦飯を進むるなどのことに至らば如何。かくのごときはすなわち日本国中の人民、身みずからその身を制するの権義なくしてかえって他人を制するの権あり。

人の身と心とはまったくその居処を別にして、その身はあたかも他人の魂を止むる旅宿のごとし。下戸の身に上戸の魂を入れ、子供の身に老人の魂を止め、盗賊の魂は孔夫子の身を借用し、猟師の魂は釈迦の身に旅宿し、下戸が酒を酌んで愉快を尽くせば、上戸は砂糖湯を飲んで満足を唱え、老人が樹に攀じて戯るれば、子供は杖をついて人の世話をやき、孔夫子が門人を率いて賊をなせば、釈迦如来は鉄砲を携えて殺生に行くならん。奇なり、妙なり、また不可思議なり。

これを天理人情と言わんか、これを文明開化と言わんか。三歳の童子にてもその返答は容易なるべし。数千百年の古より和漢の学者先生が、上下貴賤の名分とて喧しく言いしも、つまるところは他人の魂をわが身に入れんとするの趣向ならん。これを教えこれを説き、涙を流してこれを諭し、末世の今日に至りてはその功徳もようやく顕われ、大は小を制し強は弱を圧するの風俗となりたれば、学者先生も得意の色をなし、神代の諸尊、周の世の聖賢も、草葉の蔭にて満足なるべし。いまその功徳の一、二を挙げて示すこと左のごとし。

政府の強大にして小民を制圧するの議論は、前編にも記したるゆえここにはこれを略し、まず人間男女の間をもってこれを言わん。そもそも世に生まれたる者は、男も人なり女も人なり。この世に欠くべからざる用をなすところをもって言えば、天下一日も男なかるべからず、また女なかるべからず。その功能いかにも同様なれども、ただその異なるところは、男は強く女は弱し。大の男の力にて女と闘わば必ずこれに勝つべし。すなわちこれ男女の同じからざるところなり。いま世間を見るに、力ずくにて人の物を奪うか、または人を恥ずかしむる者あれば、これを罪人と名づけて刑にも行なわるることあり。しかるに家の内にては公然と人を恥ずかしめ、かつてこれを咎むる者なきはなんぞや。

『女大学』という書に、「婦人に三従の道あり、稚き時は父母に従い、嫁いる時は夫に従い、老いては子に従うべし」と言えり。稚き時に父母に従うは尤もなれども、嫁いりて後に夫に従うとはいかにしてこれに従うことなるや、その従うさまを問わざるべからず。『女大学』の文によれば、亭主は酒を飲み、女郎に耽り、妻をののしり子を叱りて、放蕩淫乱を尽くすも、婦人はこれに従い、この淫夫を天のごとく敬い尊み、顔色を和らげ、悦ばしき言葉にてこれを意見すべしとのみありて、その先の始末をば記さず。さればこの教えの趣意は、淫夫にても姦夫にてもすでに己が夫と約束したるうえは、いかなる恥辱を蒙るもこれに従わざるをえず、ただ心にも思わぬ顔色を作りて諫むるの権義あるのみ。その諫めに従うと従わざるとは淫夫の心次第にて、すなわち淫夫の心はこれを天命と思うよりほかに手段あることなし。

仏書に罪業深き女人ということあり。実にこの有様を見れば、女は生まれながら大罪を犯したる科人に異ならず。また一方より婦人を責むることはなはだしく、『女大学』に婦人の七去とて、「淫乱なれば去る」と明らかにその裁判を記せり。男子のためには大いに便利なり。あまり片落ちなる教えならずや。畢竟、男子は強く婦人は弱しというところより、腕の力を本にして男女上下の名分を立てたる教えなるべし。

右は姦夫淫婦の話なれども、またここに妾の議論あり。世に生まるる男女の数は同様なる理なり。西洋人の実験によれば、男子の生まるることは女子よりも多く、男子二十二人に女子二十人の割合なりと。されば一夫にて二、三の婦人を娶るはもとより天理に背くこと明白なり。これを禽獣と言うも妨げなし。父をともにし母をともにする者を兄弟と名づけ、父母兄弟ともに住居するところを家と名づく。しかるに今、兄弟、父をともにして母を異にし、一父独立して衆母は群を成せり。これを人類の家と言うべきか。家の字の義を成さず。たといその楼閣は巍々たるも、その宮室は美麗なるも、余が眼をもってこれを見れば人の家にあらず、畜類の小屋と言わざるを得ず。妻妾、家に群居して家内よく熟和するものは、古今いまだその例を聞かず。妾といえども人類の子なり。一時の欲のために人の子を禽獣のごとくに使役し、一家の風俗を乱りて子孫の教育を害し、禍を天下に流して毒を後世に遺すもの、豈これを罪人と言わざるべけんや。

人あるいはいわく、「衆妾を養うもその処置よろしきを得れば人情を害することなし」と。こは夫子みずから言うの言葉なり。もしそれはたして然らば、一婦をして衆夫を養わしめ、これを男妾と名づけて家族第二等親の位にあらしめなば如何。かくのごとくしてよくその家を治め、人間交際の大義に毫も害することなくば、余が喋々の議論をもやめ、口を閉ざしてまた言わざるべし。天下の男子よろしくみずから顧みるべし。或る人またいわく、「妾を養うは後あらしめんがためなり、孟子の教えに不孝に三つあり、後なきを大なりとす」と。余答えていわく、天理に戻ることを唱うる者は孟子にても孔子にても遠慮に及ばず、これを罪人と言いて可なり。妻を娶り、子を生まざればとてこれを大不孝とは何事ぞ。遁辞と言うもあまりはなはだしからずや。

いやしくも人心を具えたる者なれば、誰か孟子の妄言を信ぜん。元来不孝とは、子たる者にて理に背きたることをなし、親の身心をして快からしめざることを言うなり。もちろん老人の心にて孫の生まるるは悦ぶことなれども、孫の誕生が晩しとて、これをその子の不幸と言うべからず。試みに天下の父母たる者に問わん。子に良縁ありてよき嫁を娶り、孫を生まずとてこれを怒り、その嫁を叱り、その子を笞うち、あるいはこれを勘当せんと欲するか。世界広しといえどもいまだかかる奇人あるを聞かず、これらはもとより空論にて弁解を費やすにも及ばず。人々みずからその心に問いてみずからこれに答うべきのみ。

親に孝行するはもとより人たる者の当然、老人とあれば他人にてもこれを丁寧にするはずなり。まして自分の父母に対し情を尽くさざるべけんや。利のためにあらず、名のためにあらず、ただ己が親と思い、天然の誠をもってこれに孝行すべきなり。古来和漢にて孝行を勧めたる話ははなはだ多く、『二十四孝』をはじめとしてそのほかの著述書も計うるに遑あらず。しかるにこの書を見れば、十に八、九は人間にでき難きことを勧むるか、または愚にして笑うべきことを説くか、はなはだしきは理に背きたることを誉めて孝行とするものあり。

寒中に裸体にて氷の上に臥しその解くるを待たんとするも人間にできざることなり。夏の夜に自分の身に酒を灌ぎて蚊に食われ親に近づく蚊を防ぐより、その酒の代をもって紙帳を買うこそ智者ならずや。父母を養うべき働きもなく途方に暮れて、罪もなき子を生きながら穴に埋めんとするその心は、鬼とも言うべし、蛇とも言うべし、天理人情を害するの極度と言うべし。最前は不孝に三つありとて、子を生まざるをさえ大不孝と言いながら、今ここにはすでに生まれたる子を穴に埋めて後を絶たんとせり。いずれをもって孝行とするか、前後不都合なる妄説ならずや。畢竟、この孝行の説も、親子の名を糺し上下の分を明らかにせんとして、無理に子を責むるものならん。そのこれを責むる箇条を聞けば、「妊娠中に母を苦しめ、生まれて後は三年父母の懐を免れず、その洪恩は如何」と言えり。されども子を生みて子を養うは人類のみにあらず、禽獣みな然り。ただ人の父母の禽獣に異なるところは、子に衣食を与うるのほかに、これを教育して人間交際の道を知らしむるの一事にあるのみ。

しかるに世間の父母たる者、よく子を生めども子を教うるの道を知らず、身は放蕩無頼を事として子弟に悪例を示し、家を汚し産を破りて貧困に陥り、気力ようやく衰えて家産すでに尽くるに至れば放蕩変じて頑愚となり、すなわちその子に向かいて孝行を責むるとは、はたしてなんの心ぞや。なんの鉄面皮あればこの破廉恥のはなはだしきに至るや。父は子の財を貪らんとし、姑は嫁の心を悩ましめ、父母の心をもって子供夫婦の身を制し、父母の不理屈は尤もにして子供の申し分は少しも立たず、嫁はあたかも餓鬼の地獄に落ちたるがごとく、起居眠食、自由なるものなし。一も舅姑の意に戻ればすなわちこれを不孝者と称し、世間の人もこれを見て心に無理とは思いながら、己が身に引き受けざることなればまず親の不理屈に左袒して理不尽にその子を咎むるか、あるいは通人の説に従えば、理非を分かたず親を欺けとて偽計を授くる者あり。豈これを人間家内の道と言うべけんや。余かつて言えることあり。「姑の鑑遠からず嫁の時にあり」と。姑もし嫁を窘しめんと欲せば、己がかつて嫁たりし時を想うべきなり。

右は上下貴賤の名分より生じたる悪弊にて、夫婦親子の二例を示したるなり。世間にこの悪弊の行なわるるははなはだ広く、事々物々、人間の交際に浸潤せざるはなし。なおその例は次編に記すべし。

\section{九編}
\subsection{学問の旨を二様に記して 中津の旧友に贈る文}
人の心身の働きを細かに見れば、これを分かちて二様に区別すべし。第一は一人たる身につきての働きなり。第二は人間交際の仲間に居り、その交際の身につきての働きなり。

第一 心身の働きをもって衣食住の安楽を致すもの、これを一人の身につきての働きと言う。然りといえども天地間の万物、一として人の便利たらざるものなし。一粒の種を蒔けば二、三百倍の実を生じ、深山の樹木は培養せざるもよく成長し、風はもって車を動かすべし、海はもって運送の便をなすべし、山の石炭を掘り、河海の水を汲み、火を点じて蒸気を造れば重大なる舟車を自由に進退すべし。このほか造化の妙工を計れば枚挙に遑あらず。人はただこの造化の妙工を藉り、わずかにその趣を変じてもってみずから利するなり。ゆえに人間の衣食住を得るは、すでに造化の手をもって九十九分の調理を成したるものへ、人力にて一分を加うるのみのことなれば、人はこの衣食住を造ると言うべからず、その実は路傍に棄てたるものを拾い取るがごときのみ。

ゆえに人としてみずから衣食住を給するは難きことにあらず。この事を成せばとて、あえて誇るべきにあらず。もとより独立の活計は人間の一大事、「汝の額の汗をもって汝の食を食らえ」とは古人の教えなれども、余が考えには、この教えの趣旨を達したればとていまだ人たるものの務めを終われりとするに足らず。この教えはわずかに人をして禽獣に劣ることなからしむるのみ。試みに見よ。禽獣魚虫、みずから食を得ざるものなし。ただにこれを得て一時の満足を取るのみならず、蟻のごときははるかに未来を図り、穴を掘りて居処を作り、冬日の用意に食料を貯うるにあらずや。

しかるに世の中にはこの蟻の所業をもってみずから満足する人あり。今その一例を挙げん。男子年長じて、あるいは工につき、あるいは商に帰し、あるいは官員となりて、ようやく親類朋友の厄介たるを免れ、相応に衣食して他人へ不義理の沙汰もなく、借屋にあらざれば自分にて手軽に家を作り、家什はいまだ整わずとも細君だけはまずとりあえずとて、望みのとおりに若き婦人を娶り、身の治まりもつきて倹約を守り、子供は沢山に生まれたれども教育もひととおりのことなればさしたる銭もいらず、不時病気等の入用に三十円か五十円の金にはいつも差しつかえなくして、細く永く長久の策に心配し、とにもかくにも一軒の家を守る者あれば、みずから独立の活計を得たりとて得意の色をなし、世の人もこれを目して不覊独立の人物と言い、過分の働きをなしたる手柄もののように称すれども、その実は大なる間違いならずや。この人はただ蟻の門人と言うべきのみ。生涯の事業は蟻の右に出ずるを得ず。その衣食を求め家を作るの際に当たりては、額に汗を流せしこともあらん、胸に心配せしこともあらん、古人の教えに対して恥ずることなしといえども、その成功を見れば万物の霊たる人の目的を達したる者と言うべからず。

右のごとく一身の衣食住を得てこれに満足すべきものとせば、人間の渡世はただ生まれて死するのみ、その死するときの有様は生まれしときの有様に異ならず。かくのごとくして子孫相伝えなば、幾百代を経るも一村の有様は旧の一村にして、世上に公の工業を起こす者なく、船をも造らず、橋をも架せず、一身一家の外は悉皆天然に任せて、その土地に人間生々の痕跡を遺すことなかるべし。西人言えることあり、「世の人みなみずから満足するを知りて小安に安んぜなば、今日の世界は開闢のときの世界にも異なることなかるべし」と。このことまことに然り。もとより満足に二様の区別ありてその界を誤るべからず。一を得てまた二を欲し、したがって足ればしたがって不足を覚え、ついに飽くことを知らざるものはこれを欲と名づけ、あるいは野心と称すべしといえども、わが心身の働きを拡めて達すべきの目的を達せざるものはこれを蠢愚と言うべきなり。

第二 人の性は群居を好み、けっして独歩孤立するを得ず。夫婦親子にてはいまだこの性情を満足せしむるに足らず、必ずしも広く他人に交わり、その交わりいよいよ広ければ一身の幸福いよいよ大なるを覚ゆるものにて、すなわちこれ人間交際の起こる所以なり。すでに世間に居てその交際中の一人となれば、またしたがってその義務なかるべからず。およそ世に学問と言い、工業と言い、政治と言い、法律と言うも、みな人間交際のためにするものにて、人間の交際あらざればいずれも不用のものたるべし。

政府なんの所以をもって法律を設くるや、悪人を防ぎ善人を保護し、もって人間の交際を全からしめんがためなり。学者なんの所以をもって書を著述し、人を教育するや。後進の智見を導きて、もって人間の交際を保たんがためなり。往古或る支那人の言に、「天下を治むること肉を分かつがごとく公平ならん」と言い、「また庭前の草を除くよりも天下を掃除せん」と言いしも、みな人間交際のために益をなさんとするの志を述べたるものにて、およそ何人にてもいささか身に所得あればこれによりて世の益をなさんと欲するは人情の常なり。あるいは自分には世のためにするの意なきも、知らず識らずして後世、子孫みずからその功徳を蒙ることあり。人にこの性情あればこそ人間交際の義務を達し得るなり。

古より世にかかる人物なかりせば、わが輩今日に生まれて今の世界中にある文明の徳沢を蒙るを得ざるべし。親の身代を譲り受くればこれを遺物と名づくといえども、この遺物はわずかに地面、家財等のみにて、これを失えば失うて跡なかるべし。世の文明はすなわち然らず。世界中の古人を一体にみなし、この一体の古人より今の世界中の人なるわが輩へ譲り渡したる遺物なれば、その洪大なること地面、家財の類にあらず。されども今、誰に向かいて現にこの恩を謝すべき相手を見ず。これを譬えば人生に必要なる日光、空気を得るに銭を須いざるがごとし。その物は貴しといえども、所持の主人あらず。ただこれを古人の陰徳恩賜と言うべきのみ。

開闢のはじめには人智いまだ開けず。その有様を形容すれば、あたかも初生の小児にいまだ智識の発生を見ざるもののごとし。譬えば麦を作りてこれを粉にするには、天然の石と石とをもってこれを搗き砕きしことならん。その後或る人の工夫にて二つの石を円く平たき形に作り、その中心に小さき孔を掘りて、一つの石の孔に木か金の心棒をさし、この石を下に据えてその上に一つの石を重ね、下の石の心棒を上の石の孔にはめ、この石と石との間に麦を入れて上の石を回し、その石の重さにて麦を粉にする趣向を設けたることならん。すなわちこれ挽碓なり。古はこの挽碓を人の手にて回すことなりしが、後世に至りては碓の形をもしだいに改め、あるいはこれを水車、風車に仕掛け、あるいは蒸気の力を用うることとなりて、しだいに便利を増したるなり。

何事もこのとおりにて、世の中の有様はしだいに進み、昨日便利とせしものも今日は迂遠となり、去年の新工夫も今年は陳腐に属す。西洋諸国日新の勢いを見るに、電信・蒸気・百般の器械、したがって出ずればしたがって面目を改め、日に月に新奇ならざるはなし。ただに有形の器械のみ新奇なるにあらず、人智いよいよ開くれば交際いよいよ広く、交際いよいよ広ければ人情いよいよ和らぎ、万国公法の説に権を得て、戦争を起こすこと軽率ならず、経済の議論盛んにして政治・商売の風を一変し、学校の制度、著書の体裁、政府の商議、議院の政談、いよいよ改むればいよいよ高く、その至るところの極を期すべからず。試みに西洋文明の歴史を読み、開闢の時より紀元一六〇〇年代に至りて巻を閉ざし、二百年の間を超えて、とみに一八〇〇年代の巻を開きてこれを見ば、誰かその長足の進歩に驚駭せざるものあらんや。ほとんど同国の史記とは信じ難かるべし。然りしこうしてその進歩をなせし所以の本を尋ぬれば、みなこれ古人の遺物、先進の賜なり。

わが日本の文明も、そのはじめは朝鮮・支那より来たり、爾来わが国人の力にて切磋琢磨、もって近世の有様に至り、洋学のごときはその源遠く宝暦年間にあり〔『蘭学事始』という版本を見るべし〕。輓近外国の交際始まりしより、西洋の説ようやく世上に行なわれ、洋学を教うる者あり、洋書を訳する者あり、天下の人心さらに方向を変じて、これがため政府をも改め、諸藩をも廃して、今日の勢いになり、重ねて文明の端を開きしも、これまた古人の遺物、先進の賜と言うべし。

右所論のごとく、古の時代より有力の人物、心身を労して世のために事をなす者少なからず。今この人物の心事を想うに、豈衣食住の饒かなるをもってみずから足れりとする者ならんや。人間交際の義務を重んじて、その志すところけだし高遠にあるなり。今の学者はこの人物より文明の遺物を受けて、まさしく進歩の先鋒に立ちたるものなれば、その進むところに極度あるべからず。今より数十の星霜を経て後の文明の世に至れば、また後人をしてわが輩の徳沢を仰ぐこと、今わが輩が古人を崇むがごとくならしめざるべからず。概してこれを言えば、わが輩の職務は今日この世に居り、わが輩の生々したる痕跡を遺して遠くこれを後世子孫に伝うるの一事にあり。その任また重しと言うべし。

豈ただ数巻の学校本を読み、商となり工となり、小吏となり、年に数百の金を得てわずかに妻子を養いもってみずから満足すべけんや。こはただ他人を害せざるのみ、他人を益する者にあらず。かつ事をなすには時に便不便あり、いやしくも時を得ざれば有力の人物もその力を逞しゅうすること能わず。古今その例少なからず。近くはわが旧里にも俊英の士君子ありしは明らかにわが輩の知るところなり。もとより今の文明の眼をもってこの士君子なる者を評すれば、その言行あるいは方向を誤るもの多しといえども、こは時論の然らしむるところにて、その人の罪にあらず、その実は事をなすの気力に乏しからず。ただ不幸にして時に遇わず、空しく宝を懐にして生涯を渡り、あるいは死しあるいは老し、ついに世上の人をして大いにその徳を蒙らしむるを得ざりしは遺憾と言うべきのみ。

今やすなわち然らず。前にも言えるごとく、西洋の説ようやく行なわれてついに旧政府を倒し諸藩を廃したるは、ただこれを戦争の変動とみなすべからず。文明の功能はわずかに一場の戦争をもってやむべきものにあらず。ゆえにこの変動は戦争の変動にあらず、文明に促されたる人心の変動なれば、かの戦争の変動はすでに七年前にやみてその跡なしといえども、人心の変動は今なお依然たり。およそ物動かざればこれを導くべからず。学問の道を首唱して天下の人心を導き、推してこれを高尚の域に進ましむるには、とくに今の時をもって好機会とし、この機会に逢う者はすなわち今の学者なれば、学者世のために勉強せざるべからず。  以下十編につづく。

\section{十編}
\subsection{前編のつづき、中津の旧友に贈る}
前編に学問の旨を二様に分けてこれを論じ、その議論を概すれば、「人たるものはただ一身一家の衣食を給し、もってみずから満足すべからず、人の天性にはなおこれよりも高き約束あるものなれば、人間交際の仲間に入り、その仲間たる身分をもって世のために勉むるところなかるべからず」との趣意を述べたるなり。

学問するには、その志を高遠にせざるべからず。飯を炊き、風呂の火を焚くも学問なり。天下の事を論ずるもまた学問なり。されども一家の世帯は易くして、天下の経済は難し。およそ世の事物、これを得るに易きものは貴からず。物の貴き所以はこれを得るの手段難ければなり。私に案ずるに、今の学者あるいはその難を棄てて易きにつくの弊あるに似たり。昔封建の世においては、学者あるいは所得あるも、天下の事みなきりつめたる有様にて、その学問を施すべき場所なければ、やむをえずして学びしうえにもまた学問を勉め、その学風はよろしからずといえども、読書に勉強して、その博識なるは今人の及ぶところにあらず。今の学者はすなわち然らず。したがって学べばしたがってこれを実地に施すべし。たとえば洋学生、三年の修業をすればひととおりの歴史・窮理書を知り、すなわち洋学教師と称して学校を開くべし、また人に雇われて教授すべし、あるいは政府に仕えて大いに用いらるべし。なおこれよりも易きことあり。当時流行の訳書を読み、世間に奔走して内外の新聞を聞き、機に投じて官につけば、すなわち厳然たる官員なり。かかる有様をもって風俗を成さば、世の学問はついに高尚の域に進むことなかるべし。筆端少しく卑劣にわたり、学者に向かいて言うべきことにあらずといえども、銭の勘定をもってこれを説かん。学塾に入りて修業するには一年の費え百円に過ぎず、三年の間に三百円の元入れを卸し、すなわち一月に五、七十円の利益を得るは、洋学生の商売なり。かの耳の学問にて官員となる者はこの三百円の元入れをも費やさざれば、その得るところの月給は正味手取りの利益なり。

世間諸商売のうちにかかる割合の大利を得るものあるべきや、高利貸といえどもこれに三舎を譲るべし。もとより物価は世の需要の多寡により高低あるものにて、方今、政府をはじめ諸方にて洋学者流を求むること急なるがため、この相場の景気をも生じたるものなれば、あえてその人を奸なりとて咎むるにあらず、またこれを買う者を愚なりとて謗るにあらず、ただわが輩の存意には、この人をしてなお三、五年の艱苦を忍び真に実学を勉強して後に事につかしめなば、大いに成すこともあらんと思うのみ。かくありてこそ日本全国に分布せる智徳に力を増して、はじめて西洋諸国の文明と鋒を争うの場合に至るべきなり。

今の学者何を目的として学問に従事するや。不覊独立の大義を求むると言い、自主自由の権義を恢復すると言うにあらずや。すでに自由独立と言うときは、その字義の中におのずからまた義務の考えなかるべからず。独立とは一軒の家に住居して他人へ衣食を仰がずとの義のみにあらず。こはただ内の義務なり。なお一歩を進めて外の義務を論ずれば、日本国に居て日本人たる名を恥ずかしめず、国中の人とともに力を尽くし、この日本国をして自由独立の地位を得せしめ、はじめて内外の義務を終わりたりと言うべし。ゆえに一軒の家に居てわずかに衣食する者は、これを一家独立の主人と言うべし、いまだ独立の日本人と言うべからず。

試みに見よ、方今、天下の形勢、文明はその名あれどもいまだその実を見ず、外の形は備われども内の精神は耗し。今のわが海陸軍をもって西洋諸国の兵と戦うべきや、けっして戦うべからず。今のわが学術をもって西洋人に教ゆべきや、けっして教ゆべきものなし。かえってこれを彼に学んでなおその及ばざるを恐るるのみ。外国に留学生あり、内国に雇いの教師あり、政府の省・寮・学校より、諸府諸港に至るまで、大概みな外国人を雇わざるものなし。あるいは私立の会社・学校の類といえども、新たに事を企つるものは必ずまず外国人を雇い、過分の給料を与えてこれに依頼するもの多し。彼の長を取りてわが短を補うとは人の口吻なれども、今の有様を見れば我は悉皆短にして彼は悉皆長なるがごとし。

もとより数百年来の鎖国を開きて、とみに文明の人に交わることなれば、その状あたかも火をもって水に接するがごとく、この交際を平均せしめんがためには、あるいは彼の人物を雇い、あるいは彼の器品を買いて、もって急須の欠を補い、水火相触るるの動乱を鎮静するは必ずやむをえざるの勢いなれば、一時の供給を彼に仰ぐも国の失策と言うべからず。然りといえども、他国の物を仰いで自国の用を便ずるは、もとより永久の計にあらず、ただこれを一時の供給とみなして強いてみずから慰むるのみなれども、その一時なるものはいずれの時に終わるべきや。その供給を他に仰がずしてみずから供するの法はいかがして得べきや。これを期することはなはだ難し。

ただ、今の学者の成業を待ち、この学者をして自国の用を便ぜしむるのほか、さらに手段あるべからず。すなわちこれ学者の身に引き受けたる職分なれば、その責め急なりと言うべし。今わが国内に雇い入れたる外国人は、わが学者未熟なるがゆえにしばらくその名代を勤めしむるものなり。今わが国内に外国の器品を買い入るるは、わが国の工業拙なるがゆえにしばらく銭と交易して用を便ずるものなり。この人を雇いこの品を買うがために金を費やすは、わが学術のいまだ彼に及ばざるがために日本の財貨を外国へ棄つることなり。国のためには惜しむべし。学者の身となりては慚ずべし。かつ人として前途の望みなかるべからず、望みあらざれば世に事を勉むる者なし。明日の幸を望んで今日の不幸をも慰むべし。来年の楽を望んで今年の苦をも忍ぶべし。昔日は世の事物みな旧格に制せられて有志の士といえども望みを養うべき目的なかりしが、今や然らず、この制限を一掃せしより後は、あたかも学者のために新世界を開きしがごとく、天下ところとして事をなすの地位あらざるはなし。

農となり、商となり、学者となり、官員となり、書を著わし、新聞紙を書き、法律を講じ、芸術を学び、工業も起こすべし、議院も開くべし、百般の事業行なうべからざるものなし。しかもこの事業を成し得て、国中の兄弟相鬩ぐにあらず、その智恵の鋒を争うの相手は外国人なり、この智戦に利あればすなわちわが国の地位を高くすべし。これに敗すればわが地位を落とすべし。その望み大にして期するところ明らかなりと言うべし。もとより天下の事を現に施行するには前後緩急あるべしといえども、到底この国に欠くべからざるの事業は、人々の所長によりて今より研究せざるべからず。いやしくも処世の義務を知る者は、この時に当たりてこの事情を傍観するの理なし。学者勉めざるべからず。

これによりて考うれば、今の学者たる者はけっして尋常学校の教育をもって満足すべからず、その志を高遠にして学術の真面目に達し、不覊独立もって他人に依頼せず、あるいは同志の朋友なくば一人にてこの日本国を維持するの気力を養い、もって世のために尽くさざるべからず。余輩もとより和漢の古学者流が人を治むるを知りてみずから修むるを知らざる者を好まず。これを好まざればこそ、この書の初編より人民同権の説を主張し、人々みずからその責めに任じてみずからその力に食むの大切なるを論じたれども、この自力に食むの一事にてはいまだわが学問の趣意を終われりとするに足らず。

これを譬えば、ここに沈湎冒色、放蕩無頼の子弟あらん。これを御するの法いかがすべきや。これを導きて人となさんとするには、まずその飲酒を禁じ遊冶を制し、しかる後に相当の業につかしむることなるべし。その飲酒、遊冶を禁ぜざるの間は、いまだともに家業の事を語るべからず。されども人にして酒色に耽らざればとて、これをその人の徳義と言うべからず。ただ世の害をなさざるのみにて、いまだ無用の長物たるの名は免れ難し。その飲酒、遊冶を禁じたるうえ、またしたがって業につき、身を養い、家に益することありて、はじめて十人並みの少年と言うべきなり。自食の論もまたかくのごとし。

わが国士族以上の人、数千百年の旧習に慣れて、衣食の何ものたるを知らず、富有のよりて来たるところを弁ぜず、傲然みずから無為に食して、これを天然の権義と思い、その状あたかも沈湎冒色、前後を忘却する者のごとし。この時に当たり、この輩の人に告ぐるに何事をもってすべきや。ただ自食の説を唱えて、その酔夢を驚かすのほか手段なかるべし。この流の人に向かいて豈高尚の学を勧むべけんや。世を益するの大義を説くべけんや。たといこれに説き勧むるも、夢中学に入れば、その学問もまた夢中の夢のみ。すなわちこれわが輩がもっぱら自食の説を主張して、いまだ真の学問を勧めざりし所以なり。ゆえにこの説は、あまねく徒食の輩に告ぐるものにて、学者に諭すべき言にあらず。

しかるに聞く、近日中律の旧友、学問につく者のうち、まれには学業いまだ半ばならずして早くすでに生計の道を求むる人ありと。生計もとより軽んずべからず。あるいはその人の才に長短もあることなれば、後来の方向を定むるはまことに可なりといえども、もしこの風を互いに相倣い、ただ生計をこれ争うの勢いに至らば、俊英の少年はその実を未熟に残うの恐れなきにあらず。本人のためにも悲しむべし、天下のためにも惜しむべし。かつ生計難しといえども、よく一家の世帯を計れば、早く一時に銭を取りこれを費やして小安を買わんより、力を労して倹約を守り大成の時を待つに若かず。学問に入らば大いに学問すべし。農たらば大農となれ、商たらば大商となれ。学者小安に安んずるなかれ。粗衣粗食、寒暑を憚らず、米も搗くべし、薪も割るべし。学問は米を搗きながらもできるものなり。人間の食物は西洋料理に限らず、麦飯を食らい味噌汁を啜り、もって文明の事を学ぶべきなり。

\section{十一編}
\subsection{名分をもって偽君子を生ずるの論}
第八編に、上下貴賤の名分よりして夫婦・親子の間に生じたる弊害の例を示し、「その害の及ぶところはこのほかにもなお多し」との次第を記せり。そもそもこの名分のよって起こるところを案ずるに、その形は強大の力をもって小弱を制するの義に相違なしといえども、その本意は必ずしも悪念より生じたるにあらず。畢竟世の中の人をば悉皆愚にして善なるものと思い、これを救い、これを導き、これを教え、これを助け、ひたすら目上の人の命に従いて、かりそめにも自分の了簡を出ださしめず、目上の人はたいてい自分に覚えたる手心にて、よきように取り計らい、一国の政事も、一村の支配も、店の始末も、家の世帯も、上下心を一にして、あたかも世の中の人間交際を親子の間柄のごとくになさんとする趣意なり。

譬えば十歳前後の子供を取り扱うには、もとよりその了簡を出ださしむべきにあらず、たいてい両親の見計らいにて衣食を与え、子供はただ親の言に戻らずしてその指図にさえ従えば、寒き時にはちょうど綿入れの用意あり、腹のへる時にはすでに飯の支度ととのい、飯と着物はあたかも天より降り来たるがごとく、わが思う時刻にその物を得て、何一つの不自由なく安心して家に居るべし。両親は己が身にも易えられぬ愛子なれば、これを教え、これを諭し、これを誉むるも、これを叱るも、みな真の愛情より出でざるはなく、親子の間一体のごとくして、その快きこと譬えん方なし。すなわちこれ親子の交際にして、その際には上下の名分も立ち、かつて差しつかえあることなし。世の名分を主張する人はこの親子の交際をそのまま人間の交際に写し取らんとする考えにて、ずいぶん面白き工夫のようなれども、ここに大なる差しつかえあり。親子の交際はただ智力の熟したる実の父母と十歳ばかりの実の子供との間に行なわるべきのみ。他人の子供に対してはもとより叶い難し。たとい実の子供にても、もはや二十歳以上に至ればしだいにその趣を改めざるを得ず。いわんや年すでに長じて大人となりたる他人と他人との間においてをや。とてもこの流儀にて交際の行なわるべき理なし。いわゆる願うべくして行なわれ難きものとはこのことなり。

さて今、一国と言い、一村と言い、政府と言い、会社と言い、すべて人間の交際と名づくるものはみな大人と大人との仲間なり。他人と他人との付合いなり。この仲間付合いに実の親子の流儀を用いんとするもまた難きにあらずや。されども、たとい実には行なわれ難きことにても、これを行のうてきわめて都合よからんと心に想像するものは、その想像を実に施したく思うもまた人情の常にて、すなわちこれ世に名分なるものの起こりて専制の行なわるる所以なり。ゆえにいわく、名分の本は悪念より生じたるにあらず、想像によりてしいて造りたるものなり。

アジヤ諸国においては、国君のことを民の父母と言い、人民のことを臣子または赤子と言い、政府の仕事を牧民の職と唱えて、支那には地方官のことを何州の牧と名づけたることあり。この牧の字は獣類を養うの義なれば、一州の人民を牛羊のごとくに取り扱うつもりにて、その名目を公然と看板に掛けたるものなり。あまり失礼なる仕方にはあらずや。かく人民を子供のごとく、牛羊のごとく取り扱うといえども、前段にも言えるとおり、そのはじめの本意は必ずしも悪念にあらず、かの実の父母が実の子供を養うがごとき趣向にて、第一番に国君を聖明なるものと定め、賢良方正の士を挙げてこれを輔け、一片の私心なく半点の我欲なく、清きこと水のごとく、直きこと矢のごとく、己が心を推して人に及ぼし、民を撫するに情愛を主とし、饑饉には米を給し、火事には銭を与え、扶助救育して衣食住の安楽を得せしめ、上の徳化は南風の薫ずるがごとく、民のこれに従うは草の靡くがごとく、その柔らかなるは綿のごとく、その無心なるは木石のごとく、上下合体ともに太平を謡わんとするの目論見ならん。実に極楽の有様を模写したるがごとし。

されどもよく事実を考うれば、政府と人民とはもと骨肉の縁あるにあらず、実に他人の付合いなり。他人と他人との付合いには情実を用ゆべからず、必ず規則約束なるものを作り、互いにこれを守りて厘毛の差を争い、双方ともにかえって円く治まるものにて、これすなわち国法の起こりし所以なり。かつ右のごとく、聖明の君と賢良の士と柔順なる民とその注文はあれども、いずれの学校に入れば、かく無疵なる聖賢を造り出だすべきや、なんらの教育を施せばかく結構なる民を得べきや、唐人も周の世以来しきりにここに心配せしことならんが、今日まで一度も注文どおりに治まりたる時はなく、とどのつまりは今のとおりに外国人に押し付けられたるにあらずや。

しかるにこの意味を知らずして、きかぬ薬を再三飲むがごとく、小刀細工の仁政を用い、神ならぬ身の聖賢が、その仁政に無理を調合してしいて御恩を蒙らしめんとし、御恩は変じて迷惑となり、仁政は化して苛法となり、なおも太平を謡わんとするか。謡わんと欲せばひとり謡いて可なり。これを和する者はなかるべし。その目論見こそ迂遠なれ。実に隣ながらも捧腹に堪えざる次第なり。

この風儀はひとり政府のみに限らず、商家にも、学塾にも、宮にも、寺にも行なわれざるところなし。今その一例を挙げて言わん。店中に旦那が一番の物知りにて、元帳を扱う者は旦那一人、したがって番頭あり、手代ありて、おのおのその職分を勤むれども、番頭・手代は商売全体の仕組みを知ることなく、ただ喧しき旦那の指図に任せて、給金も指図次第、仕事も指図次第、商売の損得は元帳を見て知るべからず、朝夕旦那の顔色を窺い、その顔に笑みを含むときは商売の当たり、眉の上に皺をよするときは商売の外れと推量するくらいのことにて、なんの心配もあることなし。

ただ一つの心配は己が預かりの帳面に筆の働きをもって極内の仕事を行なわんとするの一事のみ。鷲に等しき旦那の眼力もそれまでには及び兼ね、律儀一偏の忠助と思いのほかに、駆落ちかまたは頓死のその跡にて帳面を改むれば、洞のごとき大穴をあけ、はじめて人物の頼み難きを歎息するのみ。されどもこは人物の頼み難きにあらず、専制の頼み難きなり。旦那と忠助とは赤の他人の大人にあらずや。その忠助に商売の割合をば約束もせずして、子供のごとくにこれを扱わんとせしは旦那の不了簡と言うべきなり。

右のごとく上下貴賤の名分を正し、ただその名のみを主張して専制の権を行なわんとするの原因よりして、その毒の吹き出すところは人間に流行する欺詐術策の容体なり。この病に罹る者を偽君子と名づく。譬えば封建の世に大名の家来は表向きみな忠臣のつもりにて、その形を見れば君臣上下の名分を正し、辞儀をするにも敷居一筋の内外を争い、亡君の逮夜には精進を守り、若殿の誕生には上下を着し、年頭の祝儀、菩提所の参詣、一人も欠席あることなし。その口吻にいわく、「貧は士の常、尽忠報国」またいわく、「その食を食む者はその事に死す」などと、たいそうらしく言い触らし、すはといわば今にも討死せん勢いにて、ひととおりの者はこれに欺かるべき有様なれども、竊に一方より窺えば、はたして例の偽君子なり。

大名の家来によき役儀を勤むる者あれば、その家に銭のできるは何ゆえぞ。定まりたる家禄と定まりたる役料にて一銭の余財も入るべき理なし。しかるに出入差引きして余りあるははなはだ怪しむべし。いわゆる役得にもせよ、賄賂にもせよ、旦那の物をせしめたるに相違はあらず。そのもっともいちじるしきものを挙げて言えば、普請奉行が大工に割前を促し、会計の役人が出入りの町人より付け届けを取るがごときは、三百諸侯の家にほとんど定式の法のごとし。旦那のためには御馬前に討死さえせんと言いし忠臣義士が、その買物の棒先を切るとはあまり不都合ならずや。金箔付きの偽君子と言うべし。

あるいはまれに正直なる役人ありて賄賂の沙汰も聞こえざれば、前代未聞の名臣とて一藩中の評判なれども、その実はわずかに銭を盗まざるのみ。人に盗心なければとてさまで誉むべきことにあらず。ただ偽君子の群集するその中に十人並みの人が雑るゆえ、格別に目立つまでのことなり。畢竟この偽君子の多きもその本を尋ぬれば古人の妄想にて、世の人民をばみな結構人にして御しやすきものと思い込み、その弊ついに専制抑圧に至り、詰まるところは飼犬に手を噛まるるものなり。返す返すも世の中に頼みなきものは名分なり。毒を流すの大なるものは専制抑圧なり。恐るべきにあらずや。

或る人いわく、「かくのごとく人民不実の悪例のみを挙ぐれば際限もなきことなれども、悉皆然るにもあらず。わが日本は義の国にて、古来義士の身を棄てて君のためにしたる例ははなはだ多し」と。答えていわく、「まことに然り、古来義士なきにあらず、ただその数少なくして算当に合わぬなり。元禄年中は義気の花盛りとも言うべき時代なり。この時に赤穂七万石の内に義士四十七名あり。七万石の領分におよそ七万の人口あるべし。七万の内に四十七あれば、七百万の内には四千七百あるべし。物換わり星移り、人情はしだいに薄く、義気も落花の時節となりたるは、世人の常に言うところにて相違もあらず。ゆえに元禄年中より人の義気に三割を減じて七掛けにすれば、七百万につき三千二百九十の割合なり。今、日本の人口を三千万となし義士の数は一万四千百人なるべし。この人数にて日本国を保護するに足るべきや。三歳の童子にも勘定はできることならん」

右の議論によれば名分は丸つぶれの話なれども、念のためここに一言を足さん。名分とは虚飾の名目を言うなり。虚名とあれば上下貴賤悉皆無用のものなれども、この虚飾の名目と実の職分とを入れ替えにして、職分をさえ守ればこの名分も差しつかえあることなし。すなわち政府は一国の帳場にして、人民を支配するの職分あり。人民は一国の金主にして、国用を給するの職分あり。文官の職分は政法を議定するにあり。武官の職分は命ずるところに赴きて戦うにあり。このほか、学者にも町人にもおのおの定まりたる職分あらざるはなし。

しかるに半解半知の飛び揚がりものが、名分は無用と聞きて、早くすでにその職分を忘れ、人民の地位にいて政府の法を破り、政府の命をもって人民の産業に手を出だし、兵隊が政を議してみずから師を起こし、文官が腕の力に負けて武官の指図に任ずる等のことあらば、これこそ国の大乱ならん。自主自由のなま噛りにて無政無法の騒動なるべし。名分と職分とは文字こそ相似たれ、その趣意はまったく別物なり。学者これを誤り認むることなかれ。

\section{十二編}
\subsection{演説の法を勧むるの説}
演説とは英語にてスピイチと言い、大勢の人を会して説を述べ、席上にてわが思うところを人に伝うるの法なり。わが国には古よりその法あるを聞かず、寺院の説法などはまずこの類なるべし。西洋諸国にては演説の法もっとも盛んにして、政府の議院、学者の集会、商人の会社、市民の寄合いより、冠婚葬祭、開業・開店等の細事に至るまでも、わずかに十数名の人を会することあれば、必ずその会につき、あるいは会したる趣意を述べ、あるいは人々平生の持論を吐き、あるいは即席の思い付きを説きて、衆客に披露するの風なり。この法の大切なるはもとより論を俟たず。譬えば今、世間にて議院などの説あれども、たとい院を開くも第一に説を述ぶるの法あらざれば、議院もその用をなさざるべし。

演説をもって事を述ぶれば、その事柄の大切なると否とはしばらく擱き、ただ口上をもって述ぶるの際におのずから味を生ずるものなり。譬えば文章に記せばさまで意味なきことにても、言葉をもって述ぶればこれを了解すること易くして人を感ぜしむるものあり。古今に名高き名詩名歌というものもこの類にて、この詩歌を尋常の文に訳すれば絶えておもしろき味もなきがごとくなれども、詩歌の法に従いてその体裁を備うれば、限りなき風致を生じて衆心を感動せしむべし。ゆえに一人の意を衆人に伝うるの速やかなると否とは、そのこれを伝うる方法に関することはなはだ大なり。

学問はただ読書の一科にあらずとのことは、すでに人の知るところなれば今これを論弁するに及ばず。学問の要は活用にあるのみ。活用なき学問は無学に等し。在昔或る朱子学の書生、多年江戸に修業して、その学流につき諸大家の説を写し取り、日夜怠らずして数年の間にその写本数百巻を成し、もはや学問も成業したるがゆえに故郷へ帰るべしとて、その身は東海道を下り、写本は葛籠に納めて大回しの船に積み出だせしが、不幸なるかな、遠州洋において難船に及びたり。この災難によりて、かの書生もその身は帰国したれども、学問は悉皆海に流れて心身に付したるものとてはなに一物もあることなく、いわゆる本来無一物にて、その愚はまさしく前日に異なることなかりしという話あり。

今の洋学者にもまたこの懸念なきにあらず。今日都会の学校に入りて読書講論の様子を見れば、これを評して学者と言わざるを得ず。されども今にわかにその原書を取り上げてこれを田舎に放逐することあらば、親戚、朋友に逢うて「わが輩の学問は東京に残し置きたり」と言い訳するなどの奇談もあるべし。

ゆえに学問の本趣意は読書のみにあらずして、精神の働きにあり。この働きを活用して実地に施すにはさまざまの工夫なかるべからず。オブセルウェーションとは事物を視察することなり。リーゾニングとは事物の道理を推究して自分の説を付くることなり。この二ヵ条にてはもとよりいまだ学問の方便を尽くしたりと言うべからず。なおこのほかに書を読まざるべからず、書を著わさざるべからず、人と談話せざるべからず、人に向かいて言を述べざるべからず、この諸件の術を用い尽くしてはじめて学問を勉強する人と言うべし。すなわち視察、推究、読書はもって智見を集め、談話はもって智見を交易し、著書、演説はもって智見を散ずるの術なり。然りしこうしてこの諸術のうちに、あるいは一人の私をもって能くすべきものありといえども、談話と演説とに至りては必ずしも人とともにせざるを得ず。演説会の要用なることもって知るべきなり。

方今わが国民においてもっとも憂うべきはその見識の賤しきことなり。これを導きて高尚の域に進めんとするはもとより今の学者の職分なれば、いやしくもその方便あるを知らば力を尽くしてこれに従事せざるべからず。しかるに学問の道において、談話、演説の大切なるはすでに明白にして、今日これを実に行なう者なきはなんぞや。学者の懶惰と言うべし。人間の事には内外両様の別ありて、両ながらこれを勉めざるべからず。今の学者は内の一方に身を委して、外の務めを知らざる者多し。これを思わざるべからず。私に沈深なるは淵のごとく、人に接して活発なるは飛鳥のごとく、その密なるや内なきがごとく、その豪大なるや外なきがごとくして、はじめて真の学者と称すべきなり。

\subsection{人の品行は高尚ならざるべからざるの論}
前条に「方今わが国においてもっとも憂うべきは人民の見識いまだ高尚ならざるの一事なり」と言えり。人の見識品行は、微妙なる理を談ずるのみにて高尚なるべきにあらず。禅家に悟道などの事ありて、その理すこぶる玄妙なる由なれども、その僧侶の所業を見れば、迂遠にして用に適せず、事実においては漠然としてなんらの見識もなき者に等し。

また人の見識、品行はただ聞見の博きのみにて高尚なるべきにあらず。万巻の書を読み、天下の人に交わり、なお一己の定見なき者あり。古習を墨守する漢儒者のごときこれなり。ただ儒者のみならず、洋学者といえどもこの弊を免れず。いま西洋日新の学に志し、あるいは経済書を読み、あるいは修身論を講じ、あるいは理学、あるいは智学、日夜精神を学問に委ねて、その状あたかも荊棘の上に坐して刺衝に堪ゆべからざるのはずなるに、その人の私につきてこれを見ればけっして然らず、眼に経済書を見て一家の産を営むを知らず、口に修身論を講じて一身の徳を修むるを知らず、その所論とその所行とを比較するときは、まさしく二個の人あるがごとくして、さらに一定の見識あるを見ず。

畢竟この輩の学者といえども、その口に講じ、眼に見るところの事をばあえて非となすにはあらざれども、事物の是を是とするの心と、その是を是としてこれを事実に行なうの心とは、まったく別のものにて、この二つの心なるものあるいは並び行なわるることあり、あるいは並び行なわれざることあり。「医師の不養生」といい、「論語読みの論語知らず」という諺もこれらの謂ならん。ゆえにいわく、人の見識、品行は玄理を談じて高尚なるべきにあらず、また聞見を博くするのみにて、高尚なるべきにあらざるなり。

しからばすなわち、人の見識を高尚にして、その品行を提起するの法いかがすべきや。その要訣は事物の有様を比較して上流に向かい、みずから満足することなきの一事にあり。ただし有様を比較するとはただ一事一物を比較するにあらず、この一体の有様と、かの一体の有様とを並べて、双方の得失を残らず察せざるべからず。譬えば今、少年の生徒、酒色に溺るるの沙汰もなくして謹慎勉強すれば、父兄・長老に咎めらるることなく、あるいは得意の色をなすべきに似たれども、その得色はただ他の無頼生に比較してなすべき得色のみ。謹慎勉強は人類の常なり、これを賞するに足らず、人生の約束は別にまた高きものなかるべからず。広く古今の人物を計え、誰に比較して誰の功業に等しきものをなさばこれに満足すべきや。必ず上流の人物に向かわざるべからず。あるいは我に一得あるも彼に二得あるときは、我はその一得に安んずるの理なし。いわんや後進は先進に優るべき約束なれば、古を空しゅうして比較すべき人物なきにおいてをや。今人の職分は大にして重しと言うべし。

しかるに今わずかに謹慎勉強の一事をもって人類生涯の事となすべきや。思わざるのはなはだしきものなり。人として酒色に溺るる者はこれを非常の怪物と言うべきのみ。この怪物に比較して満足する者は、これを譬えば双眼を具するをもって得意となし、盲人に向かいて誇るがごとし。いたずらに愚を表するに足るのみ。ゆえに酒色云々の談をなして、あるいはこれを論破し、あるいはこれを是非するの間は、到底諸論の賤しきものと言わざるを得ず。人の品行少しく進むときはこれらの醜談はすでにすでに経過し了して、言に発するも人に厭わるるに至るべきはずなり。

方今日本にて学校を評するに、「この学校の風俗はかくのごとし。かの学塾の取締りは云々」とて、世の父兄はもっぱらこの風俗取締りの事に心配せり。そもそも風俗取締りとはなんらの箇条をさして言うか。塾法厳にして生徒の放蕩無頼を防ぐにつき、取締りの行き届きたることを言うならん。これを学問所の美事と称すべきか。余輩はかえってこれを羞ずるなり。西洋諸国の風俗けっして美なるにあらず、あるいはその醜見るに忍びざるもの多しといえども、その国の学校を評するに、風俗の正しきと取締りの行き届きたるとのみによりて名誉を得るものあるを聞かず。

学校の名誉は学科の高尚なると、その教法の巧みなると、その人物の品行高くして、議論の賤しからざるとによるのみ。ゆえに今の学校を支配して今の学校に学ぶ者は、他の賤しき学校に比較せずして、世界中上流の学校を見て得失を弁ぜざるべからず。風俗の美にして取締りの行き届きたるも学校の一得と言うべしといえども、その得は学校たるもののもっとも賤しむべき部分の得なれば、毫もこれを誇るに足らず。上流の学校に比較せんとするには別に勉むるところなかるべからず。ゆえに学校の急務としていわゆる取締りの事を談ずるの間は、たといその取締りはよく行き届くも、けっしてその有様に満足すべからざるなり。

一国の有様をもって論ずるもまたかくのごとし。譬えばここに一政府あらん。賢良方正の士を挙げて政を任し、民の苦楽を察して適宜の処置を施し、信賞必罰、恩威行なわれざるところなく、万民腹を鼓して太平を謡うがごときは、まことに誇るべきに似たり。然りといえども、その賞罰と言い、恩威といい、万民といい、太平というも、悉皆一国内の事なり、一人あるいは数人の意に成りたるものなり。その得失はその国の前代に比較するか、または他の悪政府に比較して誇るべきのみにて、けっしてその国悉皆の有様を詳らかにして他国と相対し、一より十に至るまで比較したるものにあらず。もし一国を全体の一物とみなして他の文明の一国に比較し、数十年の間に行なわるる双方の得失を察して互いに加減乗除し、その実際に見われたるところの損益を論ずることあらば、その誇るところのものはけっして誇るに足らざるものならん。

譬えばインドの国体旧ならざるにあらず、その文物の開けたるは西洋紀元の前数千年にありて、理論の精密にして玄妙なるは、おそらくは今の西洋諸国の理学に比して恥ずるなきもの多かるべし。また在昔トルコの政府も、威権もっとも強盛にして、礼楽征伐の法、斉整ならざるはなし。君長賢明ならざるにあらず、廷臣方正ならざるにあらず。人口の衆多なること兵士の武勇なること近国に比類なくして、一時はその名誉を四方に燿かしたることあり。ゆえにインドとトルコとを評すれば、甲は有名の文国にして、乙は武勇の大国と言わざるを得ず。

しかるに方今この二大国の有様を見るに、インドはすでに英国の所領に帰してその人民は英政府の奴隷に異ならず、今のインド人の業はただ阿片を作りて支那人を毒殺し、ひとり英商をしてその間に毒薬売買の利を得せしむるのみ。トルコの政府も名は独立と言うといえども、商売の権は英仏の人に占められ、自由貿易の功徳をもって国の物産は日に衰微し、機を織る者もなく、器械を製する者もなく、額に汗して土地を耕すか、または手を袖にしていたずらに日月を消するのみにて、いっさいの製作品は英仏の輸入を仰ぎ、また国の経済を治むるに由なく、さすがに武勇なる兵士も貧乏に制せられて用をなさずと言う。

右のごとく、インドの文も、トルコの武も、かつてその国の文明に益せざるはなんぞや。その人民の所見わずかに一国内にとどまり、自国の有様に満足し、その有様の一部分をもって他国に比較し、その間に優劣なきを見てこれに欺かれ、議論もここに止まり、徒党もここに止まり、勝敗栄辱ともに他の有様の全体を目的とすることを知らずして、万民太平を謡うか、または兄弟墻に鬩ぐのその間に、商売の権威に圧しられて国を失うたるものなり。洋商の向かうところはアジヤに敵なし。恐れざるべからず。もしこの勁敵を恐れて、兼ねてまたその国の文明を慕うことあらば、よく内外の有様を比較して勉むるところなかるべからず。

\section{十三編}
\subsection{怨望の人間に害あるを論ず}
およそ人間に不徳の筒条多しといえども、その交際に害あるものは怨望より大なるはなし。貪吝、奢侈、誹謗の類はいずれも不徳のいちじるしきものなれども、よくこれを吟味すれば、その働きの素質において不善なるにあらず。これを施すべき場所柄と、その強弱の度と、その向かうところの方角とによりて、不徳の名を免るることあり。譬えば銭を好んで飽くことを知らざるを貪吝と言う。されども銭を好むは人の天性なれば、その天性に従いて十分にこれを満足せしめんとするもけっして咎むべきにあらず。ただ理外の銭を得んとしてその場所を誤り、銭を好むの心に限度なくして理の外に出で、銭を求むるの方向に迷うて理に反するときは、これを貪吝の不徳と名づくるのみ。ゆえに銭を好む心の働きを見て、直ちに不徳の名をくだすべからず。その徳と不徳との分界には一片の道理なるものありて、この分界の内にあるものはすなわちこれを節倹と言い、また経済と称して、まさに人間の勉むべき美徳の一ヵ条なり。

奢侈もまたかくのごとし。ただ身の分限を越ゆると否とによりて、徳不徳の名をくだすべきのみ。軽暖を着て安宅に居るを好むは人の性情なり。天理に従いてこの情欲を慰むるに、なんぞこれを不徳と言うべけんや。積んでよく散じ、散じて則を踰えざる者は、人間の美事と称すべきなり。

また誹謗と弁駁とその間に髪を容るべからず。他人に曲を誣うるものを誹謗と言い、他人の惑いを解きてわが真理と思うところを弁ずるものを弁駁と名づく。ゆえに世にいまだ真実無妄の公道を発明せざるの間は、人の議論もまた、いずれを是としていずれを非とすべきやこれを定むべからず。是非いまだ定まらざるの間は仮りに世界の衆論をもって公道となすべしといえども、その衆論のあるところを明らかに知ることはなはだ易からず。ゆえに他人を誹謗する者を目して直ちにこれを不徳者と言うべからず。そのはたして誹謗なるか、または真の弁駁なるかを区別せんとするには、まず世界中の公道を求めざるべからず。

右のほか、驕傲と勇敢と、粗野と率直と、固陋と実着と、浮薄と穎敏と相対するがごとく、いずれもみな働きの場所と、強弱の度と、向かうところの方角とによりて、あるいは不徳ともなるべく、あるいは徳ともなるべきのみ。ひとり働きの素質においてまったく不徳の一方に偏し、場所にも方向にもかかわらずして不善の不善なる者は怨望の一ヵ条なり。怨望は働きの陰なるものにて、進んで取ることなく、他の有様によりて我に不平をいだき、我を顧みずして他人に多を求め、その不平を満足せしむるの術は、我を益するにあらずして他人を損ずるにあり。譬えば他人の幸と我の不幸とを比較して、我に不足するところあれば、わが有様を進めて満足するの法を求めずして、かえって他人を不幸に陥れ、他人の有様を下して、もって彼我の平均をなさんと欲するがごとし。いわゆるこれを悪んでその死を欲するとはこのことなり。ゆえにこの輩の不平を満足せしむれば、世上一般の幸福をば損ずるのみにて少しも益するところあるべからず。

或る人いわく、「欺詐虚言の悪事も、その実質において悪なるものなれば、これを怨望に比していずれか軽重の別あるべからず」と。答えていわく、「まことに然るがごとしといえども、事の原因と事の結果とを区別すれば、おのずから軽重の別なしと言うべからず。欺詐虚言はもとより大悪事たりといえども、必ずしも怨望を生ずるの原因にはあらずして、多くは怨望によりて生じたる結果なり。怨望はあたかも衆悪の母のごとく、人間の悪事これによりて生ずべからざるものなし。疑猜、嫉妬、恐怖、卑怯の類は、みな怨望より生ずるものにて、その内形に見わるるところは、私語、密話、内談、秘計、その外形に破裂するところは、徒党、暗殺、一揆、内乱、秋毫も国に益すことなくして、禍の全国に波及するに至りては主客ともに免るることを得ず。いわゆる公利の費をもって私を逞しゅうするものと言うべし」

怨望の人間交際に害あることかくのごとし。今その原因を尋ぬるに、ただ窮の一事にあり。ただしその窮とは困窮、貧窮等の窮にあらず、人の言路を塞ぎ、人の業作を妨ぐる等のごとく、人類天然の働きを窮せしむることなり。貧窮、困窮をもって怨望の源とせば、天下の貧民は悉皆不平を訴え、富貴はあたかも怨みの府にして、人間の交際は一日も保つべからざるはずなれども、事実においてけっして然らず、いかに貧賤なる者にても、その貧にして賤しき所以の原因を知り、その原因の己が身より生じたることを了解すれば、けっしてみだりに他人を怨望するものにあらず。その証拠はことさらに掲示するに及ばず、今日世界中に貧富・貴賤の差ありて、よく人間の交際を保つを見て、明らかにこれを知るべし。ゆえにいわく、富貴は怨みの府にあらず、貧賤は不平の源にあらざるなり。

これによりて考うれば怨望は貧賤によりて生ずるものにあらず。ただ人類天然の働きを塞ぎて、禍福の来去みな偶然に係るべき地位においてはなはだしく流行するのみ。昔孔子が「女子と小人とは近づけ難し、さてさて困り入りたることかな」とて歎息したることあり。今をもって考うるに、これ夫子みずから事を起こしてみずからその弊害を述べたるものと言うべし。人の心の性は男子も女子も異なるの理なし。また小人とは下人と言うことならんか。下人の腹から出でたる者は必ず下人と定まりたるにあらず。下人も貴人も生まれ落ちたる時の性に異同あらざるはもとより論を俟たず。しかるにこの女子と下人とに限りて取扱いに困るとは何ゆえぞ。平生卑屈の旨をもってあまねく人民に教え、小弱なる婦人・下人の輩を束縛して、その働きに毫も自由を得せしめざるがために、ついに怨望の気風を醸成し、その極度に至りてさすがに孔子さまも歎息せられたることなり。

元来人の性情において働きに自由を得ざれば、その勢い必ず他を怨望せざるを得ず。因果応報の明らかなるは、麦を蒔きて麦の生ずるがごとし。聖人の名を得たる孔夫子がこの理を知らず、別に工夫もなくしていたずらに愚痴をこぼすとはあまりたのもしからぬ話なり。そもそも孔子の時代は明治を去ること二千有余年、野蛮草昧の世の中なれば、教えの趣意もその時代の風俗人情に従い、天下の人心を維持せんがためには、知りてことさらに束縛するの権道なかるべからず。もし孔子をして真の聖人ならしめ、万世の後を洞察するの明識あらしめなば、当時の権道をもって必ず心に慊しとしたることはなかるべし。ゆえに後世の孔子を学ぶ者は、時代の考えを勘定のうちに入れて取捨せざるべからず。二千年前に行なわれたる教えをそのままに、しき写しして明治年間に行なわんとする者は、ともに事物の相場を談ずべからざる人なり。

また近く一例を挙げて示さんに、怨望の流行して交際を害したるものは、わが封建の時代に沢山なる大名の御殿女中をもって最とす。そもそも御殿の大略を言えば、無識無学の婦女子群居して無智無徳の一主人に仕え、勉強をもって賞せらるるにあらず、懶惰によりて罰せらるるにあらず、諫めて叱らるることもあり、諫めずして叱らるることもあり、言うも善し言わざるも善し、詐るも悪し詐らざるも悪し、ただ朝夕の臨機応変にて主人の寵愛を僥倖するのみ。

その状あたかも的なきに射るがごとく、当たるも巧なるにあらず、当たらざるも拙なるにあらず、まさにこれを人間外の一乾坤と言うも可なり。この有様のうちに居れば、喜怒哀楽の心情必ずその性を変じて、他の人間世界に異ならざるを得ず。たまたま朋輩に立身する者あるも、その立身の方法を学ぶに由なければ、ただこれを羨むのみ。これを羨むのあまりにはただこれを嫉むのみ。朋輩を嫉み、主人を怨望するに忙わしければ、なんぞお家のおんためを思うに遑あらん。忠信節義は表向きの挨拶のみにて、その実は畳に油をこぼしても、人の見ぬところなれば拭いもせずに捨て置く流儀となり、はなはだしきは主人の一命にかかる病の時にも、平生、朋輩の睨み合いにからまりて、思うままに看病をもなし得ざる者多し。なお一歩を進めて怨望嫉妬の極度に至りては、毒害の沙汰もまれにはなきにあらず。古来もしこの大悪事につきその数を記したるスタチスチクの表ありて、御殿に行なわれたる毒害の数と、世間に行なわれたる毒害の数とを比較することあらば、御殿に悪事の盛んなること断じて知るべし。怨望の禍豈恐怖すべきにあらずや。

右御殿女中の一例を見ても大抵、世の中の有様は推して知るべし。人間最大の禍は怨望にありて、怨望の源は窮より生ずるものなれば、人の言路は開かざるべからず、人の業作は妨ぐべからず。試みに英亜諸国の有様とわが日本の有様とを比較して、その人間の交際において、いずれかよくかの御殿の趣を脱したるやと問う者あらば、余輩は今の日本を目してまったく御殿に異ならずと言うにはあらざれども、その境界を去るの遠近を論ずれば、日本はなおこれに近く、英亜諸国はこれを去ること遠しと言わざるを得ず。英亜の人民、貪吝驕奢ならざるにあらず、粗野乱暴ならざるにあらず、あるいは詐る者あり、あるいは欺く者ありて、その風俗けっして善美ならずといえども、ただ怨望隠伏の一事に至りては必ずわが国と趣を異にするところあるべし。

今、世の識者に民選議院の説あり、また出版自由の論あり。その得失はしばらく擱き、もともとこの論説の起こる所以を尋ぬるに、識者の所以はけだし今の日本国中をして古の御殿のごとくならしめず、今の人民をして古の御殿女中のごとくならしめず、怨望に易うるに活動をもってし、嫉妬の念を絶ちて相競うの勇気を励まし、禍福譏誉ことごとくみな自力をもってこれを取り、満天下の人をして自業自得ならしめんとするの趣意なるべし。

人民の言路を塞ぎ、その業作を妨ぐるは、もっぱら政府上に関して、にわかにこれを聞けば、ただ政治に限りたる病のごとくなれども、この病は必ずしも政府のみに流行するものにあらず、人民の間にも行なわれて、毒を流すこともっともはなはだしきものなれば、政治のみを改革するもその源を除くべきにあらず。今また数言を巻末に付し、政府のほかにつきてこれを論ずべし。

元来人の性は交わりを好むものなれども、習慣によればかえってこれを嫌うに至るべし。世に変人奇物とて、ことさらに山村僻邑におり世の交際を避くる者あり。これを隠者と名づく。あるいは真の隠者にあらざるも、世間の付合いを好まずして一家に閉居し、俗塵を避くるなどとて得意の色をなす者なきにあらず。この輩の意を察するに、必ずしも政府の所置を嫌うのみにて身を退くるにあらず、その心志怯弱にして物に接するの勇なく、その度量狭小にして人を容るること能わず、人を容るること能わざれば人もまたこれを容れず、彼も一歩を退け我もまた一歩を退け、歩々相遠ざかりてついに異類の者のごとくなり、後には讐敵のごとくなりて、互いに怨望するに至ることあり。世の中に大なる禍と言うべし。

また人間の交際において、相手の人を見ずしてそのなしたる事を見るか、もしくはその人の言を遠方より伝え聞きて、少しくわが意に叶わざるものあれば、必ず同情相憐れむの心をば生ぜずして、かえってこれを忌み嫌うの念を起こし、これを悪んでその実に過ぐること多し。これまた人の天性と習慣とによりて然るものなり。物事の相談に伝言、文通にて整わざるものも直談にて円く治まることあり。また人の常の言に、「実はかくかくのわけなれども、面と向かいてはまさかさようにも」ということあり。すなわちこれ人類の至情にて、堪忍の心のあるところなり。すでに堪忍の心を生ずるときは、情実互いに相通じて怨望嫉妬の念はたちまち消散せざるを得ず。古今に暗殺の例少なからずといえども、余常に言えることあり、「もし好機会ありてその殺すものと殺さるる者とをして数日の間同処に置き、互いに隠すところなくしてその実の心情を吐かしむることあらば、いかなる讐敵にても必ず相和するのみならず、あるいは無二の朋友たることもあるべし」と。

右の次第をもって考うれば、言路を塞ぎ、業作を妨ぐるのことは、ひとり政府のみの病にあらず、全国人民の間に流行するものにて、学者といえども、あるいはこれを免れ難し。人生活発の気力は物に接せざれば生じ難し。自由に言わしめ、自由に働かしめ、富貴も貧賤もただ本人のみずから取るにまかして、他よりこれを妨ぐべからざるなり。

\section{十四編}
\subsection{心事の棚卸し}
人の世を渡る有様を見るに、心に思うよりも案外に悪をなし、心に思うよりも案外に愚を働き、心に企つるよりも案外に功を成さざるものなり。いかなる悪人にても、生涯の間勉強して悪事のみをなさんと思う者はなけれども、物に当たり事に接して、ふと悪念を生じ、わが身みずから悪と知りながら、いろいろに身勝手なる説をつけて、しいてみずから慰むる者あり。またあるいは物事に当たりて行なうときはけっしてこれを悪事と思わず、毫も心に恥ずるところなきのみならず、一心一向に善きことと信じて、他人の異見などあれば、かえってこれを怒り、これを怨むほどにありしことにても、年月を経て後に考うれば、大いにわが不行届きにて心に恥じ入ることあり。

また人の性に智愚強弱の別ありといえども、みずから禽獣の智恵にも叶わぬと思う者はあるべからず。世の中にあるさまざまの仕事を見分けて、この事なれば自分の手にも叶うことと思い、自分相応にこれを引き受くることなれども、その事を行なうの間に、思いのほかに失策多くして最初の目的を誤り、世間にも笑われ、自分にも後悔すること多し。世に功業を企てて誤る者を傍観すれば、実に捧腹にも堪えざるほどの愚を働きたるように見ゆれども、そのこれを企てたる人は必ずしもさまで愚なるにあらず、よくその情実を尋ぬれば、また尤もなる次第あるものなり。畢竟世の事変は活物にて容易にその機変を前知すべからず。これがために智者といえども案外に愚を働くもの多し。

また人の企ては常に大なるものにて、事の難易大小と時日の長短とを比較することはなはだ難し。フランキリン言えることあり、「十分と思いし時も事に当たれば必ず足らざるを覚ゆるものなり」と。この言まことに然り。大工に普請を言いつけ、仕立屋に衣服を注文して、十に八、九は必ずその日限を誤らざる者なし。こは大工・仕立屋のことさらに企てたる不埒にあらず。そのはじめに仕事と時日とを精密に比較せざりしより、はからずも違約に立ち至りたるのみ。さて世間の人は大工・仕立屋に向かいて違約を責むることは珍しからず、これを責むるにまた理屈なきにあらず。大工・仕立屋は常に恐れ入り、旦那はよく道理のわかりたる人物のように見ゆれども、その旦那なる者がみずから自分の請け合いたる仕事につき、はたして日限のとおりに成したることあるや。

田舎の書生、国を出ずるときは、難苦を嘗めて三年のうちに成業とみずから期したる者、よくその心の約束を践みたるや。無理な才覚をして渇望したる原書を求め、三ヵ月の間にこれを読み終わらんと約したる者、はたしてよくその約のごとくしたるや。有志の士君子「某が政府に出ずれば、この事務もかくのごとく処し、かの改革もかくのごとく処し、半年の間に政府の面目を改むべし」とて、再三建白のうえようやく本望を達して出仕の後、はたしてその前日の心事に背かざるや。貧書生が「われに万両の金あれば、明日より日本国中の門並みに学校を設けて家に不学の輩なからしめん」と言う者を、今日良縁によりて三井・鴻ノ池の養子たらしむることあらば、はたしてその言のごとくなるべきや。この類の夢想を計れば枚挙に遑あらず。みな事の難易と時の長短とを比較せずして、時を計ること寛に過ぎ、事を視ること易に過ぎたる罪なり。

また世間に事を企つる人の言を聞くに、「生涯のうち」または「十年のうちにこれを成す」と言う者はもっとも多く、「三年のうち」、「一年のうちに」と言う者はやや少なく、「一月のうち」、あるいは「今日この事を企てて今まさにこれを行なう」と言う者はほとんどまれにして、「十年前に企てたることを今すでに成したり」と言うがごときは余輩いまだその人を見ず。かくのごとく、期限の長き未来を言うときにはたいそうなることを企つるようなれども、その期限ようやく近くして今月今日と迫るに従いて、明らかにその企ての次第を述ぶること能わざるは、畢竟ことを企つるに当たりて時日の長短を勘定に入れざるより生ずる不都合なり。

右所論のごとく、人生の有様は徳義のことにつきても思いのほかに悪事をなし、智恵のことにつきても思いのほかに愚を働き、思いのほかに事業を遂げざるものなり。この不都合を防ぐの方便はさまざまなれども、今ここに人のあまり心づかざる一ヵ条あり。その箇条とはなんぞや。事業の成否得失につき、ときどき自分の胸中に差引きの勘定を立つることなり。商売にて言えば棚卸しの総勘定のごときものこれなり。

およそ商売において、最初より損亡を企つる者あるべからず。まず自分の才力と元金とを顧み、世間の景気を察して事を始め、千状万態の変に応じて、あるいは当たりあるいは外れ、この仕入れに損を蒙りかの売捌きに益を取り、一年または一ヵ月の終わりに総勘定をなすときは、あるいは見込みのとおりに行なわれたることもあり、あるいは大いに相違したることもあり、またあるいは売買繁劇の際にこの品につきては必ず益あることなりと思いしものも、棚卸しにできたる損益平均の表を見れば案に相違して損亡なることあり。あるいは仕入れのときは品物不足と思いしものも、棚卸しのときに残品を見れば、売捌きに案外の時日を費やして、その仕入れかえって多きに過ぎたるものもあり。ゆえに商売に一大緊要なるは平日の帳合いを精密にして、棚卸しの期を誤らざるの一事なり。

他の人事もまたかくのごとし。人間生々の商売は十歳前後人心のできし時よりはじめたるものなれば、平生、智徳事業の帳合いを精密にして、勉めて損亡を引き受けざるように心がけざるべからず。「過ぐる十年の間には何を損し何を益したるや。現今はなんらの商売をなしてその繁盛の有様はいかなるや。今は何品を仕入れていずれの時いずれのところに売り捌くつもりなるや。年来心の店の取締りは行き届きて遊冶懶惰など名のる召使のために穴を明けられたることはなきや。来年も同様の商売にて慥かなる見込みあるべきや。もはや別に智徳を益すべき工夫もなきや」と、諸帳面を点検して棚卸しの総勘定をなすことあらば、過去現在、身の行状につき必ず不都合なることも多かるべし。その一、二を挙ぐれば、「貧は士の常、尽忠報国」などとて、みだりに百姓の米を食い潰して得意の色をなし、今日に至りて事実に困る者は、舶来の小銃あるを知らずして刀剣を仕入れ、一時の利を得て、残品に後悔するがごとし。和漢の古書のみを研究して西洋日新の学を顧みず、古を信じて疑わざりし者は、過ぎたる夏の景気を忘れずして冬の差入りに蚊帷を買い込むがごとし。青年の書生いまだ学問も熟せずしてにわかに小官を求め、一生の間、等外に徘徊するは、半ば仕立てたる衣服を質に入れて流すがごとし。地理、歴史の初歩をも知らず、日用の手紙を書くこともむずかしくして、みだりに高尚の書を読まんとし、開巻五、六葉を見てまた他の書を求むるは、元手なしに商売をはじめて日に業を変ずるがごとし、和漢洋の書を読めども天下国家の形勢を知らず一身一家の生計にも苦しむ者は、算盤を持たずして万屋の商売をなすがごとし。

天下を治むるを知りて身を修むるを知らざる者は、隣家の帳合いに助言して自家に盗賊の入るを知らざるがごとし。口に流行の日新を唱えて心に見るところなくわが一身の何ものたるをも考えざる者は、売品の名を知りて値段を知らざるもののごとし。これらの不都合は現に今の世に珍しからず。その原因は、ただ流れ渡りにこの世を渡りて、かつてその身の有様に注意することなく、生来今日に至るまでわが身は何事をなしたるや。今は何事をなせるや。今後は何事をなすべきや」と、みずからその身を点検せざるの罪なり。ゆえにいわく、商売の有様を明らかにして後日の見込みを定むるものは帳面の総勘定なり、一身の有様を明らかにして後日の方向を立つるものは智徳事業の棚卸しなり。

\subsection{世話の字の義}
世話の字に二つの意味あり、一は「保護」の義なり、一は「命令」の義なり。保護とは人の事につき、傍より番をして防ぎ護り、あるいはこれに財物を与え、あるいはこれがために時を費やし、その人をして利益をも面目をも失わしめざるように世話をすることなり。命令とは人のために考えて、その人の身に便利ならんと思うことを指図し、不便利ならんと思うことには意見を加え、心の丈を尽くして忠告することにて、これまた世話の義なり。

右のごとく世話の字に、保護と指図と両様の義を備えて人の世話をするときは、真によき世話にて世の中は円く治まるべし。譬えば父母の子供におけるがごとく、衣食を与えて保護の世話をすれば、子供は父母の言うことを聞きて指図を受け、親子の間柄に不都合あることなし。また政府にては法律を設けて、国民の生命と面目と私有とを大切に取り扱い、一般の安全を謀りて保護の世話をなし、人民は政府の命令に従いて指図の世話に戻ることあらざれば、公私の間円く治まるべし。

ゆえに保護と指図とは両ながらその至るところをともにし、寸分も境界を誤るべからず。保護の至るところはすなわち指図の及ぶところなり。指図の及ぶところは必ず保護の至るところならざるを得ず。もし然らずしてこの二者の至り及ぶところの度を誤り、わずかに齟齬することあれば、たちまち不都合を生じて禍の原因となるべし。世間にその例少なからず。けだしその所以は、世の人々常に世話の字の義を誤りて、あるいは保護の意味に解し、あるいは指図の意味に解し、ただ一方にのみ偏して文字のまったき義を尽くすことなく、もって大なる間違いに及びたるなり。

譬えば父母の指図を聴かざる道楽息子へみだりに銭を与えて、その遊冶放蕩を逞しゅうせしむるは、保護の世話は行き届きて指図の世話は行なわれざるものなり。子供は謹慎勉強して父母の命に従うといえども、この子供に衣食をも十分に給せずして無学文盲の苦界に陥らしむるは、指図の世話のみをなして保護の世話を怠るものなり。甲は不孝にして乙は不慈なり。ともにこれを人間の悪事と言うべし。

古人の教えに「朋友に屡すれば疎ぜらるる」とあり。そのわけは、「わが忠告をも用いざる朋友に向かいて余計なる深切を尽くし、その気前をも知らずして厚かましく意見をすれば、ついにはかえってあいそつかしとなりて、先の人に嫌われ、あるいは怨まれ、あるいは馬鹿にせられて、事実に益なきゆえ、大概に見計ろうてこちらから寄りつかぬようにすべし」との趣意なり。この趣意もすなわち指図の世話の行き届かぬところには保護の世話をなすべからずということなり。

また昔かたぎに、田舎の老人が旧き本家の系図を持ち出して別家の内を掻きまわし、あるいは銭もなき叔父さまが実家の姪を呼びつけてその家事を指図し、その薄情を責めその不行届きを咎め、はなはだしきに至りては、知らぬ祖父の遺言などとて姪の家の私有を奪い去らんとするがごときは、指図の世話は厚きに過ぎて保護の世話の痕跡もなきものなり。諺にいわゆる「大きにお世話」とはこのことなり。

また世に貧民救助とて、人物の良否を問わず、その貧乏の原因を尋ねず、ただ貧乏の有様を見て米銭を与うることあり。鰥寡孤独、実に頼るところなき者へは救助も尤もなれども、五升の御救米を貰うて三升は酒にして飲む者なきにあらず。禁酒の指図もできずしてみだりに米を与うるは、指図の行き届かずして保護の度を越えたるものなり。諺にいわゆる「大きに御苦労」とはこのことなり。英国などにても救窮の法に困却するはこの一条なりという。

この理を拡めて一国の政治上に論ずれば、人民は租税を出だして政府の入用を給し、その世帯向きを保護するものなり。しかるに専制の政にて、人民の助言をば少しも用いず、またその助言を述ぶべき場所もなきは、これまた保護の一方は達して指図の路は塞がりたるものなり。人民の有様は大きに御苦労なりと言うべし。

この類を求めて例を挙ぐればいちいち計うるに遑あらず。この「世話」の字義は経済論のもっとも大切なる箇条なれば、人間の渡世において、その職業の異同事柄の軽重にかかわらず、常にこれに注意せざるべからず。あるいはこの議論はまったく算盤ずくにて薄情なるに似たれども、薄くすべきところを無理に厚くせんとし、あるいはその実の薄きを顧みずしてその名を厚くせんとし、かえって人間の至情を害して世の交際を苦々しくするがごときは、名を買わんとして実を失うものと言うべし。

右のごとく議論は立てたれども、世人の誤解を恐れて念のためここに数言を付せん。修身道徳の教えにおいてはあるいは経済の法と相戻るがごときものあり。けだし一身の私徳は悉皆天下の経済にさし響くものにあらず、見ず知らずの乞食に銭を投与し、あるいは貧人の憐れむべき者を見れば、その人の来歴をも問わずして多少の財物を給することあり。そのこれを投与しこれを給するはすなわち保護の世話なれども、この保護は指図とともに行なわるるものにあらず、考えの領分を窮屈にしてただ経済上の公をもってこれを論ずれば不都合なるに似たれども、一身の私徳において恵与の心はもっとも貴ぶべく最も好みすべきものなり。譬えば天下に乞食を禁ずるの法はもとより公明正大なるものなれども、人々の私において乞食に物を与えんとするの心は咎むべからず。人間万事算盤を用いて決定すべきものにあらず、ただその用ゆべき場所と用ゆべからざる場所とを区別すること緊要なるのみ。世の学者、経済の公論に酔いて仁恵の私徳を忘るるなかれ。

\section{十五編}
\subsection{事物を疑いて取捨を断ずること}
信の世界に偽詐多く、疑いの世界に真理多し。試みに見よ、世間の愚民、人の言を信じ、人の書を信じ、小説を信じ、風聞を信じ、神仏を信じ、卜筮を信じ、父母の大病に按摩の説を信じて草根木皮を用い、娘の縁談に家相見の指図を信じて良夫を失い、熱病に医師を招かずして念仏を申すは阿弥陀如来を信ずるがためなり。三七日の断食に落命するは不動明王を信ずるがゆえなり。この人民の仲間に行なわるる真理の多寡を問わば、これに答えて多しと言うべからず。真理少なければ偽詐多からざるを得ず。けだしこの人民は事物を信ずといえども、その信は偽を信ずる者なり。ゆえにいわく、「信の世界に偽詐多し」と。

文明の進歩は、天地の間にある有形の物にても、無形の人事にても、その働きの趣を詮索して真実を発明するにあり。西洋諸国の人民が今日の文明に達したるその源を尋ぬれば、疑いの一点より出でざるものなし。ガリレオが天文の旧説を疑いて地動を発明し、ガルハニが蟆の脚の搦するを疑いて動物のエレキを発明し、ニュートンが林檎の落つるを見て重力の理に疑いを起こし、ワットが鉄瓶の湯気を弄んで蒸気の働きに疑いを生じたるがごとく、いずれもみな疑いの路によりて真理の奥に達したるものと言うべし。格物窮理の域を去りて、顧みて人事進歩の有様を見るもまたかくのごとし。売奴法の当否を疑いて天下後世に惨毒の源を絶えたる者は、トーマス・クラレクソンなり。ローマ宗教の妄誕を疑いて教法に一面目を改めたる者はマルチン・ルーザなり。フランスの人民は貴族の跋扈に疑いを起こして騒乱の端を開き、アメリカの州民は英国の成法に疑いを容れて独立の功を成したり。今日においても、西洋の諸大家が日新の説を唱えて人を文明に導くものを見るに、その目的はただ古人の確定して駁すべからざるの論説を駁し、世上に普通にして疑いを容るべからざるの習慣に疑いを容るるにあるのみ。

今の人事において男子は外を務め婦人は内を治むるとてその関係ほとんど天然なるがごとくなれども、スチュアルト・ミルは『婦人論』を著わして、万古一定動かすべからざるのこの習慣を破らんことを試みたり。英国の経済家に自由法を悦ぶ者多くして、これを信ずる輩はあたかももって世界普通の定法のごとくに認むれども、アメリカの学者は保護法を唱えて自国一種の経済論を主張する者あり。一議したがって出ずれば一説したがってこれを駁し、異説争論その極まるところを知るべからず。これをかのアジヤ諸州の人民が、虚誕妄説を軽信して巫蠱神仏に惑溺し、あるいはいわゆる聖賢者の言を聞きて一時にこれに和するのみならず、万世の後に至りてなおその言の範囲を脱すること能わざるものに比すれば、その品行の優劣、心志の勇怯、もとより年を同じゅうして語るべからざるなり。

異説争論の際に事物の真理を求むるは、なお逆風に向かいて舟を行るがごとし。その舟路を右にし、またこれを左にし、浪に激し風に逆らい、数十百里の海を経過するも、その直達の路を計れば、進むことわずかに三、五里に過ぎず。航海にはしばしば順風の便ありといえども、人事においてはけっしてこれなし。人事の進歩して真理に達するの路は、ただ異説争論の際に間切るの一法あるのみ。しこうしてその説論の生ずる源は疑いの一点にありて存するものなり。「疑いの世界に真理多し」とはけだしこの謂なり。

然りといえども、事物の軽々信ずべからざることはたして是ならば、またこれを軽々疑うべからず。この信疑の際につき必ず取捨の明なかるべからず。けだし学問の要はこの明智を明らかにするにあるものならん。わが日本においても、開国以来とみに人心の趣を変じ、政府を改革し、貴族を倒し、学校を起こし、新聞局を開き、鉄道・電信・兵制・工業等、百般の事物一時に旧套を改めたるは、いずれもみな数千百年以来の習慣に疑いを容れ、これを変革せんことを試みて功を奏したるものと言うべし。

然りといえども、わが人民の精神においてこの数千年の習慣に疑いを容れたるその原因を尋ぬれば、はじめて国を開きて西洋諸国に交わり、かの文明の有様を見てその美を信じ、これに倣わんとしてわが旧習に疑いを容れたるものなれば、あたかもこれを自発の疑いと言うべからず。ただ旧を信ずるの信をもって新を信じ、昔日は人心の信、東にありしもの、今日はそこを移して西に転じたるのみにして、その信疑の取捨如何に至りては、はたして適当の明あるを保すべからず。余輩いまだ浅学寡聞、この取捨の疑問に至り、いちいち当否を論じてその箇条を枚挙する能わざるは、もとよりみずから懺悔するところなれども、世事転遷の大勢を察すれば、天下の人心この勢いに乗ぜられて、信ずるものは信に過ぎ、疑うものは疑いに過ぎ、信疑ともにその止まるところの適度を失するものあるは明らかに見るべし。左にその次第を述べん。

東西の人民、風俗を別にし情意を異にし、数千百年の久しき、おのおのその国土に行なわれたる習慣は、たとい利害の明らかなるものといえども、とみにこれを彼に取りて是に移すべからず、いわんやその利害のいまだ詳らかならざるものにおいてをや。これを採用せんとするには千思万慮歳月を積み、ようやくその性質を明らかにして取捨を判断せざるべからず。しかるに近日世上の有様を見るに、いやしくも中人以上の改革者流、あるいは開化先生と称する輩は、口を開けば西洋文明の美を称し、一人これを唱うれば万人これに和し、およそ智識、道徳の教えより治国、経済、衣食住の細事に至るまでも、悉皆西洋の風を慕うてこれに倣わんとせざるものなし。あるいはいまだ西洋の事情につきその一斑をも知らざる者にても、ひたすら旧物を廃棄してただ新をこれ求むるもののごとし。なんぞそれ事物を信ずるの軽々にして、またこれを疑うの粗忽なるや。西洋の文明はわが国の右に出ずること必ず数等ならんといえども、けっして文明の十全なるものにあらず。その欠点を計うれば枚挙に遑あらず。彼の風俗ことごとく美にして信ずべきにあらず、我の習慣ことごとく醜にして疑うべきにあらず。

譬えばここに一少年あらん。学者先生に接してこれに心酔し、その風に倣わんとしてにわかに心事を改め、書籍を買い、文房の具を求めて、日夜机に倚りて勉強するはもとより咎むべきにあらず。これを美事と言うべし。然りといえどもこの少年が先生の風を擬するのあまりに、先生の夜話に耽りて朝寝するの癖をも学び得て、ついに身体の健康を害することあらば、これを智者と言うべきか。けだしこの少年は先生を見て十全の学者と認め、その行状の得失を察せずして悉皆これに倣わんとし、もってこの不幸に陥りたるものなり。

支那の諺に、「西施の顰みに倣う」ということあり。美人の顰みはその顰みの間におのずから趣ありしがゆえにこれに倣いしことなればいまだ深く咎むるに足らずといえども、学者の朝寝になんの趣あるや。朝寝はすなわち朝寝にして、懶惰不養生の悪事なり。人を慕うのあまりにその悪事に倣うとは笑うべきのはなはだしきにあらずや。されども今の世間の開化者流にはこの少年の輩はなはだ少なからず。

仮りに今、東西の風俗習慣を交易して開化先生の評論に付し、その評論の言葉を想像してこれを記さん。西洋人は日に浴湯して日本人の浴湯は一月わずかに一、二次ならば、開化先生これを評して言わん、「文明開化の人民はよく浴湯して皮膚の蒸発を促しもって衛生の法を守れども、不文の日本人はすなわちこの理を知らず」と。日本人は寝屋の内に尿瓶を置きてこれに小便を貯え、あるいは便所より出でて手を洗うことなく、洋人は夜中といえども起きて便所に行き、なんら事故あるも必ず手を洗うの風ならば、論者評して言わん、「開化の人は清潔を貴ぶの風あれども、不開化の人民は不潔の何ものたるを知らず、けだし小児の智識いまだ発生せずして汚潔を弁ずること能わざる者に異ならず、この人民といえどもしだいに進んで文明の域に入らば、ついには西洋の美風に倣うことあるべし」と。洋人は鼻汁を拭うに毎次紙を用いて直ちにこれを投棄し、日本人は紙に代わるに布を用い、したがって洗濯してしたがってまた用うるの風ならば、論者たちまち頓智を運らし、細事を推して経済論の大義に付会して言わん、「資本に乏しき国土においては、人民みずから知らずして節倹の道に従うことあり。日本全国の人民をして鼻紙を用うること西洋人のごとくならしめなば、その国財の幾分を浪費すべきはずなるに、よくその不潔を忍んで布を代用するは、みずから資本の乏しきに迫られて節倹に赴くものと言うべし」と。日本の婦人、その耳に金環を掛け、小腹を束縛して衣裳を飾ることあらば、論者、人身窮理の端を持ち出して顰蹙して言わん、「はなはだしいかな、不開化の人民、理を弁じて天然に従うことを知らざるのみならず、ことさらに肉体を傷つけて耳に荷物を掛け、婦人の体においてもっとも貴要部たる小腹を束ねて蜂の腰のごとくならしめ、もって妊娠の機を妨げ、分娩の危難を増し、その禍の小なるは一家の不幸を致し、大なるは全国の人口生々の源を害するものなり」と。

西洋人は家の内外に錠を用うること少なく、旅中に人足を雇うて荷物を持たしめ、その行李に慥かなる錠前なきものといえども常に物を盗まるることなく、あるいは大工、左官等のごとき職人に命じて普請を請け負わしむるに、約定書の密なるものを用いずして、後日に至り、その約定につき公事訴訟を起こすことまれなれども、日本人は家内の一室ごとに締りを設けて座右の手箱に至るまでも錠を卸し、普請請負いの約定書等には一字一句を争うて紙に記せども、なおかつ物を盗まれ、あるいは違約等の事につき、裁判所に訴うること多き風ならば、論者また歎息していわん。「ありがたきかな耶蘇の聖教、気の毒なるかなパガン外教の人民、日本の人はあたかも盗賊と雑居するがごとし、これをかの西洋諸国自由正直の風俗に比すれば万々同日の論にあらず、実に聖教の行なわるる国土こそ道に遺を拾わずと言うべけれ」と。日本人が煙草を咬み、巻煙草を吹かして、西洋人が煙管を用うることあらば、「日本人は器械の術に乏しくしていまだ煙管の発明もあらず」と言わん。日本人が靴を用いて西洋人が下駄をはくことあらば、「日本人は足の指の用法を知らず」と言わん。味噌も舶来品ならばかくまでに軽蔑を受くることもなからん。豆腐も洋人のテーブルに上らばいっそうの声価を増さん。鰻の蒲焼き、茶碗蒸し等に至りては世界第一美味の飛び切りとて評判を得ることなるべし。

これらの箇条を枚挙すれば際限あることなし。今少しく高尚に進みて宗旨のことに及ばん。四百年前西洋に親鸞上人を生じ、日本にマルチン・ルーザを生じ、上人は西洋に行なわるる仏法を改革して浄土真宗を弘め、ルーザは日本のローマ宗教に敵してプロテスタントの教えを開きたることあらば、論者必ず評して言わん、「宗教の大趣意は衆生済度にありて人を殺すにあらず。いやしくもこの趣意を誤ればその余は見るに足らざるなり。西洋の親鸞上人はよくこの旨を体し、野に臥し、石を枕にし、千辛万苦、生涯の力を尽くしてついにその国の宗教を改革し、今日に至りては全国人民の大半を教化したり。その教化の広大なることかくのごとしといえども、上人の死後、その門徒なる者、宗教の事につき、あえて他宗の人を殺したることなくまた殺されたることもなきは、もっぱら宗徳をもって人を化したるものと言うべし。顧みて日本の有様を見れば、ルーザひとたび世に出でてローマの旧教に敵対したりといえども、ローマの宗徒容易にこれに服するにあらず、旧教は虎のごとく新教は狼のごとく、虎狼相闘い食肉流血、ルーザの死後、宗教のために日本の人民を殺し日本の国財を費やし、師を起こし国を滅ぼしたるその禍は、筆もって記すべからず、口もって語るべからず、殺伐なるかな、野蛮の日本人は、衆生済度の教えをもって生霊を塗炭に陥れ、敵を愛するの宗旨によりて無辜の同類を屠り、今日に至りてその成跡如何を問えば、ルーザの新教はいまだ日本人民の半ばを化すること能わずと言えり。東西の宗教その趣を異にすることかくのごとし。余輩ここに疑いを容るること日すでに久しといえども、いまだその原因の確かなるものを得ず。竊に按ずるに日本の耶蘇教も西洋の仏法も、その性質は同一なれども、野蛮の国土に行なわるればおのずから殺伐の気を促し、文明の国に行なわるればおのずから温厚の風を存するによりて然るものか、あるいは東方の耶蘇教と西方の仏法とは、はじめよりその元素を異にするによりて然るものか、あるいは改革の始祖たる日本のルーザと西洋の親鸞上人とその徳義に優劣ありて然るものか、みだりに浅見をもって臆断すべからず。ただ後世博識家の確説を待つのみ」と。

しからばすなわち今の改革者流が日本の旧習を厭うて西洋の事物を信ずるは、まったく軽信軽疑の譏を免るべきものと言うべからず。いわゆる旧を信ずるの信をもって新を信じ、西洋の文明を慕うのあまりに兼ねてその顰蹙朝寝の癖をも学ぶものと言うべし。なおはなはだしきはいまだ新の信ずべきものを探り得ずして早くすでに旧物を放却し、一身あたかも空虚なるがごとくにして安心立命の地位を失い、これがためついには発狂する者あるに至れり。憐れむべきにあらずや〔医師の話を聞くに、近来は神経病および発狂の病人多しという〕。

西洋の文明もとより慕うべし。これを慕いこれに倣わんとして日もまた足らずといえども、軽々これを信ずるは信ぜざるの優に若かず。彼の富強はまことに羨むべしといえども、その人民の貧富不平均の弊をも兼ねてこれに倣うべからず。日本の租税寛なるにあらざれども、英国の小民が地主に虐せらるるの苦痛を思えば、かえってわが農民の有様を祝せざるべからず。西洋諸国、婦人を重んずるの風は人間世界の一美事なれども、無頼なる細君が跋扈して良人を窘しめ、不順なる娘が父母を軽蔑して醜行を逞しゅうするの俗に心酔すべからず。

されば今の日本に行なわるるところの事物は、はたして今のごとくにしてその当を得たるものか、商売会社の法、今のごとくにして可ならんか、政府の体裁、今のごとくにして可ならんか、教育の制、今のごとくにして可ならんか、著書の風、今のごとくにして可ならんか、しかのみならず、現に余輩学問の法も今日の路に従いて可ならんか、これを思えば百疑並び生じてほとんど暗中に物を探るがごとし。この雑沓混乱の最中にいて、よく東西の事物を比較し、信ずべきを信じ、疑うべきを疑い、取るべきを取り、捨つべきを捨て、信疑取捨そのよろしきを得んとするはまた難きにあらずや。

然りしこうして今この責めに任ずる者は、他なし、ただ一種わが党の学者あるのみ。学者勉めざるべからず。けだしこれを思うはこれを学ぶに若かず。幾多の書を読み、幾多の事物に接し、虚心平気、活眼を開き、もって真実のあるところを求めなば、信疑たちまちところを異にして、昨日の所信は今日の疑団となり、今日の所疑は明日氷解することもあらん。学者勉めざるべからざるなり。

\section{十六編}
\subsection{手近く独立を守ること}
不覊独立の語は近来世間の話にも聞くところなれども、世の中の話にはずいぶん間違いもあるものゆえ、銘々にてよくその趣意を弁えざるべからず。

独立に二様の別あり、一は有形なり、一は無形なり。なお手近く言えば品物につきての独立と、精神につきての独立と、二様に区別あるなり。

品物につきての独立とは、世間の人が銘々に身代を持ち、銘々に家業を勤めて、他人の世話厄介にならぬよう、一身一家内の始末をすることにて、一口に申せば人に物を貰わぬという義なり。

有形の独立は右のごとく目にも見えて弁じやすけれども、無形の精神の独立に至りては、その意味深く、その関係広くして、独立の義に縁なきように思わるることにもこの趣意を存して、これを誤るものはなはだ多し。細事ながら左にその一ヵ条を撮りてこれを述べん。

「一杯、人、酒を呑み、三杯、酒、人を呑む」という諺あり。今この諺を解けば、「酒を好むの欲をもって人の本心を制し、本心をして独立を得せしめず」という義なり。今日世の人々の行状を見るに、本心を制するものは酒のみならず、千状万態の事物ありて本心の独立を妨ぐることはなはだ多し。

この着物に不似合いなりとてかの羽織を作り、この衣裳に不相当なりとてかの煙草入れを買い、衣服すでに備われば屋宅の狭きも不自由となり、屋宅の普請はじめて落成すれば宴席を開かざるもまた不都合なり、鰻飯は西洋料理の媒酌となり、西洋料理は金の時計の手引きとなり、比より彼に移り、一より十に進み、一進また一進、段々限りあることなし。この趣を見れば一家の内には主人なきがごとく、一身の内には精神なきがごとく、物よく人をして物を求めしめ、主人は品物の支配を受けてこれに奴隷使せらるるものと言うべし。

なおこれよりはなはだしきものあり。前の例は品物の支配を受くる者なりといえども、その品物は自家の物なれば、一身一家の内にて奴隷の境界に居るまでのことなれども、ここにまた他人の物に使役せらるるの例あり。かの人がこの洋服を作りたるゆえ我もこれを作ると言い、隣に二階の家を建てたるがゆえにわれは三階を建つると言い、朋友の品物はわが買物の見本となり、同僚の噂咄はわが注文書の腹稿となり、色の黒き大の男が節くれ立ちたるその指に金の指輪はちと不似合いと自分も心に知りながら、これも西洋人の風なりとて無理に了簡を取り直して銭を奮発し、極暑の晩景、浴後には浴衣に団扇と思えども、西洋人の真似なれば我慢を張りて筒袖に汗を流し、ひたすら他人の好尚に同じからんことを心配するのみ。他人の好尚に同じゅうするはなおかつ許すべし。その笑うべきの極度に至りては他人の物を誤り認め、隣りの細君が御召縮緬に純金の簪をと聞きて大いに心を悩まし、急に我もと注文して後によくよく吟味すれば、豈計らんや、隣家の品は綿縮緬に鍍金なりしとぞ。かくのごときは、すなわちわが本心を支配するものは自分の物にあらずまた他人の物にもあらず、煙のごとき夢中の妄想に制せられて、一身一家の世帯は妄想の往来に任ずるものと言うべし。精神独立の有様とは多少の距離あるべし。その距離の遠近は銘々にて測量すべきものなり。

かかる夢中の世渡りに心を労し、身を役し、一年千円の歳入も、一月百円の月給も、遣い果たしてその跡を見ず、不幸にして家産歳入の路を失うか、または月給の縁に離るることあれば、気抜けのごとく、間抜けのごとく、家に残るものは無用の雑物、身に残るものは奢侈の習慣のみ。憐れと言うもなおおろかならずや。産を立つるは一身の独立を求むるの基なりとて心身を労しながら、その家産を処置するの際に、かえって家産のために制せられて独立の精神を失い尽くすとは、まさにこれを求むるの術をもってこれを失うものなり。余輩あえて守銭奴の行状を称誉するにあらざれども、ただ銭を用うるの法を工夫し、銭を制して銭に制せられず、毫も精神の独立を害することなからんを欲するのみ。

\subsection{心事と働きと相当すべきの論}
議論と実業と両ながらそのよろしきを得ざるべからずとのことは、あまねく人の言うところなれども、この言うところなるものもまたただ議論となるのみにして、これを実地に行なう者はなはだ少なし。そもそも議論とは、心に思うところを言に発し、書に記すものなり。あるいはいまだ言と書に発せざれば、これをその人の心事と言い、またはその人の志と言う。ゆえに議論は外物に縁なきものと言うも可なり。畢竟内に存するものなり、自由なるものなり、制限なきものなり。実業とは心に思うところを外に顕わし、外物に接して処置を施すことなり。ゆえに実業には必ず制限なきを得ず、外物に制せられて自由なるを得ざるものなり、古人がこの両様を区別するには、あるいは言と行と言い、あるいは志と功と言えり。また今日俗間にて言うところの説と働きなるものも、すなわちこれなり。

言行齟齬するとは、議論に言うところと実地に行なうところと一様ならずということなり。「功に食ましめて志に食ましめず」とは、「実地の仕事次第によりてこそ物をも与うべけれ、その心になんと思うとも形もなき人の心事をば賞すべからず」との義なり。また俗間に、「某の説はともかくも、元来働きのなき人物なり」とてこれを軽蔑することあり。いずれも議論と実業と相当せざるを咎めたるものならん。

さればこの議論と実業とは寸分も相齟齬せざるよう正しく平均せざるべからざるものなり。今、初学の人の了解に便ならしめんがため、人の心事と働きという二語を用いて、その互いに相助けて平均をなし、もって人間の益を致す所以と、この平均を失うよりして生ずるところの弊害を論ずること左のごとし。

第一 人の働きには、大小軽重の別あり。芝居も人の働きなり、学問も人の働きなり、人力車を挽くも、蒸気船を運用するも、鍬をとりて農業するも、筆を揮いて著述するも、等しく人の働きなれども、役者たるを好まずして学者たるを勤め、車挽きの仲間に入らずして航海の術を学び、百姓の仕事を不満足なりとして著書の業に従事するがごときは、働きの大小軽重を弁別し、軽小を捨てて重大に従うものなり。人間の美事と言うべし。然りしこうして、そのこれを弁別せしむるものはなんぞや。本人の心なり、また志なり。かかる心志ある人を名づけて心事高尚なる人物と言う。ゆえにいわく、人の心事は高尚ならざるべからず、心事高尚ならざれば働きもまた高尚なるを得ざるなり。

第二 人の働きはその難易にかかわらずして、用をなすの大なるものと小なるものとあり。囲碁・将棋等の技芸も易きことにあらず、これらの技芸を研究して工夫を運らすの難きは、天文・地理・器械・数学等の諸件に異ならずといえども、その用をなすの大小に至りてはもとより同日の論にあらず。今この有用無用を明察して有用の方につかしむるものは、すなわち心事の明らかなる人物なり。ゆえにいわく、心事明らかならざれば人の働きをしていたずらに労して功なからしむることあり。

第三 人の働きには規則なかるべからず。その働きをなすに場所と時節とを察せざるべからず。譬えば道徳の説法はありがたきものなれども、宴楽の最中に突然とこれを唱うればいたずらに人の嘲りを取るに足るのみ。書生の激論も時には面白からざるにあらずといえども、親戚児女子団座の席にこれを聞けば発狂人と言わざるを得ず。この場所柄と時節柄とを弁別して規則あらしむるはすなわち心事の明らかなるものなり。人の働きのみ活発にして明智なきは、蒸気に機関なきがごとく、船に楫なきがごとし。ただに益をなさざるのみならずかえって害を致すこと多し。

第四 前の条々は人に働きありて心事の不行届きなる弊害なれども、今これに反し、心事のみ高尚遠大にして事実の働きなきも、またはなはだ不都合なるものなり。心事高大にして働きに乏しき者は、常に不平をいだかざるを得ず。世間の有様を通覧して仕事を求むるに当たり、己が手に叶うことは悉皆己が心事より以下のことなればこれに従事するを好まず、さりとて己が心事を逞しゅうせんとするには実の働きに乏しくしてことに当たるべからず、ここにおいてかその罪を己れに責めずして他を咎め、あるいは「時に遇わず」と言い、あるいは「天命至らず」と言い、あたかも天地の間になすべき仕事なきもののごとくに思い込み、ただ退きて私に煩悶するのみ。口に怨言を発し、面に不平を顕わし、身外みな敵のごとく、天下みな不親切なるがごとし。その心中を形容すれば、かつて人に金を貸さずして返金の遅きを怨むものと言うも可なり。

儒者は己れを知る者なきを憂い、書生は己れを助くる者なきを憂い、役人は立身の手がかりなきを憂い、町人は商売の繁盛せざるを憂い、廃藩の士族は活計の路なきを憂い、非役の華族は己れを敬する者なきを憂い、朝々暮々憂いありて楽あることなし。今日世間にこの類の不平はなはだ多きを覚ゆ。その証を得んと欲せば、日常交際の間によく人の顔色を窺い見て知るべし。言語・容貌、活発にして胸中の快楽外に溢るるがごとき者は、世上にその人はなはだまれなるべし。余輩の実験にては、常に人の憂うるを見て悦ぶを見ず、その面を借用したらば不幸の見舞いなどに至極よろしからんと思わるるものこそ多けれ、気の毒千万なる有様ならずや。もしこれらの人をしておのおのその働きの分限に従いて勤むることあらしめなば、おのずから活発為事の楽地を得て、しだいに事業の進歩をなし、ついには心事と働きと相平均するの場合にも至るべきはずなるに、かつてここに心づかず、働きの位は一におり、心事の位は十にとどまり、一にいて十を望み、十にいて百を求め、これを求めて得ずしていたずらに憂いを買う者と言うべし。これを譬えば石の地蔵に飛脚の魂を入れたるがごとく、中風の患者に神経の穎敏を増したるがごとし。その不平不如意は推して知るべきなり。

また心事高尚にして働きに乏しき者は、人に厭われて孤立することあり。己が働きと他人の働きとを比較すればもとより及ぶべきにあらざれども、己が心事をもって他の働きを見れば、これに満足すべからずして、おのずから私に軽蔑の念なきを得ず。みだりに人を軽蔑する者は、必ずまた人の軽蔑を免るべからず。互いに相不平をいだき、互いに相蔑視して、ついには変人奇物の嘲りを取り、世間に歯すべからざるに至るものなり。今日世の有様を見るにあるいは傲慢不遜にして人に厭わるる者あり、あるいは人に勝つことを欲して人に厭わるる者あり、あるいは人に多を求めて人に厭わるる者あり、あるいは人を誹謗して人に厭わるる者あり。いずれもみな人に対して比較するところを失い、己が高尚なる心事をもって標的となし、これに照らすに他の働きをもってして、その際に恍惚たる想像を造り、もって人に厭わるるの端を開き、ついにみずから人を避けて独歩孤立の苦界に陥る者なり。試みに告ぐ、後進の少年輩、人の仕事を見て心に不満足なりと思わば、みずからその事を執りてこれを試むべし。人の商売を見て拙なりと思わば、みずからその商売に当たりてこれを試むべし。隣家の世帯を見て不取締りと思わば、みずからこれを自家に試むべし。人の著書を評せんと欲せば、みずから筆を執りて書を著わすべし。学者を評せんと欲せば学者たるべし。医者を評せんと欲せば医者たるべし。至大のことより至細のことに至るまで、他人の働きに喙を容れんと欲せば、試みに身をその働きの地位に置きて躬みずから顧みざるべからず。あるいは職業のまったく相異なるものあらば、よくその働きの難易軽重を計り、異類の仕事にてもただ働きと働きとをもって自他の比較をなさば大なる謬りなかるべし。

\section{十七編}
\subsection{人望論}
十人の見るところ、百人の指すところにて、「何某は慥かなる人なり、たのもしき人物なり、この始末を託しても必ず間違いなからん、この仕事を任しても必ず成就することならん」と、あらかじめその人柄を当てにして世上一般より望みをかけらるる人を称して、人望を得る人物という。およそ人間世界に人望の大小軽重はあれども、かりそめにも人に当てにせらるる人にあらざれば、なんの用にも立たぬものなり。その小なるを言えば、十銭の銭を持たせて町使いに遣る者も、十銭だけの人望ありて、十銭だけは人に当てにせらるる人物なり。十銭より一円、一円より千円万円、ついには幾百万円の元金を集めたる銀行の支配人となり、または一府一省の長官となりて、ただに金銭を預かるのみならず、人民の便不便を預かり、その貧富を預かり、その栄辱をも預かることあるものなれば、かかる大任に当たる者は、必ず平生より人望を得て、人に当てにせらるる人にあらざれば、とても事をなすことは叶い難し。

人を当てにせざるはその人を疑えばなり。人を疑えば際限もあらず。目付に目をつけるがために目付を置き、監察を監察するがために監察を命じ、結局なんの取締りにもならずしていたずらに人の気配を損じたるの奇談は、古今にその例はなはだ多し。また三井・大丸の品は正札にて大丈夫なりとて品柄をも改めずしてこれを買い、馬琴の作なれば必ずおもしろしとて、表題ばかりを聞きて注文する者多し。ゆえに三井・大丸の店はますます繁盛し、馬琴の著書はますます流行して、商売にも著述にもはなはだ都合よきことあり。人望を得るの大切なることもって知るべし。

「十六貫目の力量ある者へ十六貫目の物を負わせ、千円の身代ある者へ千円の金を貸すべし」と言うときは、人望も栄名も無用に属し、ただ実物を当てにして事をなすべきようなれども、世の中の人事はかく簡易にして淡泊なるものにあらず、十貫目の力量なき者も坐して数百万貫の物を動かすべし、千円の身代なき者も数十万の金を運用すべし。試みに今、富豪の聞こえある商人の帳場に飛び込み、一時に諸帳面の精算をなさば、出入差引きして幾百幾千円の不足する者あらん。この不足はすなわち身代の零点より以下の不足なるゆえ、無一銭の乞食に劣ること幾百幾千なれども、世人のこれを視ること乞食のごとくせざるはなんぞや。他なし、この商人に人望あればなり。されば人望はもとより力量によりて得べきものにあらず、また身代の富豪なるのみによりて得べきものにもあらず、ただその人の活発なる才智の働きと正直なる本心の徳義とをもってしだいに積んで得べきものなり。

人望は智徳に属すること当然の道理にして、必ず然るべきはずなれども、天下古今の事実においてあるいはその反対を見ること少なからず。藪医者が玄関を広大にして盛んに流行し、売薬師が看板を金にして大いに売り弘め、山師の帳場に空虚なる金箱を据え、学者の書斎に読めぬ原書を飾り、人力車中に新聞紙を読みて宅に帰りて午睡を催す者あり、日曜日の午後に礼拝堂に泣きて月曜日の朝に夫婦喧嘩する者あり。滔々たる天下、真偽雑駁、善悪混同、いずれを是としいずれを非とすべきや。はなはだしきに至りては、人望の属するを見て、本人の不智不徳を卜すべき者なきにあらず。ここにおいてか、やや見識高き士君子は世間に栄誉を求めず、あるいはこれを浮世の虚名なりとして、ことさらに避くる者あるもまた無理ならぬことなり。士君子の心がけにおいて称すべき一ヵ条と言うべし。

然りといえども、およそ世の事物につきその極度の一方のみを論ずれば弊害あらざるものなし。かの士君子が世間の栄誉を求めざるは大いに称すべきに似たれども、そのこれを求むると求めざるとを決するの前に、まず栄誉の性質を詳らかにせざるべからず。その栄誉なるもの、はたして虚名の極度にして、医者の玄関、売薬の看板のごとくならば、もとよりこれを遠ざけ、これを避くべきは論を俟たずといえども、また一方より見れば社会の人事は悉皆虚をもって成るものにあらず。人の智徳はなお花樹のごとく、その栄誉人望はなお花のごとし。花樹を培養して花を開くに、なんぞことさらにこれを避くることをせんや。栄誉の性質を詳らかにせずして、概してこれを投棄せんとするは、花を払いて樹木の所在を隠すがごとし。これを隠してその功用を増すにあらず、あたかも活物を死用するに異ならず、世間のためを謀りて不便利の大なるものと言うべし。

しからばすなわち栄誉人望はこれを求むべきものか。いわく、然り、勉めてこれを求めざるべからず。ただこれを求むるに当たりて分に適すること緊要なるのみ。心身の働きをもって世間の人望を収むるは、米を計りて人に渡すがごとし。升取りの巧みなる者は一斗の米を一斗三合に計り出し、その拙なる者は九升七合に計り込むことあり。余輩のいわゆる分に適するとは、計り出しもなくまた計り込みもなく、まさに一斗の米を一斗に計ることなり。升取りには巧拙あるも、これによりて生ずるところの差はわずかに内外の二、三分なれども、才徳の働きを升取りするに至りてはその差けっして三分にとどまるべからず、巧みなるは正味の二倍三倍にも計り出し、拙なるは半分にも計り込む者あらん。この計り出しの法外なる者は世間に法外なる妨げをなしてもとより悪むべきなれども、しばらくこれを擱き、今ここには正味の働きを計り込む人のために少しく論ずるところあらんとす。

孔子のいわく、「君子は人の己れを知らざるを憂えず、人を知らざるを憂う」と。この教えは当時世間に流行する弊害を矯めんとして述べたる言ならんといえども、後世無気無力の腐儒は、この言葉をまともに受けて、引込み思案にのみ心を凝らし、その悪弊ようやく増長して、ついには奇物変人、無言無情、笑うことも知らず、泣くことも知らざる木の切れのごとき男を崇めて奥ゆかしき先生なぞと称するに至りしは、人間世界の一奇談なり。今この陋しき習俗を脱して活発なる境界に入り、多くの事物に接し博く世人に交わり、人をも知り己れをも知られ、一身に持ち前正味の働きを逞しゅうして、自分のためにし、兼ねて世のためにせんとするには、

第一 言語を学ばざるべからず。文字に記して意を通ずるは、もとより有力なるものにして、文通または著述等の心がけも等閑にすべからざるは無論なれども、近く人に接して、直ちにわが思うところを人に知らしむるには、言葉のほかに有力なるものなし。ゆえに言葉は、なるたけ流暢にして活発ならざるべからず。近来世上に演説会の設けあり。この演説にて有益なる事柄を聞くはもとより利益なれども、このほかに言葉の流暢活発を得るの利益は、演説者も聴聞者もともにするところなり。

また今日不弁なる人の言を聞くに、その言葉の数はなはだ少なくしていかにも不自由なるがごとし。譬えば学校の教師が訳書の講義なぞするときに、「円き水晶の玉」とあれば、わかりきったることと思うゆえか、少しも弁解をなさず、ただむずかしき顔をして子供を睨みつけ、「円き水晶の玉」と言うばかりなれども、もしこの教師が言葉に富みて言い回しのよき人物にして、「円きとは角の取れて団子のようなということ、水晶とは山から掘り出すガラスのようなもので甲州なぞからいくらも出ます。この水晶でこしらえたごろごろする団子のような玉」と解き聞かせたらば、婦人にも子供にも腹の底からよくわかるべきはずなるに、用いて不自由なき言葉を用いずして不自由するは、畢竟演説を学ばざるの罪なり。

あるいは書生が「日本の言語は不便利にして、文章も演説もできぬゆえ、英語を使い英文を用うる」なぞと、取るにも足らぬ馬鹿を言う者あり。按ずるにこの書生は日本に生まれていまだ十分に日本語を用いたることなき男ならん。国の言葉はその国に事物の繁多なる割合に従いて、しだいに増加し、毫も不自由なきはずのものなり。何はさておき今の日本人は今の日本語を巧みに用いて弁舌の上達せんことを勉むべきなり。

第二 顔色容貌を快くして、一見、直ちに人に厭わるることなきを要す。肩をそびやかして諂い笑い、巧言令色、太鼓持ちの媚を献ずるがごとくするはもとより厭うべしといえども、苦虫を噛み潰して熊の胆をすすりたるがごとく、黙して誉められて笑いて損をしたるがごとく、終歳胸痛を患うるがごとく、生涯父母の喪にいるがごとくなるもまたはなはだ厭うべし。顔色容貌の活発愉快なるは人の徳義の一ヵ条にして、人間交際においてもっとも大切なるものなり。人の顔色はなお家の門戸のごとし、広く人に交わりて客来を自由にせんには、まず門戸を開きて入口を洒掃し、とにかくに寄りつきを好くするこそ緊要なれ。

しかるに今、人に交わらんとして顔色を和するに意を用いざるのみならず、かえって偽君子を学んで、ことさらに渋き風を示すは、戸の入口に骸骨をぶら下げて、門の前に棺桶を安置するがごとし。誰かこれに近づく者あらんや。世界中にフランスを文明の源と言い、智識分布の中心と称するも、その由縁を尋ぬれば、国民の挙動常に活発気軽にして言語容貌ともに親しむべく近づくべきの気風あるをもって原因の一ヵ条となせり。

人あるいは言わん、「言語・容貌は人々の天性に存するものなれば勉めてこれを如何ともすべからず、これを論ずるも詰まるところは無益に属するのみ」と。この言あるいは是なるがごとくなれども、人智発育の理を考えなば、その当たらざるを知るべし。およそ人心の働き、これを進めて進まざるものあることなし。その趣は人身の手足を役してその筋を強くするに異ならず。されば言語・容貌も人の心身の働きなれば、これを放却して上達するの理あるべからず。しかるに古来日本国中の習慣において、この大切なる心身の働きを捨てて顧みる者なきは、大なる心得違いにあらずや。ゆえに余輩の望むところは、改めて今日より言語容貌の学問と言うにはあらざれども、この働きを人の徳義の一ヵ条として等閑にすることなく、常に心にとどめて忘れざらんことを欲するのみ。

或る人またいわく、「容貌を快くするとは表を飾ることなり。表を飾るをもって人間交際の要となすときは、ただに容貌顔色のみならず、衣服も飾り飲食も飾り、気に叶わぬ客をも招待して、身分不相応の馳走するなぞ、まったく虚飾をもって人に交わるの弊あらん」と。この言もまた一理あるがごとくなれども、虚飾は交際の弊にしてその本色にあらず。事物の弊害はややもすればその本色に反対するもの多し。「過ぎたるはなお及ばざるがごとし」とは、すなわち弊害と本色と相反対するを評したる語なり。譬えば食物の要は身体を養うにありといえども、これを過食すればかえってその栄養を害するがごとし。栄養は食物の本色なり、過食はその弊害なり。弊害と本色と相反対するものと言うべし。

されば人間交際の要も和して真率なるにあるのみ。その虚飾に流るるものはけっして交際の本色にあらず。およそ世の中に夫婦親子より親しき者はあらず、これを天下の至親と称す。しこうしてこの至親の間を支配するは何ものなるや、ただ和して真率なる丹心あるのみ。表面の虚飾を却け、またこれを掃い、これを却掃し尽くして、はじめて至親の存するものを見るべし。しからばすなわち交際の親睦は、真率のうちに存して、虚飾と並び立つべからざるものなり。

余輩もとより今の人民に向かいて、その交際、親子夫婦のごとくならんことを望むにあらざれども、ただその赴くべきの方向を示すのみ。今日俗間の言に人を評して、あの人は気軽な人と言い、気のおけぬ人と言い、遠慮なき人と言い、さっぱりした人と言い、男らしき人と言い、あるいは多言なれどもほどのよき人と言い、騒々しけれども悪からぬ人と言い、無言なれども親切らしき人と言い、こわいようなれどもあっさりした人と言うがごときは、あたかも家族交際の有様を表わし出して、和して真率なるを称したるものなり。

第三 「道同じからざれば相ともに謀らず」と。世人またこの教えを誤解して、学者は学者、医者は医者、少しくその業を異にすれば相近づくことなし、同塾同窓の懇意にても、塾を巣立ちしたる後に、一人が町人となり一人が役人となれば、千里隔絶、呉越の観をなす者なきにあらず。はなはだしき無分別なり。人に交わらんとするには、ただに旧友を忘れざるのみならず、兼ねてまた新友を求めざるべからず。人類相接せざれば互いにその意を尽くすこと能わず、意を尽くすこと能わざればその人物を知るに由なし。試みに思え、世間の士君子、いったんの偶然に人に遭うて生涯の親友たる者あるにあらずや。十人に遭うて一人の偶然に当たらば、二十人に接して二人の偶然を得べし。人を知り、人に知らるるの始源は、多くこの辺にありて存するものなり。人望栄名なぞの話はしばらく擱き、今日世間に知己朋友の多きは、差し向きの便利にあらずや。先年宮の渡しに同船したる人を、今日銀座の往来に見かけて双方図らず便利を得ることあり。今年出入りの八百屋が、来年奥州街道の旅籠屋にて腹痛の介抱してくれることもあらん。

人類多しといえども、鬼にもあらず蛇にもあらず、ことさらにわれを害せんとする悪敵はなきものなり。恐れはばかるところなく、心事を丸出しにしてさっさと応接すべし。ゆえに交わりを広くするの要は、この心事をなるたけ沢山にして、多芸多能一色に偏せず、さまざまの方向によりて人に接するにあり。あるいは学問をもって接し、あるいは商売によりて交わり、あるいは書画の友あり、あるいは碁・将棋の相手あり、およそ遊冶放蕩の悪事にあらざるより以上のことなれば、友を会するの方便たらざるものなし。あるいはきわめて芸能なき者ならばともに会食するもよし、茶を飲むもよし。なお下りて筋骨の丈夫なる者は腕押し、枕引き、足角力も一席の興として交際の一助たるべし。腕押しと学問とは道同じからずして相ともに謀るべからざるようなれども、世界の土地は広く、人間の交際は繁多にして、三、五尾の鮒が井中に日月を消するとは少しく趣を異にするものなり。人にして人を毛嫌いするなかれ。



\end{document}